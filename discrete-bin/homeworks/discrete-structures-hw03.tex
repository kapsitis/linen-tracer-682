\documentclass[jou]{apa6}
%\documentclass[11pt]{article}
\usepackage{ucs}
\usepackage[utf8x]{inputenc}
\usepackage{changepage}
\usepackage{graphicx}
\usepackage{amsmath}
\usepackage{gensymb}
\usepackage{amssymb}
\usepackage{enumerate}
\usepackage{tabularx}
\usepackage{lipsum}
\usepackage{hyperref}

\oddsidemargin 0.0in
\evensidemargin 0.0in
\textwidth 6.27in
\headheight 1.0in
\topmargin -0.1in
\headheight 0.0in
\headsep 0.0in
\textheight 9.0in

\usepackage{xcolor}

\setlength\parindent{0pt}

\newenvironment{myenv}{\begin{adjustwidth}{0.4in}{0.4in}}{\end{adjustwidth}}
\renewcommand{\abstractname}{Anotācija}
\renewcommand\refname{Atsauces}



\newcounter{alphnum}
\newenvironment{alphlist}{\begin{list}{(\Alph{alphnum})}{\usecounter{alphnum}\setlength{\leftmargin}{2.5em}} \rm}{\end{list}}


%16.3-6

\makeatletter
\let\saved@bibitem\@bibitem
\makeatother

\usepackage{bibentry}

\title{Homework 3}
\author{Discrete Structures}
\affiliation{RBS}

\begin{document}
\maketitle
\noindent
Submit a PDF to the "Homework 3" folder by 2020-03-16.




%\begin{document}

%\thispagestyle{empty}

%\twocolumn
%{\Large Discrete Structures: Homework 3}

%\noindent
%{\bf Due Date:} March 16, 2020. Submit PDF file to the "Homework 3" folder in ORTUS.

\vspace{4pt}
{\bf Problem 1 (Rosen2019, \#33, p.444)} \textendash{} {\em After 6.4.}\\
Prove that if $n$ is a positive integer, then 
${\displaystyle \sum\limits_{k=1}^n k \cdot{} {n \choose k} = n \cdot{} 2^{n-1}}$. 

\vspace{8pt}
{\bf Problem 2 (Rosen2019, \#48, p.456)} \textendash{} {\em After 6.5.}\\
A shelf holds $12$ books in a row. How many ways are there to choose five books so that no two adjacent books 
are chosen?

\vspace{8pt}
{\bf Problem 3 (Rosen2019, \#16, p.464)} \textendash{} {\em After Ch.6.}\\
Show that in any set of $n+1$ positive integers not exceeding $2n$ there must be two 
that are relatively prime.

\vspace{8pt}
{\bf Problem 4 (Rosen2019, \#33, p.465)} \textendash{} {\em After Ch.6.}\\
How many bit strings of length $n$, where $n \geq 4$, contain exactly two 
occurrences of {\tt 01}.

\vspace{8pt}
{\bf Problem 5 (Miller2014, Exercise1.18 \url{https://bit.ly/2TfZErQ})} {\em The Theory and Applications of 
Benford's Law. Steven J. Miller (editor).}\\
Compute the values of this function $f(x) = \left| x^2 \cdot \tan x \right|$ for 
all integers $x \in \{ 1,\ldots,100000 \}$. Record the very first digit that appears in every value $f(x)$.\\
{\bf (A)} What is the ratio of the digit {\tt 1} among these $10^5$ digits (empirical probability)?\\
{\bf (B)} What is the theoretical ratio of the first digit {\tt 1} predicted by the Benford's law? 

{\em Note.} Benford's Law is routinely checked by people who falsify the results of elections 
or otherwise fabricate large amounts of data. 
Generating digits with the uniform random distribution (where each 
digit has the same chance to appear) would create data sets that look highly artificial
when statistically examined.

\vspace{8pt}
{\bf Problem 6 (Rosen2019, \#23, p.503)} \textendash{} {\em After 7.3.}\\
Suppose that $E_1$ and $E_2$ are the events that an incoming mail message contains the words
$w_1$ and $w_2$, respectively. 
Assuming that $E_1$ and $E_2$ are independent events and that $(E_1 \,\mid\, S)$ 
and $(E_2 \,\mid\, S)$ are independent events, 
where $S$ is the event that an incoming message is spam, and that we have 
no prior knowledge regarding whether or not the message is spam, show that


$$p\left( S \,\mid\, E_1 \cap E_2 \right) =$$
$$= \frac{p(E_1 \,\mid\, S) \cdot p(E_2 \,\mid\, S)}{P(E_1 \,\mid\, S) \cdot P(E_2 \,\mid\, S)
+ P(E_1 \,\mid\, \overline{S} ) \cdot P(E_2 \,\mid\, \overline{S} )}.$$

\vspace{8pt}
{\bf Problem 7 (Rosen2019, \#39, p.519)} \textendash{} {\em After 7.4.}\\
Suppose that the number of aluminum cans recycled in a day at a recycling center
is a random variable with an expected value of $50000$ and a variance of $10000$.\\
{\bf (A)} Use Markov's inequality (Exercise 37) to find an upper bound 
on the probability that the center will recycle more than $55000$ cans on a particular day.\\
{\bf (B)} Use Chebyshev's inequality to provide a lower bound on the probability that the center
will recycle $40000$ to $60000$ cans on a certain day.

\vspace{8pt}
{\bf Problem 8 (Rosen2019, \#15, p.522)} \textendash{} {\em After Ch.7.}\\
Suppose that $m$ and $n$ are positive integers. What is the probability that 
a randomly chosen positive integer less than $n$ is not divisible by either $p$ or $q$? 

\vspace{8pt}
{\bf Problem 9 (Rosen2019, \#22, p.523)} \textendash{} {\em After Ch.7.}\\
Suppose that $n$ balls are tossed into $b$ bins so that each 
ball is equally likely to fall into any of the bins and that the tosses are independent.\\
{\bf (A)} Find the probability that a particular ball lands in a specified bin.\\
{\bf (B)} What is the expected number of balls that land in a particular bin.\\
{\bf (C)} What is the expected number of balls tossed until a particular bin contains a ball?\\
{\bf (D)} What is the expected number of balls tossed until all bins contain a ball?\\
{\em Hint:} Let $X_i$ denote the number of tosses required to have a ball land in 
the $i$th bin once $i-1$ bins contain a ball. Find $E(X_i)$ and use the linearity of 
expectations.

\vspace{8pt}
{\bf Problem 10 (Rosen2019, \#30, p.524)} \textendash{} {\em After Ch.7.}\\
Use Chebyshev's inequality to show that the probability that more than $10$ people get the 
correct hat back when a hatcheck person returns hats at random does not exceed $1/100$
no matter how many people check their hats.\\
{\em Hint.} See Example 6, (Rosen2019, p.507) about the random hat assigning experiment 
and Exercise 43, (Rosen2019, p.520) about the fixed elements in a random permutation.



\end{document}



