\documentclass[a4paper]{article}
\usepackage{ucs}
\usepackage[utf8x]{inputenc}
\usepackage{changepage}
\usepackage{graphicx}
\usepackage{amsmath}
\usepackage{gensymb}
\usepackage{amssymb}
\usepackage{enumerate}
\usepackage{tabularx}
\usepackage{lipsum}
\usepackage{amsthm}
\usepackage{thmtools}
\usepackage{xcolor}



%\documentclass[jou]{apa6}
%\usepackage[american]{babel}

%\usepackage{csquotes}
%\usepackage[style=apa,sortcites=true,sorting=nyt,backend=biber]{biblatex}
%\DeclareLanguageMapping{american}{american-apa}
%\addbibresource{bibliography.bib}


%%%%%%%%%%%%%%%%%%%%%%%%%%%%%%%%%%%%%%%%
%% Discrete Structures
%% The start of RBS stuff
%%%%%%%%%%%%%%%%%%%%%%%%%%%%%%%%%%%%%%%%

% Working internal and external links in PDF
\usepackage{hyperref}
% Extra math symbols in LaTeX
\usepackage{amsmath}
\usepackage{gensymb}
\usepackage{amssymb}
% Enumerations with (a), (b), etc.
\usepackage{enumerate}
\usepackage[framemethod=TikZ]{mdframed}
\usepackage{xcolor}

\let\OLDitemize\itemize
\renewcommand\itemize{\OLDitemize\addtolength{\itemsep}{-6pt}}

\usepackage{etoolbox}
\makeatletter
\preto{\@verbatim}{\topsep=3pt \partopsep=3pt }
\makeatother

% These sizes redefine APA for A4 paper size
\oddsidemargin 0.0in
\evensidemargin 0.0in
\textwidth 6.27in
\headheight 1.0in
\topmargin -24pt
\headheight 12pt
\headsep 12pt
\textheight 9.19in

\setlength{\parindent}{0pt}
\setlength{\columnsep}{1cm}



\begin{document}

\twocolumn


\begin{center}
{\Large Vidus eksāmens (Midterm)}
\end{center}


{\bf Termiņš:} 2020.gada 2.novembris, līdz vakaram\\ (23:59:59 EEST).\\
{\bf Iesniegšanas veids:} E-studiju vide.

\vspace{10pt}
{\bf 1.uzdevums(Berouza-Vīlera trans\-for\-mā\-ci\-ja)}\\
{\bf (A)} Izpildīt Berouza-Vīlera transformāciju ie\-va\-des virknei jeb stringam
$\textcolor{blue}{\mathtt{PANAMACANAL\$}}$, izmantojot 
ciklisko permutāciju parasto alfabētisko sakārtojumu 
(algoritmu sk.\ \url{https://bit.ly/37XTbd3}).\\
Veikt transformācijas rezultātam ``move to front'' kodējumu, 
pieņemot, ka alfabēts satur sekojošos burtus (uzskaitītajā secībā): 
$$\mathcal{A}=\{\textcolor{blue}{\mathtt{\$}},\textcolor{blue}{\mathtt{A}},\textcolor{blue}{\mathtt{C}},
\textcolor{blue}{\mathtt{L}},\textcolor{blue}{\mathtt{M}},\textcolor{blue}{\mathtt{N}},\textcolor{blue}{\mathtt{P}} \}$$

{\bf (B)} Izpildīt Berouza-Vīlera transformāciju tam pa\-šam stringam 
$\textcolor{blue}{\mathtt{PANAMACANAL\$}}$, izmantojot inversi leksikogrāfisko sakārtojumu 
(algoritmu sk.\ \url{https://bit.ly/3mIu7e9}, 37.lpp.). 


\vspace{20pt}
{\bf 2.uzdevums (Aritmētiskais kods).}\\
Iekodēt $\textcolor{blue}{\mathtt{PANAMACANAL\$}}$ ar aritmētisko kodu, ja visu septiņu
alfabēta burtu varbūtības ir šādas: 

{\small
\begin{tabular}{|l|l|l|l|l|l|l|} \hline
$\textcolor{blue}{\mathtt{\$}}$ & $\textcolor{blue}{\mathtt{A}}$ & 
$\textcolor{blue}{\mathtt{C}}$ & $\textcolor{blue}{\mathtt{L}}$ &
$\textcolor{blue}{\mathtt{M}}$ & $\textcolor{blue}{\mathtt{N}}$ & $\textcolor{blue}{\mathtt{P}}$ \\ \hline
$1/10$ & $3/10$ & $1/10$ & $1/10$ & $1/10$ & $2/10$ & $1/10$ \\ \hline
\end{tabular}
}

\vspace{5pt}
{\bf (A)} Uzrakstīt pusatvērtu intervālu $I_{12} = [c;d)$, kurš izveidojas pēc visu $12$ burtu iekodēšanas
(var parādīt arī starprezultātus, lai varētu saņemt pozitīvu vērtējumu arī neuzmanības kļūdu gadījumā). 

\vspace{5pt}
{\bf (B)} Uzrakstīt bitu virknīti $\beta = b_1b_2\ldots{}b_k$ ar $k$ bitiem, ka 
skaitlis $u$, kura binārais pieraksts ir $u = 0.b_1b_2\ldots{}b_k$ pieder iekodējamajam intervālam 
$I_{12}$ un arī skaitlis $v = u + \frac{1}{2^k}$ pieder $I_{12}$. Ja tādas virknītes ir 
vairākas, izvēlieties to, kura ir īsākā (kurai $k$ vērtība ir minimālā), lai $[u;v) \subseteq [c;d)$. 

\vspace{5pt}
Piemēram, bitu virknīte $\beta = \textcolor{blue}{\mathtt{010}}$ iekodē intervālu $[u;v) = [1/4;3/8)$, 
bet virknīte $\beta = \textcolor{blue}{\mathtt{01}}$ iekodē intervālu $[u;v) = [1/4; 1/2)$. 
(Atkodēšanas algoritmam palīdz virknes/stringa beigu simbols $\textcolor{blue}{\mathtt{\$}}$, jo 
tiklīdz kā tas ir atkodēts un izvadīts, tad var beigt darbu.)


\vspace{20pt}
{\bf 3.uzdevums (Heminga kodi).}\\
Atrast kļūdas (ja tādas ir) sekojošos Heminga koda ziņojumos:\\
{\bf (A)} $\mathtt{1100000}$ (tas ir $[7,4,1]$-kods, bitu secība no $x_{001}$ līdz $x_{111}$). 

\vspace{5pt}
{\bf (B)} $\mathtt{0110011}$ (tas ir $[7,4,1]$-kods, bitu secība no $x_{001}$ līdz $x_{111}$). 

\vspace{5pt}
{\bf (C)} $\mathtt{001001111000101}$ (tas ir $[15,11,1]$ Heminga kods, bitu secība \textendash{} no $x_{0001}$ līdz
$x_{1111}$ ).



\vspace{20pt}
{\bf 4.uzdevums (Rīda-Solomona kods).}\\
Divi kodētāji kodē to pašu bitu virkni ar Rīda-Solomona kodu:

\vspace{5pt}
{\bf (A)} Pirmais kodētājs griež bitu virkni 
gabaliņos pa $4$ bitiem, katru gabaliņu iekodē par skaitli $a_i$ no $0$ līdz $15$, sakrāj $n$ šādas vērtības 
un tad sūta $n-1$ pakāpes polinoma 
$$p(x) = a_{n-1}x^{n-1} + a_{n-2}x^{n-2} + \ldots + a_1x + a_0$$
vērtības visos $17$ punktos $x \in \{ 0,1,\ldots,16 \}$, veicot aprēķinus atbilstoši galīgā lauka $GF(17)$ 
aritmētikai (tā ir aritmētika pēc moduļa $17$).

\vspace{5pt}
{\bf (B)} Otrais kodētājs griež bitu virkni gabaliņos pa $8$ bitiem, katru iekodē par skaitli $b_i$ no $0$ līdz $255$ 
un tad sūta $n-1$ pakāpes polinoma 
$$q(x) = b_{n-1}x^{n-1} + b_{n-2}x^{n-2} + \ldots + b_1x + b_0$$
vērtības visos $257$ punktos $x \in \{ 0,1,\ldots,256 \}$, vei\-cot aprēķinus atbilstoši galīgā lauka $GF(257)$ 
aritmētikai (tā ir aritmētika pēc pirmskaitļa $257$ moduļa). 

\vspace{5pt}
Abos gadījumos atrast, cik kļūdas var izlabot (at\-ka\-rī\-bā no parametra $n$). Par kļūdu uzskatām situāciju, kad 
attiecīgo polinoma $p(x)$ vai $q(x)$ vērtību saņēmējam neizdevās saņemt (saņēmējam ir zināms, kuras vērtības pa ceļam 
sabojājās, bet viņam nav zināms, kādas bija šīs vērtības). 




\newpage
\vspace{10pt}
{\bf 1.uzdevums.}\\ 
{\bf (A)} Izrakstām cikliskās permutācijas:

\begin{verbatim}
PANAMACANAL$
$PANAMACANAL
L$PANAMACANA
AL$PANAMACAN
NAL$PANAMACA
ANAL$PANAMAC
CANAL$PANAMA
ACANAL$PANAM
MACANAL$PANA
AMACANAL$PAN
NAMACANAL$PA
ANAMACANAL$P
\end{verbatim}

Sakārtojam parastajā alfabētiskajā secībā:

\begin{verbatim}
$PANAMACANAL
ACANAL$PANAM
AL$PANAMACAN
AMACANAL$PAN
ANAL$PANAMAC
ANAMACANAL$P
CANAL$PANAMA
L$PANAMACANA
MACANAL$PANA
NAL$PANAMACA
NAMACANAL$PA
PANAMACANAL$
\end{verbatim}

Pēdējā kolonna ir transformācijas rezultāts:\\ 
$\textcolor{blue}{\mathtt{LMNNCPAAAAA\$}}$.

Pielietojam ``Move to Front'' kodējumu. 
Pašā sākumā alfabēts ir sākotnējā secībā. Iepriekšējā soļa ievadē 
sastaptais burts ikreiz pārceļo uz alfabēta sākumu.

\begin{tabular}{|l|l|l|} \hline
{\bf Ievade} & {\bf Alfabēts} & {\bf Izvade} \\ \hline
L & 
$\{
\textcolor{blue}{\mathtt{\$}},
\textcolor{blue}{\mathtt{A}},
\textcolor{blue}{\mathtt{C}},
\textcolor{blue}{\mathtt{L}},
\textcolor{blue}{\mathtt{M}},
\textcolor{blue}{\mathtt{N}}
,\textcolor{blue}{\mathtt{P}} 
\}$ & 
3 \\ \hline
M & 
$\{
\textcolor{blue}{\mathtt{L}},
\textcolor{blue}{\mathtt{\$}},
\textcolor{blue}{\mathtt{A}},
\textcolor{blue}{\mathtt{C}},
\textcolor{blue}{\mathtt{M}},
\textcolor{blue}{\mathtt{N}},
\textcolor{blue}{\mathtt{P}} 
\}$ & 
4 \\ \hline
N & 
$\{
\textcolor{blue}{\mathtt{M}},
\textcolor{blue}{\mathtt{L}},
\textcolor{blue}{\mathtt{\$}},
\textcolor{blue}{\mathtt{A}},
\textcolor{blue}{\mathtt{C}},
\textcolor{blue}{\mathtt{N}},
\textcolor{blue}{\mathtt{P}} 
\}$ & 
5 \\ \hline
N & 
$\{
\textcolor{blue}{\mathtt{N}},
\textcolor{blue}{\mathtt{M}},
\textcolor{blue}{\mathtt{L}},
\textcolor{blue}{\mathtt{\$}},
\textcolor{blue}{\mathtt{A}},
\textcolor{blue}{\mathtt{C}},
\textcolor{blue}{\mathtt{P}} 
\}$ & 
0 \\ \hline
C & 
$\{
\textcolor{blue}{\mathtt{N}},
\textcolor{blue}{\mathtt{M}},
\textcolor{blue}{\mathtt{L}},
\textcolor{blue}{\mathtt{\$}},
\textcolor{blue}{\mathtt{A}},
\textcolor{blue}{\mathtt{C}},
\textcolor{blue}{\mathtt{P}} 
\}$ & 
5 \\ \hline
P & 
$\{
\textcolor{blue}{\mathtt{N}},
\textcolor{blue}{\mathtt{M}},
\textcolor{blue}{\mathtt{L}},
\textcolor{blue}{\mathtt{\$}},
\textcolor{blue}{\mathtt{A}},
\textcolor{blue}{\mathtt{C}},
\textcolor{blue}{\mathtt{P}} 
\}$ & 
6 \\ \hline
A & 
$\{
\textcolor{blue}{\mathtt{P}}, 
\textcolor{blue}{\mathtt{N}},
\textcolor{blue}{\mathtt{M}},
\textcolor{blue}{\mathtt{L}},
\textcolor{blue}{\mathtt{\$}},
\textcolor{blue}{\mathtt{A}},
\textcolor{blue}{\mathtt{C}}
\}$ & 
5 \\ \hline
A & 
$\{
\textcolor{blue}{\mathtt{A}},
\textcolor{blue}{\mathtt{P}},
\textcolor{blue}{\mathtt{N}},
\textcolor{blue}{\mathtt{M}},
\textcolor{blue}{\mathtt{L}},
\textcolor{blue}{\mathtt{\$}},
\textcolor{blue}{\mathtt{C}}
\}$ & 
0 \\ \hline
A & 
$\{
\textcolor{blue}{\mathtt{A}},
\textcolor{blue}{\mathtt{P}}, 
\textcolor{blue}{\mathtt{N}},
\textcolor{blue}{\mathtt{M}},
\textcolor{blue}{\mathtt{L}},
\textcolor{blue}{\mathtt{\$}},
\textcolor{blue}{\mathtt{C}}
\}$ & 
0 \\ \hline
A & 
$\{
\textcolor{blue}{\mathtt{A}},
\textcolor{blue}{\mathtt{P}}, 
\textcolor{blue}{\mathtt{N}},
\textcolor{blue}{\mathtt{M}},
\textcolor{blue}{\mathtt{L}},
\textcolor{blue}{\mathtt{\$}},
\textcolor{blue}{\mathtt{C}}
\}$ & 
0 \\ \hline
A & 
$\{
\textcolor{blue}{\mathtt{A}},
\textcolor{blue}{\mathtt{P}},
\textcolor{blue}{\mathtt{N}},
\textcolor{blue}{\mathtt{M}},
\textcolor{blue}{\mathtt{L}},
\textcolor{blue}{\mathtt{\$}},
\textcolor{blue}{\mathtt{C}}
\}$ & 
0 \\ \hline
\$ & 
$\{
\textcolor{blue}{\mathtt{A}},
\textcolor{blue}{\mathtt{P}},
\textcolor{blue}{\mathtt{N}},
\textcolor{blue}{\mathtt{M}},
\textcolor{blue}{\mathtt{L}},
\textcolor{blue}{\mathtt{\$}},
\textcolor{blue}{\mathtt{C}}
\}$ & 
5 \\ \hline
\end{tabular}

Tātad pie dotā alfabēta 
$\mathcal{A}=\{\textcolor{blue}{\mathtt{\$}},\textcolor{blue}{\mathtt{A}},\textcolor{blue}{\mathtt{C}},
\textcolor{blue}{\mathtt{L}},\textcolor{blue}{\mathtt{M}},\textcolor{blue}{\mathtt{N}},\textcolor{blue}{\mathtt{P}} \}$
Berouza-Vīlera iekodēšanas rezultāts:\\
$\textcolor{red}{\mathtt{3,4,5,0,5,6,5,0,0,0,0,5}}$.


\newpage
{\bf (B)} Izrakstām cikliskās permutācijas
\begin{verbatim}
PANAMACANAL $
$PANAMACANA L
L$PANAMACAN A
AL$PANAMACA N
NAL$PANAMAC A
ANAL$PANAMA C
CANAL$PANAM A
ACANAL$PANA M
MACANAL$PAN A
AMACANAL$PA N
NAMACANAL$P A
ANAMACANAL$ P
\end{verbatim}

Sakārtojam inversi leksikogrāfiski virknītes (no sākuma līdz priekšpēdējai pozīcijai):

\begin{verbatim}
ANAMACANAL$ P
AL$PANAMACA N
ANAL$PANAMA C
$PANAMACANA L
ACANAL$PANA M
AMACANAL$PA N
NAL$PANAMAC A
PANAMACANAL $
CANAL$PANAM A
L$PANAMACAN A
MACANAL$PAN A
NAMACANAL$P A
\end{verbatim}

Pēdējā kolonna ir transformācijas rezultāts:\\ 
$\textcolor{blue}{\mathtt{PNCLMNA\$AAAA}}$.


\vspace{20pt}
{\bf 2.uzdevums.} 
Izmantojot rekurentas sakarības, atrodam, ka 
$$I_{12} = [0.935721517000; 0.935721517972).$$





\end{document}

