\documentclass[11pt]{article}
\usepackage{ucs}
\usepackage[utf8x]{inputenc}
\usepackage{changepage}
\usepackage{graphicx}
\usepackage{amsmath}
\usepackage{gensymb}
\usepackage{amssymb}
\usepackage{enumerate}
\usepackage{tabularx}
\usepackage{lipsum}
\usepackage{amsthm}
\usepackage{thmtools}


\usepackage{fontspec} % loaded by polyglossia, but included here for transparency 
\usepackage{polyglossia}

\usepackage{xeCJK}
\setCJKmainfont{SimSun}
\setmainlanguage{russian} 
\setotherlanguage{english}

\newfontfamily\cyrillicfont[Script=Cyrillic]{Times New Roman}
\newfontfamily\cyrillicfontsf[Script=Cyrillic]{Arial}
\newfontfamily\cyrillicfonttt[Script=Cyrillic]{Courier New}

\oddsidemargin 0.0in
\evensidemargin 0.0in
\textwidth 6.27in
\headheight 1.0in
\topmargin 0.0in
\headheight 0.0in
\headsep 0.0in
%\textheight 9.69in
\textheight 9.00in
 
\setlength\parindent{0pt}

\newenvironment{myenv}{\begin{adjustwidth}{0.4in}{0.4in}}{\end{adjustwidth}}
\renewcommand{\abstractname}{Anotācija}
\renewcommand\refname{Atsauces}

%\newenvironment{uzdevums}[1][\unskip]{%
%\vspace{3mm}
%\noindent
%\textbf{#1:}
%\noindent}
%{}

% (4;10;12;17)
% (p1.19;5;15;20)

% http://tex.stackexchange.com/questions/196961/thmtools-declaration-for-theorem-and-proof
\declaretheoremstyle[headfont=\normalfont\bfseries,notefont=\mdseries\bfseries,bodyfont = \normalfont,headpunct={:}]{normalhead}
\declaretheorem[name={Uzdevums}, style=normalhead,numberwithin=section]{problem}

%\def\changemargin#1#2{\list{}{\rightmargin#2\leftmargin#1}\item[]}
\def\changemargin#1#2{\list{}\item[]}
\let\endchangemargin=\endlist 


\newcommand{\subf}[2]{%
  {\small\begin{tabular}[t]{@{}c@{}}
  #1\\#2
  \end{tabular}}%
}



\newcounter{alphnum}
\newenvironment{alphlist}{\begin{list}{(\Alph{alphnum})}{\usecounter{alphnum}\setlength{\leftmargin}{2.5em}} \rm}{\end{list}}

\newenvironment{zhtext}{\fontfamily{MS PGothic}\selectfont}{\par}


\makeatletter
\let\saved@bibitem\@bibitem
\makeatother

\usepackage{bibentry}
%\usepackage{hyperref}

\newenvironment{tulkojums}[1][\unskip]{%
\begin{changemargin}{8mm}{8mm}
\fontsize{9}{11}
\selectfont
\textbf{#1:}
}
{ 
\fontsize{12}{14}
\selectfont
\end{changemargin}
}

\setcounter{section}{1}


\begin{document}

\begin{center}
{\Large \bf NMS Izlase junioriem: 1.nodarbība skaitļu teorijā}\\
{\bf Ieteicams izvēlēties un rakstiski noformēt 
5 no 8 uzdevumiem līdz 2019.g. 21.oktobrim.}\\
{Var risināt uz papīra vai iesūtīt elektroniski: "kalvis.apsitis", domēns "gmail.com"}
\end{center}

\vspace{10pt}
{\bf \large 1.nodaļa: Pirmskaitļi un dalāmība}

\begin{problem}
Dota kopa $S = \{ 105,106,\ldots,210 \}$. Noteikt mazāko 
naturālo $n$ vērtību, ka, izvēloties jebkuru $n$ skaitļu 
apakškopu $T$ no kopas $S$, tajā būs vismaz divi skaitļi, kuri nav 
savstarpēji pirmskaitļi. 
\end{problem}

\begin{problem}
Visiem veseliem pozitīviem skaitļiem $m > n$ pierādīt, ka 
$$\mbox{MKD}(m,n) + \mbox{MKD}(m+1,n+1) > \frac{2mn}{\sqrt{m-n}}.$$
\end{problem}

\begin{problem}
Vai eksistē bezgalīga 
stingri augoša naturālu skaitļu virkne $a_1 < a_2 < a_3 <\ldots$, 
ka jebkuram fiksētam naturālam skaitlim $a$ virknē $a_1+a < a_2+a < a_3 + a,\ldots$ 
ir tikai galīgs skaits pirmskaitļu? 
\end{problem}

\begin{problem}
Pierādīt, ka virkne $1,11,111,\ldots$ satur bezgalīgu apakšvirkni, 
kuras katri divi locekļi ir savstarpēji pirmskaitļi.
\end{problem}


\vspace{10pt}
{\bf \large 2.nodaļa: Modulārā aritmētika}

% (p1.19;5;15;20)

\begin{problem}
Atrast visus pirmskaitļus $p$, ka skaitlim $p^2 + 11$ ir tieši 
seši dažādi dalītāji (ieskaitot $1$ un pašu skaitli).
\end{problem}

\begin{problem}
Uz tāfeles sākumā uzrakstīts viens skaitlis: 
$$\underbrace{99\ldots{}99}_{1997\;\text{deviņnieki}}.$$
Vienā gājienā atļauts kādu uz tāfeles esošo skaitli sadalīt 
divos reizinātājos; tad katru no reizinātājiem (neatkarīgi vienu no otra) 
palielināt vai samazināt par $2$ un uzrakstīt šos rezultātus uz tāfeles.\\
Vai pēc kāda no gājieniem uz tāfeles var būt skaitļi, kas visi vienādi ar $9$?
\end{problem}

\begin{problem}
Trijstūra malu garumi ir $k$, $m$ un $n$. Pieņemsim, ka $k > m > n$ un 
$$\left\{ \frac{3^k}{10^4} \right\} = \left\{ \frac{3^m}{10^4} \right\} = \left\{ \frac{3^n}{10^4} \right\}.$$
Noteikt trijstūra perimetra mazāko iespējamo vērtību.\\ 
{\em Piezīme.} Ar $\{ x \} = x - \lfloor x \rfloor$ apzīmē skaitļa $x$ daļveida daļu - 
starpību starp skaitli $x$ un tā veselo daļu (lielāko veselo 
skaitli, kas nepārsniedz $x$).
\end{problem}

\begin{problem}
\begin{enumerate}[(a)]
\item Pierādīt, ka no $39$ pēc kārtas sekojošiem naturāliem skaitļiem atradīsies skaitlis, 
kura ciparu summa dalās ar $11$.
\item Atrast mazākos $38$ pēc kārtas sekojošus naturālus skaitļus, kuriem nevienam 
ciparu summa nedalās ar $11$.
\end{enumerate}
\end{problem}

\end{document}


