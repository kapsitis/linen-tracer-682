\documentclass[a4paper]{article}
\usepackage{ucs}
\usepackage[utf8x]{inputenc}
\usepackage{changepage}
\usepackage{graphicx}
\usepackage{amsmath}
\usepackage{gensymb}
\usepackage{amssymb}
\usepackage{enumerate}
\usepackage{tabularx}
\usepackage{lipsum}
\usepackage{amsthm}
\usepackage{thmtools}
\usepackage{xcolor}
\usepackage{caption}
\captionsetup[table]{name=Tabula}


%\documentclass[jou]{apa6}
%\usepackage[american]{babel}

%\usepackage{csquotes}
%\usepackage[style=apa,sortcites=true,sorting=nyt,backend=biber]{biblatex}
%\DeclareLanguageMapping{american}{american-apa}
%\addbibresource{bibliography.bib}


%%%%%%%%%%%%%%%%%%%%%%%%%%%%%%%%%%%%%%%%
%% Discrete Structures
%% The start of RBS stuff
%%%%%%%%%%%%%%%%%%%%%%%%%%%%%%%%%%%%%%%%

% Working internal and external links in PDF
\usepackage{hyperref}
% Extra math symbols in LaTeX
\usepackage{amsmath}
\usepackage{gensymb}
\usepackage{amssymb}
% Enumerations with (a), (b), etc.
\usepackage{enumerate}
\usepackage[framemethod=TikZ]{mdframed}
\usepackage{xcolor}

\let\OLDitemize\itemize
\renewcommand\itemize{\OLDitemize\addtolength{\itemsep}{-6pt}}

\usepackage{etoolbox}
\makeatletter
\preto{\@verbatim}{\topsep=3pt \partopsep=3pt }
\makeatother

% These sizes redefine APA for A4 paper size
\oddsidemargin 0.0in
\evensidemargin 0.0in
\textwidth 6.27in
\headheight 1.0in
\topmargin -24pt
\headheight 12pt
\headsep 12pt
\textheight 9.19in



\setlength{\parindent}{0pt}
\setlength{\columnsep}{1cm}


\begin{document}

\twocolumn

\begin{center}
{\Large 2. Mājasdarbs}\\
{\Large Attēlu saspiešana un Kļūdu labošanas kodi}
\end{center}

%\begin{changemargin}{10pt}{10pt}
{\footnotesize
Vairāki uzdevumi šajā mājasdarbā iespaidojušies no MIT Open Courseware:
\url{https://bit.ly/3dabHyG} and \url{https://bit.ly/36xvx4e}.\\
}
%\end{changemargin}


{\bf Termiņš:} 2020.gada 9.novembris; līdz vakaram\\ (23:59:59 EET).\\
{\bf Iesniegšanas veids:} E-studiju vide.


\vspace{10pt}
{\bf 1.uzdevums (Kodēšana ar tabulu).}
Alise izveidoja $[5,2,1]$-kodu $2$-bitu datiem ($k=2$), 
ko pārraida ar $5$-bitu kodējumiem
($n=5$), kas atļauj $1$ kļūdas izlabošanu, 
jo He\-min\-ga attālums starp jebkuriem diviem kodējumiem 
ir vismaz $3$.
Pirmie divi biti katrā kodējumā pārraida divus {\em derīgo datu} (payload)
bitus; vēl trīs biti pievienoti kļūdu aizsardzībai.
Diemžēl, Alises suns sagrauza viņas piezīmes un iznīcināja
daļu no kodējumu tabulas (parādīts ar jautājuma zī\-mēm).
Jūsu uzdevums ir atjaunot kodu, kas apmierina augš\-mi\-nē\-tās prasības.

{\footnotesize
\begin{table}[h]
\begin{center}
\begin{tabular}{ccccccc}
Ievade & & \multicolumn{5}{c}{Kodējums} \\
$\mathtt{00}$ & $\rightarrow$ & 0 & 0 & ? & ? & ? \\
$\mathtt{01}$ & $\rightarrow$ & 0 & 1 & ? & ? & ? \\
$\mathtt{10}$ & $\rightarrow$ & 1 & 0 & ? & ? & ? \\
$\mathtt{11}$ & $\rightarrow$ & 1 & 1 & 0 & 0 & ? \\
\end{tabular}
\caption{\label{tab:codes523} Alises kodu tabula.}
\end{center}
\end{table}
}

\begin{enumerate}[(A)]
\item Atrast kaut vienu veidu, kā atjaunot kodu tabulu,
kas parādītu vienu veidu, kā nokodēt katru no $4$ divu bitu virknītēm, 
kas var būt ievadē. (Vai arī pamatojiet, ka no šīs tabulas 
$[5,2,1]$-kodu nevar iegūt.)
\item No $32$ iespējamām $5$-bitu virknītēm, cik dau\-dzām 
ir Heminga attālums tieši $1$
līdz kādam no iekodējumiem; t.i. tās var pārlabot uz tuvāko atļauto vērtību,
pieņemot, ka tās radīja viena kļūda?
\item Cik daudzas no $5$-bitu virknītēm 
varēja ras\-ties tikai tad, ja ir vairāk nekā $1$ kļūda?
\item Jūsu priekšnieks uzskata, ka elektrības tau\-pī\-ša\-nas apstākļos
sūtīt $1$-bitu ir dārgāk nekā $0$-bitu.
Vai iespējams samazināt $1$-bitu skai\-tu, kas ir Jūsu iekodējumu tabulā?
(Tikai jau\-tā\-jum\-zī\-mes drīkst mainīt 
savas bitu vēr\-tī\-bas; visiem pārējiem bitiem,
kas nav sagrauzti, jāpaliek tādiem, kādi tie tabulā ir.)
\end{enumerate}

% https://ocw.mit.edu/courses/electrical-engineering-and-computer-science/6-895-essential-coding-theory-fall-2004/assignments/ps1.pdf
% http://www-math.mit.edu/~djk/18.310/18.310F04/matrix_hamming_codes.html
\vspace{20pt}
{\bf 2.uzdevums (Lineārie algoritmi) }
Apzīmējam matricu $H$:
{\footnotesize
$$H = \left( \begin{array}{cccc}
0 & 0 & 0 & 1 \\
0 & 0 & 1 & 1 \\
0 & 1 & 0 & 1 \\
0 & 1 & 1 & 1 \\
1 & 0 & 0 & 1 \\
1 & 0 & 1 & 1 \\
1 & 1 & 0 & 1 \\
1 & 1 & 1 & 1 \\
\end{array} \right).$$
}

\begin{enumerate}[(A)]
\item
Atrast matricu $G$ ar lielākajiem izmēriem (un ar lineāri neatkarīgām kolonnām), kurai 
$G \cdot H$ ir matrica, kura satur tikai nulles, ja visas matricu darbības veic pēc $2$ moduļa. 
\item Ko var apgalvot par $G$ ģenerēto kodu: Kādi ir mazākie iespējamie Heminga attālumi 
starp $\mathbf{x}_1^T \cdot G$ un $\mathbf{x}_2^T \cdot G$, kur
$\mathbf{x}_1^T, \mathbf{x}_2^T \in \{ 0,1 \}^4$ ir jebkuras 4-bitu virknes
kas pierakstītas kā rindas vektori.
\end{enumerate}

{\em Piezīme.} Matricas $G$ kolonnas saucam par {\em lineāri neatkarīgām} 
ja katrai netukšai $G$ kolonnu apakš\-ko\-pai, šo kolonnu summa nevar būt vektors, 
kas satur tikai nulles (saskaitot pēc $2$ moduļa).
Sk. teoriju \url{https://bit.ly/37Fv9mJ}.


\vspace{20pt}
{\bf 3.uzdevums (Diskrētais kosinusu pārveidojums).}

Dota funkcija $f(x)$, kas definēta argumentiem $x=0,1,\ldots,N-1$.
Par 1-dimensionālu DCT (diskrēto kosinusu pārveidojumu jeb {\em discrete cosine transform}) sauksim
funkciju $F(u)$, kas definēta tām pašām argumenta vērtībām $u=0,1,\ldots,N-1$
ar šādām vienādībām:
$$F(u) = \sqrt{\frac{2}{N}} \cdot \lambda_u \cdot \sum\limits_{x=0}^{N-1} \cos \left( \frac{\pi u}{N}\left(  x+\frac{1}{2} \right) \right) \cdot f(x),$$
kur $u=0,1,\ldots,N-1$ un $\lambda_u = \frac{1}{\sqrt{2}}$ (pie $u=0$) un $\lambda_u = 1$ (pie $u>0$).\\
Sk.\ \url{https://bit.ly/3mj8h0A}.

Par inverso diskrēto kosinusu pārveidojumu sauksim atgriešanos no funkcijas $F(u)$ atpakaļ pie funk\-ci\-jas $f(x)$,
ko definē ar šādām vienādībām:
$$f(x) = \sqrt{\frac{2}{N}} \sum\limits_{u=0}^{N-1} \lambda_u \cdot \cos \left( \frac{\pi u}{N}\left(  x+\frac{1}{2} \right) \right) \cdot F(u),$$
kur $x=0,1,\ldots,N-1$ un $\lambda_u$ definēti tāpat kā agrāk.
\begin{enumerate}[(A)]
\item
Atrast diskrēto kosinusu pārveidojumu $F(u)$ punktā $u=3$
funkcijai\\ $f(x) = (N-1)-x$, kur $N=8$.
Atbildi noapaļot līdz 4 cipariem aiz komata.
\item
Aprēķināt inverso diskrēto kosinusu pārveidojumu $f(x)$ visiem
punktiem\\ $x=0,1,2,3,4,5,6,7$ no funkcijas $F(u)$, kas uzdota ar
sekojošām $N=8$ vērtībām:
{\footnotesize
$$(F(0),F(1),F(2),F(3),F(4),F(5),F(6),F(7)) =$$
$$ = (57.9828,-6.4423,0,-0.6735,0,-0.2009,0,-0.0507).$$
}
Atbildi noapaļot līdz 4 cipariem aiz komata.
\end{enumerate}






\vspace{20pt}
{\bf 4.uzdevums (Rīda-Solomona kods).}

\begin{enumerate}[(A)]
\item Izmantojot galīgu lauku ar $7$ elementiem 
$\text{GF}(7)$ kodējam ziņojumus no
$7$ simbolu alfabēta $\{ 0,1,2,\ldots,6 \}$ ar 3.pakāpes polinomiem $f(n)$,
pārraidot $7$ polinoma vērtības $(f(0),\ldots,f(6))$ visos
galīgā lauka $\text{GF}(7)$ punktos.
Dažas no šīm septiņām vērtībām var pa ceļam sabojāties 
(tikt aizstātas ar citu $\text{GF}(7)$ elementu).
Kāds ir maksimālais kļūdu skaits, pie kura iespējams
viennozīmīgi atjaunot sākotnējo ziņojumu?
Pamatot, kā\-pēc ir iespējams koriģēt šādu kļūdu skaitu un kāpēc nav
iespējams koriģēt lielāku kļūdu skaitu.
\item Galīga lauka $\text{GF}(2^3)$ elementus
$$\left\{ 0,1,t,t+1,t^2,t^2+1,t^2+t,t^2+t+1 \right\}$$
apzīmējam attiecīgi ar bitu virknēm 
$$\{\mathtt{000},\mathtt{001},\mathtt{010},\mathtt{011},
\mathtt{100},\mathtt{101},\mathtt{110},\mathtt{111}\}.$$
$3$-pakāpes polinoms 
$$p(x) = \mathtt{101}\cdot{}x^3 + \mathtt{100}\cdot{}x^2 + \mathtt{001}\cdot{}x + \mathtt{010}$$
domāts, lai pārraidītu ziņojumu virknīti\\ $\mathtt{101.100.001.010}$.
Atrast polinoma vērtību $p(\mathtt{011})$, kur $\mathtt{011}\in\text{GF}(2^3)$.\\
\end{enumerate}

{\em Piezīme:} Faktiski Rīda-Solomona kļūdu la\-bo\-ša\-nas kodā vajadzētu sūtīt
visas $8$ polinoma vērtības, bet šajā vingrinājumā pietiek izrēķināt $p(x)$ tikai pie $x=\mathtt{011}$.



\vspace{20pt}
{\bf 5.uzdevums (I-Iespēja)}

Jūs veidojat produktu, kas izmanto {\em Taisnstūrveida kodu} 
(rectangular code) (sk. 48.lpp. no \url{https://bit.ly/2M5ptGR})
lai nodrošinātu kritisko bitu pareizību, kurus sūta pa trokšņainu kanālu.
Jūsu risinājumā katru bloku ar $9$ ``derīgo datu'' (payload) bitiem
aizsargā ar Taisnstūrveida kodu.
Jūsu metode katrus
deviņus derīgo datu bitus ($\mathtt{D0},\ldots,\mathtt{D8}$)
aizsargā ar septiņiem kļūdu labošanas bitiem 
($\mathtt{PR0}$, $\mathtt{PR1}$, $\mathtt{PR2}$, $\mathtt{PC0}$, 
$\mathtt{PC1}$, $\mathtt{PC2}$, $\mathtt{P0}$).
Šie biti derīgo datu bitu summas (pēc $2$ moduļa) 
pa rindiņām vai kolonnām (bet $P0$ ir visu 
datu bitu summa). 

Kļūdu labošanas metodei jāizpilda $2$ prasības:
\begin{itemize}
\item Atrast un izlabot kļūdu jebkurā vienā bitā no deviņiem 
($\mathtt{D0},\ldots,\mathtt{D8}$).
\item Atpazīt/detektēt kļūdas jebkuros divos bitos no deviņiem 
(iespējams, nepasakot, kuri ir kļūdainie biti).
\end{itemize}

\begin{table}[h]
\begin{center}
\begin{tabular}{cccc}
{\tt D0} & {\tt D1} & {\tt D2} & \textcolor{red}{\tt PR0} \\
{\tt D3} & {\tt D4} & {\tt D5} & {\tt PR1} \\
{\tt D6} & {\tt D7} & {\tt D8} & {\tt PR2} \\
\textcolor{red}{\tt PC0} & {\tt PC1} & {\tt PC2} & {\tt P0} \\
\end{tabular}
\caption{\label{tab:rectangular} Taisnstūrveida kods.}
\end{center}
\end{table}

Bens \textendash{} viens no Jūsu kolēģiem \textendash{} pārbaudījis Jūsu
metodi, ierosina to mainīt tā, ka Jūs nepārraidāt paritātes bitus
$\mathtt{PR0}$ un $\mathtt{PC0}$, tikai tos četrus bitus, kas
saistīti ar citām rindām un kolonnām 
($\mathtt{PR1},\mathtt{PR2}$ un $\mathtt{PC1},\mathtt{PC2}$), 
kā arī kopējo paritātes bitu ($P0$).
Viņš apgalvo, ka arī šāds kods izpildīs abas prasības 
un efektīvāk izmantos sakaru kanālu.
\begin{enumerate}[(A)]
\item Vai visas viena bita kļūdas var atrast un izlabot (ar Bena piedāvāto izmaiņu).\\ 
Pamatojiet, ka vienmēr var vai atrodiet pretpiemēru. 
\item Vai visas divu bitu kļūdas var atrast un izlabot (ar Bena piedāvāto izmaiņu).\\
Pamatojiet, ka vienmēr var vai atrodiet pretpiemēru. 
\item Vai visas divu bitu kļūdas var atpazīt kā divu bitu kļūdas \textendash{} 
nenosakot, kuros bitos bija kļūda (ar Bena piedāvāto izmaiņu).\\
Pamatojiet, ka vienmēr var vai atrodiet pretpiemēru. 
\end{enumerate}




\end{document}

