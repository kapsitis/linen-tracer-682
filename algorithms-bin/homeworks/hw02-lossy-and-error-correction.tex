\documentclass[11pt]{article}
\usepackage{ucs}
\usepackage[utf8x]{inputenc}
\usepackage{changepage}
\usepackage{graphicx}
\usepackage{amsmath}
\usepackage{gensymb}
\usepackage{amssymb}
\usepackage{enumerate}
\usepackage{tabularx}
\usepackage{lipsum}
\usepackage{hyperref}

\oddsidemargin 0.0in
\evensidemargin 0.0in
\textwidth 6.27in
\headheight 1.0in
\topmargin -0.1in
\headheight 0.0in
\headsep 0.0in
\textheight 9.0in

%\setlength\parindent{0pt}

\newenvironment{myenv}{\begin{adjustwidth}{0.4in}{0.4in}}{\end{adjustwidth}}
\renewcommand{\abstractname}{Anotācija}
\renewcommand\refname{Atsauces}



\newcounter{alphnum}
\newenvironment{alphlist}{\begin{list}{(\Alph{alphnum})}{\usecounter{alphnum}\setlength{\leftmargin}{2.5em}} \rm}{\end{list}}


%16.3-6

\makeatletter
\let\saved@bibitem\@bibitem
\makeatother

\usepackage{bibentry}
%\usepackage{hyperref}


\begin{document}

\thispagestyle{empty}

{\Large Lietišķie algoritmi \textendash{} 2. mājas darbs}

\noindent
{\em Termiņš: 2019.\ gada 28.\ oktobrī 23:59:59 pēc Austrumeiropas ziemas laika (EET) jeb UTC+2.}\\
{\em Iesūtīšanas veids: PDF uz epastu "kalvis.apsitis" domēnā "gmail.com".}


\begin{enumerate}
\item {\bf Diskrētā krāsu plakne YCbCr.} 
Koordinātes $(Y,Cb,Cr)$ aprēķina no koordinātēm $(R,G,B)$ atbilstoši 
sekojošai vektoru algebras sakarībai:
$$\left( \begin{array}{c}
Y\\
Cb\\
Cr
\end{array} \right) = \left( \begin{array}{ccc}
65.48 & 128.55 & 24.97\\
-37.78 & -74.16 & 111.93\\
111.96 & -93.75 & -18.21
\end{array} \right) \left( \begin{array}{ccc}
\frac{1}{255} & 0 & 0\\
0 & \frac{1}{255} & 0\\
0 & 0 & \frac{1}{255}
\end{array} \right) 
\left( \begin{array}{c}
R\\
G\\
B
\end{array} \right) + \left( \begin{array}{c}
16\\
128\\
128
\end{array} \right).$$
Sk. \url{https://onlinelibrary.wiley.com/doi/full/10.1002/col.22291}.\\
Atrast YCbCr koordinātes zemāk minētajām krāsām, kas uzdotas $(R,G,B)$ krāsu plaknē
(un noapaļot visas $(Y,Cb,Cr)$ koordinātes līdz tuvākajam veselajam skaitlim). 
\begin{enumerate}[(a)]
\item Baltai krāsai {\tt \#FFFFFF} jeb $(R,G,B) = (255,255,255)$. 
\item Ciāna krāsai  {\tt \#00FFFF} jeb $(R,G,B) = (0,255,255)$.
\item Magentas krāsai {\tt \#FF00FF} jeb $(R,G,B) = (255,0,255)$. 
\item Dzeltenai krāsai {\tt \#FFFF00} jeb $(R,G,B) = (255,255,0)$. 
\item Melnai krāsai  {\tt \#000000} jeb $(R,G,B) = (0,0,0)$.
\end{enumerate}

\item {\bf Diskrētais kosinusu pārveidojums.}
Dota funkcija $f(x)$, kas definēta argumentiem $x=0,1,\ldots,N-1$. 
Par 1-dimensionālu DCT (diskrēto kosinusu pārveidojumu jeb {\em discrete cosine transform}) sauksim 
funkciju $F(u)$, kas definēta tām pašām argumenta vērtībām $u=0,1,\ldots,N-1$ 
ar šādām vienādībām: 
$$F(u) = \sqrt{\frac{2}{N}} \sum\limits_{x=0}^{N-1} \lambda_u \cdot \cos \left( \frac{\pi u}{N}\left(  x+\frac{1}{2} \right) \right) \cdot f(x),$$
kur $u=0,1,\ldots,N-1$ un $\lambda_u = \frac{1}{\sqrt{2}}$ (pie $u=0$) un $\lambda_u = 1$ (pie $u>0$).\\
Sk.\ \url{https://en.wikipedia.org/wiki/Discrete_cosine_transform#DCT-II}.

Par inverso diskrēto kosinusu pārveidojumu sauksim atgriešanos no funkcijas $F(u)$ atpakaļ pie funkcijas $f(x)$, 
ko definē ar šādām vienādībām:
$$f(x) = \sqrt{\frac{2}{N}} \sum\limits_{u=0}^{N-1} \lambda_u \cdot \cos \left( \frac{\pi u}{N}\left(  x+\frac{1}{2} \right) \right) \cdot F(u),$$
kur $x=0,1,\ldots,N-1$ un $\lambda_u$ definēti tāpat kā agrāk.
\begin{enumerate}[(a)]
\item 
Aprēķināt diskrēto kosinusu pārveidojumu $F(u)$ punktā $u=3$
funkcijai\\ $f(x) = (N-1)-x$, kur $N=8$. 
Atbildi noapaļot līdz 4 cipariem aiz komata.
\item 
Aprēķināt inverso diskrēto kosinusu pārveidojumu $f(x)$ visiem 
punktiem $x=0,1,2,3,4,5,6,7$ no funkcijas $F(u)$, kas uzdota ar
sekojošām $N=8$ vērtībām:
$$(F(0),F(1),F(2),F(3),F(4),F(5),F(6),F(7)) =$$
$$ = (57.9828,-6.4423,0,-0.6735,0,-0.2009,0,-0.0507).$$
Atbildi noapaļot līdz 4 cipariem aiz komata.\\
{\em (Turpinājums lapas otrā pusē.)}
\end{enumerate}

\item {\bf Heminga kodi.} 
Uzrakstām skaitļa $\pi = 3.14159\ldots$ pierakstu divnieku skaitīšanas sistēmā un 
grupējam tā ciparus aiz komata - divas grupas pa $7$ un divas grupas pa $15$:
$$\pi = 11.\underbrace{0010010}\underbrace{0001111}\underbrace{110110101010001}\underbrace{000100001011010}\ldots_2.$$
Grupējam šī skaitļa ciparus aiz komata grupās pa $7$ (un pēc tam arī pa
$15$), lai iegūtu ziņojumus. Atrast kļūdas (ja tādas ir) atbilstošajos Heminga koda ziņojumos:
\begin{enumerate}[(a)]
\item Heminga koda ziņojums {\tt 0010010} ($7$-bitu Heminga kods, bitu secība apgriezti leksikogrāfiska
\textendash{} no $x_{111}$ līdz $x_{001}$) \textendash{} atrast kļūdaino pozīciju, ja tāda ir, un uzrakstīt šo $7$-bitu kodu bez kļūdām.
\item Heminga koda ziņojums {\tt 0001111} ($7$-bitu Heminga kods) \textendash{} izlabot kļūdas, ja tās ir, un uzrakstīt $4$ ziņojuma bitus $x_1x_2x_3x_4$. 
\item Heminga koda ziņojums {\tt 110110101010001} ($15$-bitu Heminga kods, bitu secība apgriezti leksikogrāfiska 
\textendash{} no $x_{1111}$ līdz $x_{0001}$) \textendash{} atrast kļūdaino pozīciju, ja tāda ir, un uzrakstīt šo $15$-bitu kodu bez kļūdām.
\item Heminga koda ziņojums {\tt 000100001011010} ($15$-bitu Heminga kods) \textendash{} 
izlabot kļūdas, ja tās ir, un uzrakstīt visus ziņojuma bitus (ziņojumu bitu secība apgriezti leksikogrāfiska $x_{1111}\ldots{}x_{0011}$).
\end{enumerate}

\item {\bf Rīda-Solomona kods.} 
\begin{enumerate}[(a)]
\item Izmantojot galīgu lauku $\text{GF}(7)$ kodējam ziņojumus no 
$7$ simbolu alfabēta $\{ 0,1,2,\ldots,6 \}$ ar $3$ pakāpes polinomiem, 
pārraidot $7$ polinoma vērtības $(f(0),\ldots,f(7))$ visos
galīgā lauka $\text{GF}(7)$ punktos. 
Kāds ir maksimālais kļūdu skaits, pie kura iespējams 
viennozīmīgi atjaunot sākotnējo ziņojumu?
Pamatot, kāpēc ir iespējams koriģēt šādu kļūdu skaitu un kāpēc nav
iespējams koriģēt lielāku kļūdu skaitu.
\item Galīga lauka $\text{GF}(2^3)$ elementus $\left\{ 0,1,t,t+1,t^2,t^2+1,t^2+t,t^2+t+1 \right\}$
apzīmējam attiecīgi ar bitu virknēm $\{\mathtt{000},\mathtt{001},\mathtt{010},\mathtt{011},
\mathtt{100},\mathtt{101},\mathtt{110},\mathtt{111}\}$.\\
$3$-pakāpes polinoms $p(x) = \mathtt{101}\cdot{}x^3 + \mathtt{100}\cdot{}x^2 + \mathtt{001}\cdot{}x + \mathtt{010}$
ir domāts, lai pārraidītu ziņojumu virknīti $\mathtt{101.100.001.010}$. 
Atrast polinoma vērtību $p(\mathtt{011})$, kur $\mathtt{011}\in\text{GF}(2^3)$.\\
{\em Piezīme:} Faktiski Rīda-Solomona kļūdu labošanas kodā vajadzētu sūtīt 
visas $8$ polinoma vērtības, bet šajā vingrinājumā pietiek izrēķināt $p(x)$ tikai pie $x=\mathtt{011}$. 
\end{enumerate}

\item {\bf I-iespēja (atzīmei 10).} 
\begin{enumerate}[(a)]
\item Uzrakstīt grafu kodu ar $9$ ziņojuma bitiem $x_1,\ldots,x_{9}$ 
un pēc iespējas mazāku skaitu kontrolbitu $y_1,\ldots,y_k$ tā, 
lai kods spētu atjaunot jebkurus $2$ pazaudētus ziņojuma bitus $x_i$.
\item Pamatot, ka mazāks kontrolbitu skaits nav iespējams.
\item Uzrakstīt grafu kodu ar $9$ ziņojuma bitiem $x_1,\ldots,x_{9}$ 
un pēc iespējas mazāku skaitu kontrolbitu $y_1,\ldots,y_k$ tā, 
lai kods spētu atjaunot jebkurus $3$ pazaudētus ziņojuma bitus $x_i$.
\end{enumerate}
\end{enumerate}



\end{document}



