%%
%% This is file `./samples/shortsample.tex',
%% generated with the docstrip utility.
%%
%% The original source files were:
%%
%% apa6.dtx  (with options: `shortsample')
%% ----------------------------------------------------------------------
%% 
%% apa6 - A LaTeX class for formatting documents in compliance with the
%% American Psychological Association's Publication Manual, 6th edition
%% 
%% Copyright (C) 2011-2017 by Brian D. Beitzel <brian at beitzel.com>
%% 
%% This work may be distributed and/or modified under the
%% conditions of the LaTeX Project Public License (LPPL), either
%% version 1.3c of this license or (at your option) any later
%% version.  The latest version of this license is in the file:
%% 
%% http://www.latex-project.org/lppl.txt
%% 
%% Users may freely modify these files without permission, as long as the
%% copyright line and this statement are maintained intact.
%% 
%% This work is not endorsed by, affiliated with, or probably even known
%% by, the American Psychological Association.
%% 
%% ----------------------------------------------------------------------
%% 
\documentclass[jou]{apa6}

\usepackage[american]{babel}

\usepackage{csquotes}
\usepackage[style=apa,sortcites=true,sorting=nyt,backend=biber]{biblatex}
\DeclareLanguageMapping{american}{american-apa}
\addbibresource{bibliography.bib}


%%%%%%%%%%%%%%%%%%%%%%%%%%%%%%%%%%%%%%%%
%% Discrete Structures
%% The start of RBS stuff
%%%%%%%%%%%%%%%%%%%%%%%%%%%%%%%%%%%%%%%%

% Working internal and external links in PDF
\usepackage{hyperref}
% Extra math symbols in LaTeX
\usepackage{amsmath}
\usepackage{gensymb}
\usepackage{amssymb}
% Enumerations with (a), (b), etc.
\usepackage{enumerate}

\let\OLDitemize\itemize
\renewcommand\itemize{\OLDitemize\addtolength{\itemsep}{-6pt}}

\usepackage{etoolbox}
\makeatletter
\preto{\@verbatim}{\topsep=3pt \partopsep=3pt }
\makeatother

% These sizes redefine APA for A4 paper size
\oddsidemargin 0.0in
\evensidemargin 0.0in
\textwidth 6.27in
\headheight 1.0in
\topmargin -24pt
\headheight 12pt
\headsep 12pt
\textheight 9.19in



\title{Discrete Structures (W1): Quiz}
\author{Kalvis}
\affiliation{RBS}

\leftheader{Discrete Structures (W1)}

\abstract{
}

%\keywords{}

\begin{document}
\maketitle

{\bf Question 1.} Fill in the missing entries in the truth table of this proposition:
$$E = \neg(r \rightarrow \neg q) \wedge (q \rightarrow r).$$

\vspace{3pt}
\noindent
{\em Fill in the $\ldots$:}

\begin{tabular}{ c | c | c | c }
$p$ & $q$ & $r$ & $E$ \\ \hline
{\tt T} & {\tt T} & {\tt T} & don't care \\ \hline
{\tt T} & {\tt T} & {\tt F} & don't care \\ \hline
{\tt T} & {\tt F} & {\tt T} & $\ldots$ \\ \hline
{\tt T} & {\tt F} & {\tt F} & $\ldots$ \\ \hline
{\tt F} & {\tt T} & {\tt T} & don't care \\ \hline
{\tt F} & {\tt T} & {\tt F} & don't care \\ \hline
{\tt F} & {\tt F} & {\tt T} & $\ldots$ \\ \hline
{\tt F} & {\tt F} & {\tt F} & $\ldots$ \\ \hline
\end{tabular}

\vspace{10pt}
{\bf Question 2.} Find the Boolean expression that has this truth table:

\begin{tabular}{ c | c | c }
$p$ & $q$ & ? \\ \hline
{\tt T} & {\tt T} & {\tt F} \\ \hline
{\tt T} & {\tt F} & {\tt F} \\ \hline
{\tt F} & {\tt T} & {\tt T} \\ \hline
{\tt F} & {\tt F} & {\tt F} \\ \hline
\end{tabular}

\noindent
\vspace{3pt}
{\em (Select 1 answer):}\\
{\bf (A)} $\neg (p \rightarrow q)$,\\
{\bf (B)} $\neg p \rightarrow \neg q$,\\
{\bf (C)} $\neg (q \rightarrow p)$,\\
{\bf (D)} $\neg q \rightarrow \neg p$.

\vspace{10pt}
{\bf Question 3.} Determine whether the following proposition is {\em satisfiable}:
$(\neg p \vee \neg q) \wedge (p \rightarrow q)$. If it is satisfiable, what are the 
truth values for $p$ and $q$ that makes it {\tt true}.\\


\vspace{3pt}
\noindent
{\em (Circle answer and fill in the $\ldots$, if appropriate.)}\\
Is the expression satisfiable: \hspace{5ex} YES \hspace{5ex} NO\\
If yes, what values can satisfy it (just 1 example):\\ 
$p = \ldots$ \hspace{5ex} $q = \ldots$ 



\vspace{10pt}
{\bf Question 4.}
Consider the following proposition: ``You cannot eat vegetables unless you also eat ice cream.''
Express it as a Boolean expression, if there are two atomic propositions:\\
$A$: ``Person $x$ can eat vegetables.''\\
$B$: ``Person $x$ eats ice cream.''

\vspace{3pt}
\noindent
{\em (Write your expression here)} $\ldots$


\vspace{10pt}
{\bf Question 5.} 
Determine whether the following two propositions are logically equivalent:
$E_1 =  p \vee \neg (q \vee r)$ and $E_2 = (p \wedge \neg q) \vee (p \wedge \neg r)$.\\
If they are not equivalent, find some values $p,q,r$ that makes $E_1$ different 
from $E_2$.



\vspace{3pt}
\noindent
{\em (Circle answer and fill in the $\ldots$, if appropriate.)}\\
Are both expressions equivalent: \hspace{5ex} YES \hspace{5ex} NO\\
If not, which truth values make them different:\\ 
$p = \ldots$ \hspace{5ex} $q = \ldots$ \hspace{5ex} $r = \ldots$. 



\vspace{10pt}
{\bf Question 6.}
Translate the given statement into propositional logic using the propositions provided:
\begin{quote}
``In Riga a person can receive {\em low income status}, 
if he or she lives in a family where the income per family member during the last 3 months did not exceed 320 EUR per month, 
or there is one person in your family, who receives an old-age or disability benefit up to 400 EUR per month.''\\
\end{quote}
Express your answer in terms of $3$ atomic propositions\\
$A$: ``You are living in a family where the average income per family member does not exceed 320 EUR per month during
the last $3$ months.''\\
$B$: ``You are a single who receives an old-age or disability benefit not exceeding 400 EUR per month.'' and\\
$C$: ``You can get low income status.''

\vspace{3pt}
\noindent
{\em (Write your expression here)} $\ldots$


\vspace{10pt}
{\bf Question 7.} 
({\em Note:} In this problem ``knights'' always tell the truth and ``knaves'' always lie.)\\
Person $A$ says ``B is a knave.''
Person $B$ says ``We are both knights.'' Determine whether each person is a knight or a knave.

\vspace{3pt}
\noindent
Is this situation possible: \hspace{5ex} YES \hspace{5ex} NO\\
If the situation is possible, who are $A,B$: $\ldots$

\newpage

\section{Answers}



\noindent
{\bf Question 1:} Answer:

\begin{tabular}{ c | c | c | c }
$p$ & $q$ & $r$ & $E$ \\ \hline
{\tt T} & {\tt T} & {\tt T} & don't care \\ \hline
{\tt T} & {\tt T} & {\tt F} & don't care \\ \hline
{\tt T} & {\tt F} & {\tt T} & $\boxed{\mathtt{F}}$  \\ \hline
{\tt T} & {\tt F} & {\tt F} & $\boxed{\mathtt{F}}$  \\ \hline
{\tt F} & {\tt T} & {\tt T} & don't care \\ \hline
{\tt F} & {\tt T} & {\tt F} & don't care \\ \hline
{\tt F} & {\tt F} & {\tt T} & $\boxed{\mathtt{F}}$  \\ \hline
{\tt F} & {\tt F} & {\tt F} & $\boxed{\mathtt{F}}$  \\ \hline
\end{tabular}

{\em Solution.} Only need to compute $E = \neg(r \rightarrow \neg q) \wedge (q \rightarrow r)$
when $q = \mathtt{false}$ (this equality takes place for all the $4$ interesting values). Also 
note that the expression does not depend on $p$. We can simplify:

\begin{align}
E \equiv & \neg(r \rightarrow \neg q) \wedge (q \rightarrow r) \equiv \nonumber \\
  \equiv & \neg(r \rightarrow \neg \mathtt{false}) \wedge (q \rightarrow \mathtt{false}) \equiv \nonumber \\
  \equiv & \neg(r \rightarrow \mathtt{true}) \wedge q \equiv \nonumber \\
  \equiv & \neg(\mathtt{true}) \wedge q \equiv \nonumber \\
  \equiv & \mathtt{false} \wedge q \equiv \mathtt{false}.
\end{align}


This table indicates that $E = \mathtt{false}$ whenever $q = \mathtt{false}$. 

\vspace{10pt}
\noindent
{\bf Question 2.} Answer: {\bf (C)}.\\
Formula $\neg(q \rightarrow p)$ is {\tt true} only when $q \rightarrow p$ is 
{\tt false} (it happens when $q = \mathtt{true}$ and $p = \mathtt{false}$. This is
exactly the line which is true in our truth table.

Other answer alternatives compute other truth tables: Implications $\neg p \implies \neg q$ 
or $\neg q \implies \neg p$ have $3$ out of $4$ values equal to $\mathtt{true}$ (instead of $1$ out of $4$).\\
On the other hand, Boolean expression $\neg(p \rightarrow q)$ is true only when 
$p \equiv \mathtt{true}$ and $q \equiv \mathtt{false}$ (this is not our case).


\vspace{10pt}
\noindent
{\bf Question 3.} Answer: Yes, the formula is satisfiable.\\
To satisfy it, select $p = \mathtt{true}$ (and $q$ can be anything). 
In this case the Boolean expression:
\begin{align}
E \equiv & (\neg p \vee \neg q) \wedge (p \rightarrow q) \equiv \nonumber \\
  \equiv & (\neg \mathtt{true} \vee \neg q) \wedge (\mathtt{true} \rightarrow q) \equiv \nonumber
\end{align}

\vspace{10pt}
\noindent
{\bf Question 4.} Answer: $A \rightarrow B$.\\
Let us reformulate the sentence: ``You cannot eat vegetables unless you also 
eat ice cream.''. It becomes this: ``Whenever you have eaten vegetables, you 
have eaten ice cream.''. Therefore vegetables imply ice cream (on the other hand, 
eating ice cream alone is fine). Therefore the answer is any of the equivalent expressions:
$$A \rightarrow B;\;\;\neg{}B \rightarrow \neg{}A;\;\;\neg{}A \vee B.$$

\vspace{10pt}
\noindent
{\bf Question 5.} Answer: No, $E_1$ and $E_2$ are not equivalent.\\
Their values differ when $(p;q;r)=(\mathtt{T},\mathtt{T},\mathtt{T})$ or 
$(p;q;r)=(\mathtt{F},\mathtt{F},\mathtt{F})$.

\vspace{10pt}
\noindent
{\bf Question 6.} Answer: $(A \vee B) \rightarrow C$.\\ 
The original text of these rules is trickier than that. 
See \url{https://bit.ly/35N8sIE}, Article 14. 
One could find some ambiguous cases: What happens, if in the appartment there is 
one person with disability benefits (slightly under 400) plus some other family members
having their average income slightly under 320? (But, in the meantime, 
the overall average monthly income in this family is between 320 and 400?) \\
And why the low-income status is
given only to the working age people, but the rule also mentions old-age pension
(vecuma pensija)? 

\vspace{10pt}
\noindent
{\bf Question 7.} Answer: $A$ is knight, $B$ is knave.\\
We can sort cases depending on $B$:\\
{\bf Case 1:} If $B$ were a knight, then what he says (that both $A$ and $B$ are knights)
must be true. But $A$ says that $B$ is a knave. This is a contradiction.\\
{\bf Case 2:} If $B$ were a knave, then there is no contradiction, plus $A$ tells the truth; 
so $A$ must be knight.




\end{document}



%%%%%%%%%%%%%%%%%%%%%%%%%%%%%%%%%%%%%%%%
%% End of RBS stuff
%%%%%%%%%%%%%%%%%%%%%%%%%%%%%%%%%%%%%%%%


%% 
%% Copyright (C) 2011-2017 by Brian D. Beitzel <brian at beitzel.com>
%% 
%% This work may be distributed and/or modified under the
%% conditions of the LaTeX Project Public License (LPPL), either
%% version 1.3c of this license or (at your option) any later
%% version.  The latest version of this license is in the file:
%% 
%% http://www.latex-project.org/lppl.txt
%% 
%% Users may freely modify these files without permission, as long as the
%% copyright line and this statement are maintained intact.
%% 
%% This work is not endorsed by, affiliated with, or probably even known
%% by, the American Psychological Association.
%% 
%% 
%% This work is "maintained" (as per LPPL maintenance status) by
%% Brian D. Beitzel.
%% 
%% This work consists of the file  apa6.dtx
%% and the derived files           apa6.ins,
%%                                 apa6.cls,
%%                                 apa6.pdf,
%%                                 README,
%%                                 APAamerican.txt,
%%                                 APAbritish.txt,
%%                                 APAdutch.txt,
%%                                 APAenglish.txt,
%%                                 APAgerman.txt,
%%                                 APAngerman.txt,
%%                                 APAgreek.txt,
%%                                 APAczech.txt,
%%                                 APAturkish.txt,
%%                                 APAendfloat.cfg,
%%                                 apa6.ptex,
%%                                 TeX2WordForapa6.bas,
%%                                 Figure1.pdf,
%%                                 shortsample.tex,
%%                                 longsample.tex, and
%%                                 bibliography.bib.
%% 
%%
%% End of file `./samples/shortsample.tex'.
