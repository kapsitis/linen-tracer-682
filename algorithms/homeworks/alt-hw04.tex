\documentclass[jou]{apa6}

\usepackage[american]{babel}

\usepackage{csquotes}
\usepackage[style=apa,sortcites=true,sorting=nyt,backend=biber]{biblatex}
\DeclareLanguageMapping{american}{american-apa}
\addbibresource{bibliography.bib}


%%%%%%%%%%%%%%%%%%%%%%%%%%%%%%%%%%%%%%%%
%% Discrete Structures
%% The start of RBS stuff
%%%%%%%%%%%%%%%%%%%%%%%%%%%%%%%%%%%%%%%%

% Working internal and external links in PDF
\usepackage{hyperref}
% Extra math symbols in LaTeX
\usepackage{amsmath}
\usepackage{gensymb}
\usepackage{amssymb}
% Enumerations with (a), (b), etc.
\usepackage{enumerate}
\usepackage[framemethod=TikZ]{mdframed}
\usepackage{xcolor}
\usepackage{graphicx}
\usepackage[justification=centering]{caption}
\usepackage{fancyvrb}
%\usepackage{changepage}
\usepackage{caption}% http://ctan.org/pkg/caption
\captionsetup[table]{format=plain,labelformat=simple,labelsep=period}%

\def\changemargin#1#2{\list{}{\rightmargin#2\leftmargin#1}\item[]}
\let\endchangemargin=\endlist 


\let\OLDitemize\itemize
\renewcommand\itemize{\OLDitemize\addtolength{\itemsep}{-6pt}}

\usepackage{etoolbox}
\makeatletter
\preto{\@verbatim}{\topsep=3pt \partopsep=3pt }
\makeatother

% These sizes redefine APA for A4 paper size
\oddsidemargin 0.0in
\evensidemargin 0.0in
\textwidth 6.27in
\headheight 1.0in
%\topmargin -24pt
\topmargin -32pt
\headheight 12pt
\headsep 12pt
%\textheight 9.19in
\textheight 9.35in


\title{Sample Quiz 8}
\author{Discrete Structures, Spring 2020}
\affiliation{RBS}

\leftheader{Discrete Sample Quiz 8}

\abstract{%
}

%\keywords{}

\setlength\parindent{0pt}

\begin{document}

\twocolumn
\thispagestyle{empty}

\begin{center}
{\Large Alternative Homework 4:}\\
{\Large String Search}
\end{center}

% https://ocw.mit.edu/courses/electrical-engineering-and-computer-science/6-441-information-theory-spring-2016/assignments/

\begin{changemargin}{10pt}{10pt}
{\footnotesize
{\em Note.} This is a parody of CMU and MIT OCW content.\\
The original assignments and related materials can be retrieved from 
\url{https://bit.ly/3i1vnHB}, \url{https://bit.ly/31k7uVI},
\url{https://bit.ly/37Wgyl4}.
}
\end{changemargin}

%https://ocw.mit.edu/courses/electrical-engineering-and-computer-science/6-046j-design-and-analysis-of-algorithms-spring-2015/assignments/MIT6_046JS15_pset2.pdf




%% KMP as automaton
%% https://www-inst.eecs.berkeley.edu//~cs172/sp09/hw5.pdf
\vspace{4pt}
{\bf Question 1.}\\
Knuth-Morris-Pratt string search algorithm makes use of a DFA (Deterministic Finite Automaton)
to keep track of the longest feasible prefix of $S$ that has been matched so far in the input text $T$. 
The transitions between the states in this DFA automaton can be represented using 
the {\em prefix function} or the {\tt memo[i]} table, where $i \in \{ 0,\ldots,\ell{}-1 \}$.\\
These concepts are defined in Section 1 of \url{https://bit.ly/3fXWnWT}. 

Assume that somebody wants to build a DFA automaton that reads text $T$ once from left to right. 
S/he searches for either one of the two strings: either $S_1 = \textcolor{blue}{\mathtt{1110111101}}$ 
or $S_2 = \textcolor{blue}{\mathtt{10101}}$ (after the first match of {\bf either} $S_1$ {\bf or} $S_2$ 
is found, the automaton reaches its final accepting state and stops).

How many states would that DFA have, and what data structure (instead of the table {\tt memo[i]}) can 
be used to encode the transitions in this DFA?


\vspace{10pt}
{\bf Question 2.}
Suppose you are given a searchable text $T[0..n-1]$ of length $n$, consisting of symbols 
$\textcolor{blue}{\mathtt{a}}$ and $\textcolor{blue}{\mathtt{b}}$.
Suppose further that you are given a pattern string $S[0..\ell-1]$ of length $\ell$ much smaller than $n$, consisting of
symbols $\textcolor{blue}{\mathtt{a}}$, $\textcolor{blue}{\mathtt{b}}$, and $\textcolor{blue}{\mathtt{\ast}}$, 
representing a pattern to be found in the original text $T$. The symbol $\textcolor{blue}{\mathtt{\ast}}$ is a ``wild card''
symbol, which matches a single symbol, either $\textcolor{blue}{\mathtt{a}}$ or 
$\textcolor{blue}{\mathtt{b}}$. The other symbols must match exactly.
The problem is to output a sorted list $M$ of valid ``match positions'', which are positions $j$ in $S$
such that pattern $P$ matches the substring $S[j..j+\ell-1]$. For example, if $T =\textcolor{blue}{\mathtt{ababbab}}$ and
$S=\textcolor{blue}{\mathtt{ab\ast}}$, then the output should be $[0,2]$.

Consider an equivalent of Knuth-Morris-Pratt algorithm that reads the searchable text $T$ once from left to right
and (upon every mismatch between the text $T$ and pattern $S$) shifts the pattern $S$ forward by the smallest
feasible amount.
We want to search for the pattern $S = \textcolor{blue}{\mathtt{11\ast0111\ast01}}$.

{\bf (A)} Write the pseudocode for this modified KMP algorithm.\\
{\bf (B)} Fill in the table of shifts (also called {\em prefix function} or {\tt memo[i]}) describing the shifts, 
if the $T$ does not match the search pattern $S = \textcolor{blue}{\mathtt{11\ast0111\ast01}}$ at some position.

This is a parody of Problem 2.1, see p.1 in \url{https://bit.ly/2XKX5AB}.






\vspace{10pt}
{\bf Question 3.}\\
Consider the following prehashing function converting strings $S$ of length $\ell$
into integer ``keys'': 

{\footnotesize
\begin{equation}
\label{eq:prehashing}
k(S) = \left(S[0]\cdot{}b^{\ell-1} + S[1] \cdot b^{\ell-2} + \ldots + S[\ell - 1]\right)\;\text{mod}\;q.
\end{equation}} 

\noindent
Here $b$ (called the ``number base'') is a number larger than the alphabet size used for $S$. 

The same prehashing function can be applied to a substring of a longer text $T$ (processing 
exactly $\ell$ consecutive symbols): 
{\footnotesize
\begin{align}
 & k(T[i..i+\ell-1]) = \nonumber \\
= & \left(S[i]\cdot{}b^{\ell-1} + S[i+1] \cdot b^{\ell-2} + \ldots + T[i + \ell{} - 1]\right)\;\text{mod}\;q, \label{eq:prehashing2}
\end{align}
}

Consider also the following
hashing function (See {\em multiplication method} in \url{https://bit.ly/2V8UfDF})
\begin{equation}
\label{eq:hashing}
h(k) = \left[ (a \cdot k)\;\text{mod}\;2^w \right]\;\gg\;(w - r),
\end{equation}
In this expression:
\begin{itemize}
\item $\gg$ denotes the bitwise ``shift right'' operator,
\item $2^r = m$ is the hash table size,
\item $w$ is the bit-length of a ``machine word'' (choose whatever is 
convenient; regardless of the hardware architecture).
\item $a$ is an odd integer between $2^{w-1}$ and $2^w$.
\end{itemize}

Our goal is to use the hashing method:
$$h(k(T[i..i+\ell-1]))$$
in the Rabin-Karp string search. 
We want to find patterns $S$ (of length $\ell \leq 100$ symbols) in a long text $T$.
Both $S$ and $T$ are written in ASCII, using its $128$ symbol alphabet.

{\bf (A)} How would you compute the {\tt r.append(c)} and {\tt r.skip(c)}
operations in the rolling hash ADT, if the prehashing and hashing are implemented 
using the expressions (\ref{eq:prehashing}), (\ref{eq:prehashing2}) and (\ref{eq:hashing}).\\
The Rolling Hash abstract datatype (ADT) and Rabin-Karp algorithm are explained in \url{https://bit.ly/2BxpNMu} (starting from 38:00).

{\bf (B)} Write the time complexity of the overall Rabin-Karp algorithm using $O(\ldots)$ (the Big-O notation)
for this hashing method. Denote the length of the text $T$ by $n = |T|$, and the length of the search pattern 
$S$ by $\ell = |S|$. 

{\em Note.} This was inspired by Problem 4-1 from \url{https://bit.ly/3dyNtgK}. 




%% Suffix trees in P1?
%%https://dspace.mit.edu/bitstream/handle/1721.1/36897/6-854JFall-1999/NR/rdonlyres/Electrical-Engineering-and-Computer-Science/6-854JAdvanced-AlgorithmsFall1999/909B7558-FE71-489F-AF3D-16C4A947D424/0/pss3.pdf



%% String encoding
% https://ocw.mit.edu/courses/electrical-engineering-and-computer-science/6-851-advanced-data-structures-spring-2012/index.htm
%% Problemset 9 - a nice, but complicated exercise on suffix tries.

%% https://ocw.mit.edu/courses/electrical-engineering-and-computer-science/6-851-advanced-data-structures-spring-2012/lecture-videos/session-16-strings/
%% Document retrieval. 






% https://ocw.mit.edu/courses/electrical-engineering-and-computer-science/6-006-introduction-to-algorithms-fall-2011/lecture-videos/lecture-9-table-doubling-karp-rabin/
%% Karp-Rabin string search and rolling hash.

%% https://ocw.mit.edu/courses/electrical-engineering-and-computer-science/6-851-advanced-data-structures-spring-2012/lecture-videos/session-16-strings/
%% Document retrieval. 



% https://ocw.mit.edu/courses/electrical-engineering-and-computer-science/6-042j-mathematics-for-computer-science-fall-2010/index.htm
% MIT Mathematics for Computer Science course

% https://ocw.mit.edu/courses/electrical-engineering-and-computer-science/6-006-introduction-to-algorithms-fall-2011/index.htm
% MIT 6.006: Introduction to Algorithms course



\end{document}

