\documentclass[a4paper,12pt]{article}

\usepackage{amsmath,amssymb,amsthm,multicol,tikz,enumitem}
\usepackage{hyperref}
\usepackage{fancyvrb}
%\usepackage{enumerate}
\usepackage[margin=2cm]{geometry}
\usetikzlibrary{calc,arrows.meta}

\newcommand\N{\mathbf{N}}
\newcommand\Q{\mathbf{Q}}
\newcommand\R{\mathbf{R}}
\newcommand\Z{\mathbf{Z}}

\def\ojoin{\setbox0=\hbox{$\bowtie$}%
  \rule[-.02ex]{.25em}{.4pt}\llap{\rule[\ht0]{.25em}{.4pt}}}
\def\leftouterjoin{\mathbin{\ojoin\mkern-5.8mu\bowtie}}
\def\rightouterjoin{\mathbin{\bowtie\mkern-5.8mu\ojoin}}
\def\fullouterjoin{\mathbin{\ojoin\mkern-5.8mu\bowtie\mkern-5.8mu\ojoin}}

% Comment out one or the other

\ifdefined\mysolution
  \newcommand\answer[1]{\mbox{}\\[-15pt]{\color{blue}{#1}}\hfill{\color{blue}$\qed$}\mbox{}\\[-15pt]} 
  \newcommand\ans[1]{{\color{blue}{#1}}}
\else 
  \newcommand\answer[1]{}
  \newcommand\ans[1]{}
\fi


\setlength{\parindent}{0pt}

\begin{document}

\thispagestyle{empty}

\begin{center}
{\bf\Huge 3.\ mājasdarbs} \\[5pt]
Lietišķie algoritmi, 2019.g.\ rudens\\
Termiņš: 2019-12-02
\end{center}

\hrule
\vspace{2pt}
\hrule
\vspace{12pt}




\vspace{6pt}
{\bf 1.uzdevums: Polinomu ``dalīšana stabiņā''.} 
\begin{enumerate}[label=(\alph*)]
\item Doti divi polinomi ar vienu reālu mainīgo $x \in \mathbf{R}$: 
$$A(x) = 3x^2 + 4x + 3,\;\;B(x) = 5x + 6.$$
Dalīt $A(x)$ ar $B(x)$ ar atlikumu. T.i. izteikt $A(x) = Q(x)B(x) + R(x)$, 
kur polinoma $R(x)$ pakāpe ir mazāka par $B(x)$ pakāpi.
\item Dalīt $A(x) = 3x^2 + 4x + 3$ ar $B(x) = 5x+6$ ar atlikumu, ja
polinoma mainīgie, koeficienti un vērtības ir nevis reāli skaitļi, bet
elementi no galīga Galuā lauka $GF(7)$. Arī šajā 
gadījumā jāizsaka $A(x) = Q(x)B(x) + R(x)$, kur arī $Q(x),R(x)$ ir polinomi 
ar koeficientiem, kas pieder atlikumu kopai $GL(7) = \{ 0,1,2,3,4,5,6 \}$.
\end{enumerate}

\vspace{6pt}
{\bf 2.uzdevums: Galuā lauks $GF(2^4)$.}
Galuā lauku $GF(2^4)$ veido visi kubiskie polinomi
$a_3t^3 + a_2t^2 + a_1t + a_0$, kuru 
koeficienti $a_i \in \{ 0,1 \}$; turklāt visu saskaitīšanas un 
reizināšanas darbību rezultātus vienkāršo, dalot ar nereducējamo
veidotājpolinomu $t^4 + t + 1$ (un, ja nepieciešams, darbību gaitā aizstājot 
nepāra koeficientus $a_i$ ar $1$, bet pāru koeficientus ar $0$). Definējam 
divus $GF(2^4)$ elementus:
$$\alpha = t^2 + t + 1,\;\;\beta=t+1.$$
Aprēķināt izteiksmes: {\bf (a)} $\alpha + \beta$, 
{\bf (b)} $\alpha - \beta$,
{\bf (c)} $\beta - \alpha$,
{\bf (d)} $\alpha \cdot \beta$,
{\bf (e)} $\alpha / \beta$,
{\bf (f)} $\beta / \alpha$
{\bf (g)} $\beta^4$.

\vspace{6pt}
{\bf 3.uzdevums: Difī-Helmana diskrēto logaritmu uzdevums.} 
Alise un Bobs nodarbojas ar Difī-Helmana atslēgu apmaiņu. 
Viņi publiskojuši pirmskaitli $p=41$ un primitīvo sakni 
pēc $p$ moduļa: $\alpha = 6$. Pēc tam Alise iedomājusies 
savu privāto atslēgu $a = 5$ un Bobs iedomājies savu privāto 
atslēgu $b = 12$. 

\begin{enumerate}[label=(\alph*)]
\item Kādu publisko atslēgu Alise sūta Bobam?
\item Kādu publisko atslēgu Bobs sūta Alisei?
\item Kāds pēc publisko atslēgu apmaiņas ir viņiem abiem vienlaikus zināmais 
{\em kopīgais noslēpums} (``common secret'')?
\item Zināms, ka Difī-Helmana diskrēto logaritmu uzdevumā jāizvēlas tāds $\alpha$, 
kurš ir {\em primitīvā sakne} pēc $p$ moduļa - citādi diskrēto 
logaritmu uzdevumam daudzi atrisinājumi neder (un tātad uzbrucējs var vieglāk 
atminēt privātās atslēgas). Kura ir mazākā iespējamā primitīvās saknes $\alpha$ vērtība,
ja Alisei un Bobam vairs nepatīk $\alpha=6$? 
\end{enumerate}


\vspace{6pt}
{\bf 4.uzdevums: LP uzdevuma sastādīšana.} 
Rūpnīcas ceham pēc plāna ik dienas
jāizgatavo ne mazāk $8$ krēsli, $4$ soli, $2$ galdi un $8$ ķebļi
(ražošanas plānu drīkst arī pārsniegt).
Ceham pieejamas trīs veidu finiera loksnes $A$, $B$ un $C$, 
kuru izmaksas ir attiecīgi $\$8.50$, $\$9.75$, $\$9.08$. 
Katru no loksnēm var vienā noteiktā veidā sagriezt gabalos, 
iegūstot pa druskai no visu četru mēbeļu daļām, kā parādīts tabulā.
(Piemēram, tabulas kolonnā zem $A$ ir skaitļi $1/16$, 
$1/4$, $1/20$, $1/4$. Tas nozīmē, ka, sagriežot loksni $A$ gabaliņos, 
radīsies $1/16$ no nepieciešamā vienam krēslam,
$1/4$ no nepieciešamā vienam solam, 
$1/20$ no nepieciešamā vienam galdam un
$1/4$ no nepieciešamā vienam ķeblītim.)

\begin{tabular}{|l|c|c|c|} \hline
{\bf Iznākums} & {\bf Loksne $A$} & {\bf Loksne $B$} & {\bf Loksne $C$} \\ \hline
Krēsli & $1/16$ & $1/14$ & $1/18$ \\ \hline
Soli & $1/4$ & $1/4$ & $1/6$ \\ \hline
Galdi & $1/20$ & $1/25$ & $1/30$ \\ \hline
Ķebļi & $1/4$ & $1/3$ & $1/6$ \\ \hline
\end{tabular}

\begin{enumerate}[label=(\alph*)]
\item Sastādīt LP uzdevumu lineāru vienādību/nevienādību sistēmas veidā, izmantojot
iespējami nelielu skaitu mainīgo $x_1,x_2,\ldots$. 
Katram mainīgajam uzrakstīt tā interpretāciju (ko tas saturīgi nozīmē
šajā teksta uzdevumā).
\item Pārveidot LP uzdevumu simpleksalgoritma standartformā. Cik 
tajā ir brīvo mainīgo, cik pamatmainīgo? (Nav nepieciešams šo LP uzdevumu risināt.)
\item Uzrakstīt pirmajā punktā izveidotajam LP uzdevumam duālo uzdevumu, izmantojot
mainīgos $y_1,y_2,\ldots$.
\item Duālā LP uzdevuma mainīgajiem $y_1,y_2,\ldots$ formulējiet tā 
interpretāciju: ko tas saturīgi nozīmē teksta uzdevumā. Interpretācija, 
kas pasaka, ko vajag maksimizēt (vai minimizēt?) duālajā uzdevumā, reizēm 
sarežģīti izsakāma cilvēku valodā, bet šoreiz to jācenšas definēt.
\end{enumerate}

\vspace{6pt}
{\bf 5.uzdevums: Simpleksalgoritms.} 
Dots LP uzdevums: Maksimizēt $2x_1 + 3x_2 + 4x_3$, kur 
$$\left\{ \begin{array}{l}
2x_1 + x_2 + 4x_3 \leq 100\\
x_1 + 3x_2 + x_3 \leq 80\\
x_1,x_2,x_3 \geq 0. \end{array} \right.$$
\begin{enumerate}[label=(\alph*)]
\item Pārveidot šo LP uzdevumu simpleksalgoritma standartformā.
\item Izveidot simpleksalgoritma sākotnējo tabulu; apzīmēt, 
kuri ir brīvie mainīgie, kuri - pamatmainīgie.
\item Veikt simpleksalgoritma soļus. 
\item Uzrakstīt atrisinājumu formā $(x_1,x_2,x_3)=\ldots$ un atrast, kāda 
ir izteiksmes maksimālā vērtība.
\item Uzrakstīt dotajam LP uzdevumam duālo uzdevumu.
\end{enumerate}

\vspace{6pt}
{\bf 6.uzdevums: I-iespēja (atzīmei 10).} 
Apskatām uzdevumu par maksimālo sapārojumu. 
Šajā uzdevumā doti $n$ cilvēki un $n$ uzdevumi ar nosacījumiem, kuri
cilvēki drīkst pildīt kurus uzdevumus. Jānosaka maksimālais uzdevumu
skaits, ko var izpildīt vienlaikus, ja viens cilvēks drīkst pildīt ne vairāk
kā vienu uzdevumu (un vienu uzdevumu drīkst pildīt ne vairāk kā viens
cilvēks). Šo uzdevumu var modelēt ar lineāru programmu ar mainīgajiem
$x_{ij}$ katram $i$, $j$, kur $i$ ir cilvēks, kas drīkst pildīt uzdevumu $j$:
$$\text{Maksimizēt}\;\;\sum\limits_{i,j=1}^{n} x_{ij}$$
ar nosacījumiem
$$\begin{array}{l}
\sum\limits_{j=1}^n x_{ij} \leq 1\;\;\text{katram}\;\;i,\\
\sum\limits_{i=1}^n x_{ij} \leq 1\;\;\text{katram}\;\;j,\\
0 \leq x_{ij},\;\;x_{ij} \leq 1\;\;\text{katram}\;\;i,j.
\end{array}.$$

\begin{enumerate}[label=(\alph*)]
\item
Pierādīt, ka šīs programmas maksimums reālos skaitļos sakrīt ar tās 
maksimumu veselos skaitļos (tas ir, katram atrisinājumam reālos skaitļos, 
kas sasniedz summu $S$, ir atrisinājums veselos skaitļos, kas sasniedz summu,
kas ir vismaz $S$).
\item
Interpretēt (aprakstīt cilvēku valodā) šai problēmai duālajā uzdevumā 
minimizējamo izteiksmi un tās mainīgos.
\end{enumerate}



\end{document}



