\documentclass[a4paper,12pt]{article}

\usepackage{amsmath,amssymb,amsthm,tikz}
\usetikzlibrary{calc,arrows.meta}
\usepackage[margin=20mm]{geometry}

\setlength{\parindent}{0pt}
\setlength{\columnsep}{1cm}

\begin{document}
\thispagestyle{empty}

\twocolumn

\begin{center}
{\Large Assignment 5. 2020-10-14},\\
{\em 12 minutes} 
\end{center}


\vspace{10pt}
{\bf Question 1 (Algorithm to find GCD).} 

This algorithm is often used to find the {\em greatest common divisor}
of nonnegative integers $a$ and $b$.

\[
\begin{array}{rl}
 & \text{\textsc{GCD-One}}(a,b):\\
1 & \text{\textbf{if\ }} a == 0:\\
2 & \hspace{.5cm} \text{\textbf{return\ }} b:\\
3 & \text{\textbf{if\ }} b == 0:\\
4 & \hspace{.5cm} \text{\textbf{return\ }} a:\\
5 & \text{\textbf{while\ }} b > 0:\\
  & \hspace{.5cm} \text{\em (Remainder when $a$ is divided by $b$)}\\
6 & \hspace{.5cm} t = a\;\operatorname{mod}\;b\\
7 & \hspace{.5cm} a = b\\
8 & \hspace{.5cm} b = t\\
9 & \text{\textbf{return\ }} a\\
\end{array}
\]

The worst case time for this algorithm is achieved 
when we input Fibonacci numbers (it has to run the longest relative to the input size). 
For example, if $a = 144$, $b = 89$, then:
$$(144,89) \rightarrow (89,55) \rightarrow (55,34) \rightarrow$$
$$\rightarrow (34,21) \rightarrow (21,13) \rightarrow (13,8) \rightarrow (8,5)$$ 
$$\rightarrow (5,3) \rightarrow (3,2) \rightarrow (2,1) \rightarrow (1,0).$$
At the last step $a = 1$ and $b = 0$, so it returns $a=1$ which equals $\operatorname{gcd}(144,89)$. 

It is known that $n$-th Fibonacci number 
$${\displaystyle F_n \approx \frac{1}{\sqrt{5}}\left(\frac{1+\sqrt{5}}{2} \right)^n}.$$
It grows as a geometric progression.

Write the time complexity of finding GCD of two numbers $(a,b)$ in terms of $n$, 
where $n$ is the total number of digits in numbers $a$ and $b$.\\
{\bf Express your answer, using the ``Big-O-Notation''.}



\vspace{20pt}
{\bf Question 2 (Another algorithm for GCDD).} 

Modify the above algorithm - instead of dividing with remainder, we subtract $b$ from $a$ 
repeatedly (and we swap $a$ and $b$, whenever $b > a$). 

\[
\begin{array}{rl}
 & \text{\textsc{GCD-Two}}(a,b):\\
1 & \text{\textbf{if\ }} a == 0:\\
2 & \hspace{.5cm} \text{\textbf{return\ }} b:\\
3 & \text{\textbf{if\ }} b == 0:\\
4 & \hspace{.5cm} \text{\textbf{return\ }} a:\\
5 & \text{\textbf{while\ }} b > 0:\\
6 & \hspace{.5cm} a = a - b\\
7 & \hspace{.5cm} \text{\textbf{if\ }} b > a:\\
  & \hspace{1.0cm} \text{\em ($a$ becomes $b$ and vice versa)}\\
8 & \hspace{1.0cm} swap(a,b)\\
9 & \text{\textbf{return\ }} a\\
\end{array}
\]

For example, if $a = 75$, $b = 30$ (GCD is $15$), we run it like this:
{\small
$$(75,30) \rightarrow (45,30) \rightarrow (15,30)_{swap} \rightarrow$$
$$ \rightarrow (30,15) \rightarrow (15,15) \rightarrow
(0,15)_{swap} \rightarrow (15,0).$$
}

Write the time complexity of finding GCD of two numbers $(a,b)$ in terms of $n$
(where $n$ is your input length).\\
{\bf Express your answer, using the ``Big-O-Notation''.}




\newpage

{\bf \large Solutions}


\vspace{10pt} 
{\bf Question 1.} Answer: $O(n)$.

The algorithm $\text{\textsc{GCD-One}}(a,b)$ has time complexity $O(n)$.\\
Intuitively, if we have $100$-digit numbers, then we would need $100k$ steps for the algorithm to 
complete - so it is linear. 
(Here we assume that all arithmetic operations take constant time; the algorithm may take longer, if numbers
$a,b$ are so large that they exceed $2^{64}$ or other CPU register size limit.)

It was told in this problem that the worst-case complexity is achieved, if both 
arguments to $\text{\textsc{GCD-Two}}(a,b)$ are Fibonacci numbers. The lengths of $a$ and $b$ 
cannot exceed $n$ digits (in decimal notation), therefore $a,b < 10^n$. 
If any of them is the $k$-th Fibonacci number (e.g. $a = F_k$, then we would spend $c \cdot k$ steps
before the algorithm reaches $F_0 = 0$. We get 
$$F_k \approx \frac{1}{\sqrt{5}}\left(\frac{1+\sqrt{5}}{2} \right)^k < 10^n,$$
$$\left(\frac{1+\sqrt{5}}{2} \right)^k < \sqrt{5} \cdot 10^n,$$
$$k < \frac{\ln (\sqrt{5} \cdot 10^n)}{\ln \frac{1+\sqrt{5}}{2}} = c \cdot n.$$
Therefore, the number of while-loop iterations $k$ is $O(n)$.


\vspace{20pt} 
{\bf Question 2.} Answer: $O(10^n)$.

The worst case (maximum number of subractions) for 
the algorithm $\text{\textsc{GCD-Two}}(a,b)$  happens, if $a = 10^n-1$ 
(the largest $n$-digit integer having $n$ digits $=9$) and $b = 1$. In this case we will subtract $b$ from $a$ 
$O(10^n)$ times.


{\em Note 1:} Time complexity should NOT expressed in 
terms of actual arguments $a$ and $b$, but it only depends on $n$, 
where $n$ denotes the length of its input (total number of digits of $a$ and $b$).

{\em Note 2:} Answer $O(\log n)$ for $\text{\textsc{GCD-One}}(a,b)$ or
$O(n)$ for $\text{\textsc{GCD-Two}}(a,b)$ would be true, if the input for $a,b$ 
is written in {\em unary counting system}. 

\vspace{20pt}
{\bf Grading:} 
\begin{itemize} 
\item Correct answers ($O(n)$ and $O(10^n)$ respectively) \textendash 10 points.
\item Answers $O(\log n)$ and $O(n)$ with some justification (why remainders are 
much faster than subtraction) \textendash 7 points. 
\item $O(\log n)$ and $O(n)$ (but without any justification) \textendash 5 points. 
\item Similar to $O(\log n)$ and $O(n)$ (but $a,b$ used instead of $n$) \textendash 4 points. 
\item Just $O(\log n)$ for the 1st item \textendash 2 points. 
\end{itemize}



\end{document}



