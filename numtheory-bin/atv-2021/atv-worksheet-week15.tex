\documentclass[a4paper,12pt]{article}

\usepackage{amsmath,amssymb,multicol,tikz,enumitem}
\usepackage[margin=2cm]{geometry}
%\usetikzlibrary{calc}
\usepackage{amsmath}
\usepackage{amsthm}
\usepackage{thmtools}
\usepackage{hyperref}
\usepackage{enumerate}
\usepackage{xcolor}
\usepackage{fancyvrb}

\pagestyle{empty}

\newcommand\Q{\mathbf{Q}}
\newcommand\R{\mathbf{R}}
\newcommand\Z{\mathbf{Z}}

\usepackage{array}
\newcolumntype{P}[1]{>{\centering\arraybackslash}p{#1}}

\newcommand\indd{${}$\hspace{20pt}}

\declaretheoremstyle[headfont=\normalfont\bfseries,notefont=\mdseries\bfseries,bodyfont = \normalfont,headpunct={:}]{normalhead}
\declaretheorem[name={Uzdevums}, style=normalhead,numberwithin=section]{problem}

\setcounter{section}{115}

\setlength\parindent{0pt}

\begin{document}

\clearpage
\begin{center}
\parbox{3.5cm}{\flushleft\bf Varbūtības \newline ATV} \hfill {\bf\LARGE Uzdevumi nedēļai \#15} \hfill \parbox{3.5cm}{\flushright\bf 2021-04-16} %\\[2pt]
\end{center}

%\hrule\vspace{2pt}\hrule
\hrule


%1996.6 
\vspace{10pt}
\begin{problem}
Sacensībās piedalās $n$ komandas ($n \geq 3$). Katras divas komandas spēlē tieši vienu spēli savā starpā; 
komanda ikvienā spēlē uzvar ar varbūtību $\frac{1}{2}$ (izredzes uzvarēt nav atkarīgas no tā, kas
ar ko spēlē un neviena spēle nebeidzas ar neizšķirtu).\\
Atrast varbūtību, ka šajās sacensībās neparādās neviena komanda, kura uzvarējusi visas citas, ne arī 
tāda komanda, kura visām citām zaudējusi.
\end{problem}

%1998.9 
\vspace{10pt}
\begin{problem}
Divi draugi katru dienu iet uz ēdnīcu. Katrs no viņiem ienāk ēdnīcā nejauši izvēlētā
laika momentā \textendash{} kaut kad starp 12:00 un 13:00. Viņu atnākšanas laiki netiek 
saskaņoti; tie ir savstarpēji neatkarīgi viens no otra. Katrs no viņiem ēdnīcā pavada tieši $t$
minūtes un pēc tam dodas prom.

Atrast $t$ vērtību pie kuras viņi ēdnīcā sastopas tieši ar varbūtību $p=0.4$: 
cik ilgi viņiem jāuzturas ēdnīcā, lai varbūtība, ka viens no viņiem ienāks ēdnīcā tanī momentā, 
kad tur ir arī otrs, būtu tieši $0.4$.\\
{\em Ieteikums.} Šajā uzdevumā ir jāizmanto vienmērīgais varbūtību sadalījums $60$ minūšu 
intervālā; ienākšanas laiks ar vienādu varbūtību $1/3600$ ir ikvienā no $3600$ šīs stundas sekundēm
(ar varbūtību $1/3600000$ ikvienā no milisekundēm utml.) 
Abu draugu ierašanās laiku pāri var attēlot kā punktu plaknē.
(
\end{problem}

%2000.1.5
\vspace{10pt}
\begin{problem}
Divas kastes satur baltas un melnas lodītes; un abās kastēs esošo lodīšu skaits ir $25$. 
No katras kastes nejauši izvēlas vienu lodīti. Varbūtība, ka abas lodītes ir melnas
ir $27/50$. Atrast varbūtību, ka abas izvēlētās lodītes ir baltas. 
\end{problem}


%2001.1.6
\vspace{10pt}
\begin{problem}
Metamo kauliņu (uz kura ar vienādām varbūtībām var uzmest skaitļus no $1$ līdz $6$) 
meta četras reizes, iegūstot skaitļus $x_1, x_2, x_3, x_4$. 
Atrast varbūtību tam, ka $x_1 \leq x_2 \leq x_3 \leq x_4$ (uzmesto skaitļu virkne
ir nedilstoša). 
\end{problem}




%2001.2.9
\vspace{10pt}
\begin{problem}
Katru no $3 \times 3$ kvadrāta rūtiņām neatkarīgi no citām 
nokrāso zaļu vai sarkanu (katru krāsu izvēlas nejauši 
ar varbūtību $1/2$). Atrast varbūtību, ka šajā $3 \times 3$ kvadrātā neeksistē 
kvadrāts ar $2 \times 2$ rūtiņām, kurā visas rūtiņas būtu sarkanas.
\end{problem}

\end{document}

