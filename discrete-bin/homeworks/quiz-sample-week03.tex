\documentclass[jou]{apa6}

\usepackage[american]{babel}

\usepackage{csquotes}
\usepackage[style=apa,sortcites=true,sorting=nyt,backend=biber]{biblatex}
\DeclareLanguageMapping{american}{american-apa}
\addbibresource{bibliography.bib}


%%%%%%%%%%%%%%%%%%%%%%%%%%%%%%%%%%%%%%%%
%% Discrete Structures
%% The start of RBS stuff
%%%%%%%%%%%%%%%%%%%%%%%%%%%%%%%%%%%%%%%%

% Working internal and external links in PDF
\usepackage{hyperref}
% Extra math symbols in LaTeX
\usepackage{amsmath}
\usepackage{gensymb}
\usepackage{amssymb}
% Enumerations with (a), (b), etc.
\usepackage{enumerate}

\let\OLDitemize\itemize
\renewcommand\itemize{\OLDitemize\addtolength{\itemsep}{-6pt}}

\usepackage{etoolbox}
\makeatletter
\preto{\@verbatim}{\topsep=3pt \partopsep=3pt }
\makeatother

% These sizes redefine APA for A4 paper size
\oddsidemargin 0.0in
\evensidemargin 0.0in
\textwidth 6.27in
\headheight 1.0in
\topmargin -24pt
\headheight 12pt
\headsep 12pt
\textheight 9.19in

\usepackage{xcolor}

\title{Sample Quiz 3}
\author{Discrete Structures, Fall 2020}
\affiliation{RBS}

\leftheader{Discrete Sample Quiz 3}

\abstract{%
}

%\keywords{}

\begin{document}

\thispagestyle{empty}

\twocolumn
{\Large Discrete Sample Quiz 3}

\vspace{6pt}
{\bf Question 1.} {\em Definition.} A Boolean formula is a 
{\em Conjunctive Normal Form (CNF)}, 
if it is a conjunction of one or more clauses, 
where a clause is a disjunction of literals. Each literal 
is either a variable ($u,v,\ldots$) or its negation ($\neg u, \neg v, \ldots$). 

Find the CNF computing the following truth table for Boolean 
expression $E(a,b,c)$:\\
\begin{tabular}{ c | c | c | c }
$a$ & $b$ & $c$ & $E(a,b,c)$  \\ \hline
{\tt T} & {\tt T} & {\tt T} & {\tt T} \\ \hline
{\tt T} & {\tt T} & {\tt F} & {\tt F} \\ \hline
{\tt T} & {\tt F} & {\tt T} & {\tt F} \\ \hline
{\tt T} & {\tt F} & {\tt F} & {\tt F} \\ \hline
{\tt F} & {\tt T} & {\tt T} & {\tt T} \\ \hline
{\tt F} & {\tt T} & {\tt F} & {\tt F} \\ \hline
{\tt F} & {\tt F} & {\tt T} & {\tt F} \\ \hline
{\tt F} & {\tt F} & {\tt F} & {\tt T} \\ \hline
\end{tabular}




\vspace{6pt}
{\bf Question 2.} The following CNF is given:
$$E = (a \vee \neg c) \wedge (b \vee \neg b).$$
Find $2$ Boolean expressions equivalent to $E$:\\[3pt]
{\bf (A)} $a \rightarrow \neg b$,\\[3pt]
{\bf (B)} $a \rightarrow c$,\\[3pt]
{\bf (C)} $\neg c \rightarrow \neg a$,\\[3pt]
{\bf (D)} $(a \wedge c) \rightarrow b$,\\[3pt]
{\bf (E)} $c \rightarrow a$,\\[3pt]
{\bf (F)} $a \rightarrow (c \rightarrow b)$,\\[3pt]
{\bf (G)} $\neg a \rightarrow \neg c$.

\vspace{6pt}
{\bf Question 3.} We have six statements about Python programs $p \in \mathcal{P}$
that convert inputs $i \in \mathbb{Z}^{+}$ into results $r \in \mathbb{Z}^{+}$.  
A Python program may either loop indefinitely or halt (i.e.\ it eventually stops).

\begin{enumerate}
\item Some programs return the correct result for all possible inputs and 
they never loop indefinitely.
\item For any program one can find another program such that it returns
the same result for the same inputs as the first one (or loops indefinitely, 
{\color{red} \bf iff} the first program does the same).
\item There is a program that only loops indefinitely for at most finitely many inputs (or maybe none at all), 
but for all other inputs it produces the correct result.
\item There is at least one Python program that always halts, and for sufficiently large inputs it produces
the correct result, but it may err for some small-size inputs.
\item For a program to produce a correct result for some input $i$ it is strictly necessary to halt.\\
\item A Python program always produces exactly one result for the given input provided that it halts.
\end{enumerate}

We use these $3$ predicates:\\
$A(p_1,i_2,r_3)$ is true iff
Python program $p_1 \in \mathcal{P}$ receives input $i_2 \in \mathbb{Z}^{+}$ and outputs 
result $r_3 \in \mathbb{Z}^{+}$.\\
$H(p_1,i_2)$ is true iff program $p_1 \in \mathcal{P}$ receives input $i_2$ and halts (i.e.\ does not
loop indefinitely).\\
$C(i_1,r_2)$ is true iff for input $i_1$ the correct result is $r_2$. 

{\em Note.} Write your answer as a comma-separated list: For example, 
{\tt F,E,D,C,B,A} tells that the 1st statement is {\bf (F)}, the 2nd one
is {\bf (E)}, $\ldots$, the last one is {\bf (A)}.


\begin{enumerate}[(A)]
\item ${\displaystyle \forall p \in \mathcal{P}\; \forall i \in \mathbb{Z}^+\; \forall r \in \mathbb{Z}^+,}$\\
${\displaystyle \left( A(p,i,r) \wedge C(i,r) \,\rightarrow\, H(p,i) \right) }$.
\item ${\displaystyle \forall p_1 \in \mathcal{P}\; \exists p_2 \in \mathcal{P}\;
\forall i \in \mathbb{Z}^{+}\; \textcolor{red}{\forall r \in \mathbb{Z}^{+},}}$\\
${\displaystyle \left( (\neg H(p_1,i) \wedge \neg H(p_2,i)) \vee 
(A(p_1,i,r) \leftrightarrow A(p_2,i,r)) \right)}$
\item ${\displaystyle \exists p \in \mathcal{P}\; \exists N \in \mathbb{Z}^{+}\; 
\forall i \in \mathbb{Z}^{+}\; \textcolor{red}{\exists r \in \mathbb{Z}^{+}},}$\\
${\displaystyle \left( ( i \leq N \wedge \neg H(p,i) ) \vee (A(p,i,r) \wedge C(i,r)) \right)}$.
\item ${\displaystyle \forall p \in \mathcal{P}\; \forall i \in \mathbb{Z}^{+}\; \forall r_1 \in \mathbb{Z}^{+}\;
\forall r_2 \in \mathbb{Z}^{+},}$\\ 
${\displaystyle \left( H(p,i) \wedge A(p,i,r_1) \wedge  A(p,i,r_2) \,\rightarrow\, r_1 = r_2 \right)}$.
\item ${\displaystyle \exists p \in \mathcal{P}\; \exists N \in \mathbb{Z}^{+}\; \forall i \in \mathbb{Z}^{+}\;
\textcolor{red}{\forall r \in \mathbb{Z}^{+}},}$\\
${\displaystyle \left( H(p,i) \,\wedge\, ( A(p,i,r) \wedge (i > N) \,\rightarrow\, C(i,r) ) \right)}$.
\item ${\displaystyle \exists p \in \mathcal{P}\; \forall i \in \mathbb{Z}^{+}\; \forall r \in \mathbb{Z}^{+},}$\\
${\displaystyle \left( H(p,i) \,\wedge\, (A(p,i,r) \rightarrow C(i,r) \right)}$. 
\end{enumerate}



\vspace{10pt}
{\bf Question 4.} There is a set of $4$ students $S = \{ s_1, s_2, s_3, s_4 \}$ and 
a set of $2$ chairs $C = \{ c_1, c_2 \}$. 
Find, how many such functions $f\,:\,S \rightarrow C$ exist, 
how many of them are injective, surjective and bijective.\\
{\em Note.} Your answer should be a comma-separated list of 
4 numbers (and 3 commas): {\tt total,injective,surjective,bijective}.


\vspace{10pt}
{\bf Question 5.}
There is a predicate $S(x, y, z)$ defined for triplets
of positive integers, 
$S: \mathbb{Z}^{+} \times \mathbb{Z}^{+} \times \mathbb{Z}^{+} \rightarrow \{ \mathtt{T}, \mathtt{F} \}$. 
$S(x,y,z)$ is true iff $x \cdot y = z$.\\
Express these statements about positive integers
using only $S(x,y,z)$, Boolean operations and quantifiers.\\
{\em Note.} Please use only the predicate $S(\ldots)$ in your answers, 
avoid any other predicates or relations (such as equality, divisibility, etc.).

\begin{enumerate}[(A)] 
\item $x/y = z$,
\item $x = 1$,
\item $x = y$, 
\item $x$ is divisible by $y$ (i.e. $y \,\mid\, x$). 
\item $x$ has odd number of positive divisors.
\item $x$ is not a prime.
\item $x$ is a prime.
\end{enumerate}


\section{Answers}

\vspace{6pt}
{\bf Question 1.} Answer: 
\begin{align}
 & \textcolor{red}{(\neg a \vee \neg b \vee c)} \wedge 
\textcolor{blue}{(\neg a \vee b \vee \neg c)} \wedge 
\textcolor{blue}{(\neg a \vee b \vee c)} \;\wedge \nonumber \\
\wedge\; & \textcolor{red}{(a \vee \neg b \vee c)} \wedge 
(a \vee b \vee \neg c) \nonumber
\end{align}
Every value {\tt F} in the truth table produces one 
clause in this expression. For example $E(\mathtt{T},\mathtt{T},\mathtt{F}) = \mathtt{F}$
is addressed by a disjunction $(\neg a \vee \neg b \vee c)$. 

If we combine red and blue clauses, we can rewrite this into a more compact expression, 
which is also CNF. Note, that this is not the only way to rewrite:
$$\textcolor{red}{(\neg b \vee c)} \wedge 
\textcolor{blue}{(\neg a \vee b)} \wedge
(a \vee b \vee \neg c).$$

\vspace{6pt}
{\bf Question 2.} Answer: {\bf (E)}, {\bf (G)}.\\
$(a \vee \neg c) \wedge (b \vee \neg b) = (a \vee \neg c) \wedge \mathtt{T} = (a \vee \neg c)$. 
This is same as $c \rightarrow a$ or the contrapositive variant of the same implication:
$\neg a \rightarrow \neg c$.

\vspace{6pt}
{\bf Question 3.} Answer:\\
\begin{tabular}{|c|c|c|c|c|c|} \hline
1 & 2 & 3 & 4 & 5 & 6 \\ \hline
{\bf (F)} & {\bf (B)} & {\bf (C)} & {\bf (E)} & {\bf (A)} & {\bf (D)} \\ \hline
\end{tabular} 

\vspace{3pt}
In {\bf (B)} we should write: the second program loops indefinitely {\bf iff} 
the first program does so. If we write {\bf if} instead, it 
would mean only one-way conditional, which does not mean 
equivalence between the programs $p_1$ and $p_2$, but something more complicated.

In {\bf (B)} you could write $(H(p_1,i) \leftrightarrow H(p_2,i))$
instead of $(\neg H(p_1,i) \wedge \neg H(p_2,i))$ or 
even omit any reference to $H(\ldots)$ predicates. 
The notation $(H(p_1,i) \leftrightarrow H(p_2,i))$ stresses the fact that $p_1$ and $p_2$ behave in the same way. 
But in fact the equivalence $(A(p_1,i,r) \leftrightarrow A(p_2,i,r))$ implicitly 
states the fact that $p_1$ and $p_2$ are defined on the same inputs.

Please note that {\bf (B)} should {\bf NOT} have quantifier  
 $\textcolor{red}{\exists r \in \mathbb{Z}^{+}}$ 
(instead of  $\textcolor{red}{\forall r \in \mathbb{Z}^{+}}$), because we need
equivalence for all positive integers $r$. If we only care about the
actual result of $p$ on input $i$, then it is possible to ``cheat'' the logic formula: 
pick result $r$ which is impossible for both $p_1$ and $p_2$; then $A(p_1,i,r)$ 
and $A(p_2,i,r)$ would both be false (and thus equivalent).

In {\bf (C)} one cannot replace $\textcolor{red}{\exists r \in \mathbb{Z}^{+}}$
with $\textcolor{red}{\forall r \in \mathbb{Z}^{+}}$ (as in an earlier draft of this test), 
because it does not make sense to ask that a program $p$ on input $i$ produces
{\bf each} positive integer result; and all results turn out to be correct. 
(Such a formula may have some useful meaning if predicates $A,H,C$ have different
interpretations, but for Python programs this is technically correct, but 
has an absurd meaning: that a given program produces any result whatsoever.)

In {\bf (E)} it is absolutely necessary to write a quantifier 
$\textcolor{red}{\forall r \in \mathbb{Z}^{+}}$, because 
leaving $r$ as a free variable (without a quantifier) would 
mean that the Python program always produces the same result
(the value of $r$). It is not 


\vspace{6pt}
{\bf Question 4.} Answer: {\tt 16,0,14,0}\\
Please note that there can be no injections (and therefore also bijections): 
students will always collide \textendash{} run to the same chair, 
since there are fewer chairs than students.

The total number of functions is $2^4 = 16$. Each student
can choose between $2$ chairs, so it is $2 \times 2 \times 2 \times 2 = 16$. 
Moreover, there are only $2$ ``constant'' functions (where all four
students go to $c_1$ or to $c_2$). They are not surjective, because
some chairs stay unoccupied in this case. 
All the other $16 - 2 = 14$ functions from $S$ (4 elements) to $C$ (2 elements)
are surjections. 

\vspace{6pt}
{\bf Question 5.} Here we write all $7$ statements with predicate $S(x,y,z)$ and quantifiers. 

\begin{enumerate}[(A)]
\item $S(z,y,x)$ (or, equivalently, $S(y,z,x)$). 
\item $\exists y \in \mathbb{Z}^{+},\; S(x,y,y)$.\\
Use the property: $x = 1$ can multiply with an $y$, and does not increase/decrease. \\
You could also write simply $S(x,x,x)$. Indeed, $x$ is the only 
positive integer such that $x \cdot x = x$. (Allowing $x = 0$ would 
spoil this, because $0^2 = 0$.)
\item $\exists z \in \mathbb{Z}^{+},\; S(x,z,y) \wedge S(z,z,z)$.\\
So, $x = y$ iff there is a $z = 1$ such that $x \cdot z = y$.
\item $\exists d \in \mathbb{Z}^{+},\; S(y,d,x)$.
\item $\exists y \in \mathbb{Z}^{+},\; S(y,y,x)$.\\ 
This means that $y^2 = x$ for some $y$; 
hence $x$ is a full square and should have odd number of divisors.
\item $\exists y \in \mathbb{Z}^{+}\;\exists z \in \mathbb{Z}^{+},\;(S(y,z,x) \wedge \neg S(y,y,y) \wedge \neg S(z,z,z))$.\\
This means that a non-prime $x$ can be expressed as a product of two positive integers $y,z$, 
where neither of them is $1$.)
\item $\forall y \in \mathbb{Z}^{+}\;\forall z \in \mathbb{Z}^{+},\;(\neg S(y,z,x) \vee S(y,y,y) \vee S(z,z,z))$.\\
We apply negation to the previous quantifier formula; all $\exists$ quantifiers turn into $\forall$; 
formulas are changed according to De Morgan's laws.
\end{enumerate}










\end{document}


