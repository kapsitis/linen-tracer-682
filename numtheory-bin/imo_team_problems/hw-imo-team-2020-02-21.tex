\documentclass[11pt]{article}
\usepackage{ucs}
\usepackage[utf8x]{inputenc}
\usepackage{changepage}
\usepackage{graphicx}
\usepackage{amsmath}
\usepackage{gensymb}
\usepackage{amssymb}
\usepackage{enumerate}
\usepackage{tabularx}
\usepackage{lipsum}
\usepackage{amsthm}
\usepackage{thmtools}


\usepackage{fontspec} % loaded by polyglossia, but included here for transparency 
\usepackage{polyglossia}

\usepackage{xeCJK}
\setCJKmainfont{SimSun}
\setmainlanguage{russian} 
\setotherlanguage{english}

\newfontfamily\cyrillicfont[Script=Cyrillic]{Times New Roman}
\newfontfamily\cyrillicfontsf[Script=Cyrillic]{Arial}
\newfontfamily\cyrillicfonttt[Script=Cyrillic]{Courier New}

\oddsidemargin 0.0in
\evensidemargin 0.0in
\textwidth 6.27in
\headheight 1.0in
\topmargin 0.0in
\headheight 0.0in
\headsep 0.0in
%\textheight 9.69in
\textheight 9.00in
 
\setlength\parindent{0pt}

\newenvironment{myenv}{\begin{adjustwidth}{0.4in}{0.4in}}{\end{adjustwidth}}
\renewcommand{\abstractname}{Anotācija}
\renewcommand\refname{Atsauces}

%\newenvironment{uzdevums}[1][\unskip]{%
%\vspace{3mm}
%\noindent
%\textbf{#1:}
%\noindent}
%{}

% (4;10;12;17)
% (p1.19;5;15;20)

% http://tex.stackexchange.com/questions/196961/thmtools-declaration-for-theorem-and-proof
\declaretheoremstyle[headfont=\normalfont\bfseries,notefont=\mdseries\bfseries,bodyfont = \normalfont,headpunct={:}]{normalhead}
\declaretheorem[name={Uzdevums}, style=normalhead,numberwithin=section]{problem}

%\def\changemargin#1#2{\list{}{\rightmargin#2\leftmargin#1}\item[]}
\def\changemargin#1#2{\list{}\item[]}
\let\endchangemargin=\endlist 


\newcommand{\subf}[2]{%
  {\small\begin{tabular}[t]{@{}c@{}}
  #1\\#2
  \end{tabular}}%
}



\newcounter{alphnum}
\newenvironment{alphlist}{\begin{list}{(\Alph{alphnum})}{\usecounter{alphnum}\setlength{\leftmargin}{2.5em}} \rm}{\end{list}}

\newenvironment{zhtext}{\fontfamily{MS PGothic}\selectfont}{\par}


\makeatletter
\let\saved@bibitem\@bibitem
\makeatother

\usepackage{bibentry}
%\usepackage{hyperref}

\newenvironment{tulkojums}[1][\unskip]{%
\begin{changemargin}{8mm}{8mm}
\fontsize{9}{11}
\selectfont
\textbf{#1:}
}
{ 
\fontsize{12}{14}
\selectfont
\end{changemargin}
}

\setcounter{section}{2}


\begin{document}

\begin{center}
{\Large \bf Uzdevumi 2020.g. 21.\ februāra nodarbībai}
\end{center}

\vspace{10pt}

\begin{problem}
Determine all positive integers $n$ such that $n$ has a multiple
with all non-zero digits in its decimal notation. 
\end{problem}

\begin{problem}
A {\em wobbly number} is a positive integer whose digits are alternately nonzero and
zero with the units digit being nonzero. Determine all positive 
integers that do not divide any wobbly numbers. 
\end{problem}

\begin{problem}
Let $n$ be a given integer with $n \geq 4$. For a positive integer $m$ 
let $S_m$ denote the set 
$\{ m, m+1, \ldots, m+n-1 \}$. 
Determine the minimum value of $f(n)$ such that every $f(n)$-element subset of $S_m$
(for every $m$) contains at least three pairwise relatively 
prime elements.
\end{problem}

\begin{problem}
By $\sigma(k)$ we denote the sum of all positive divisors of $k$ (including $1$ and $k$ itself). 
For every positive integer $n$, prove that 
$$\frac{\sigma(1)}{1} + \frac{\sigma(2)}{2} + \ldots \frac{\sigma(n)}{n} \leq 2n.$$
\end{problem}

\vspace{10pt}
{\em Note.} In the last two problems let
$\text{gpf}(n)$ denote the greatest prime factor of an integer $n$.
We also define $\text{gpf}(1) = \text{gpf}(-1) = 1$, and $\text{gpf}(0)$ is undefined.

\begin{problem}
Show that there exist infinitely many positive 
integers $n$ such that $\text{gpf}(n^4 + 1)$ is greater than $2n$. 
\end{problem}


\begin{problem}
Find all polynomials $P(n)$ with integer coefficients satisfying both properties:
\begin{itemize}
\item $P(n^2) \neq 0$ for all integers $n = 0,1,2,\ldots$ and 
\item There exists $M >0$ such that $\text{gpf}(P(n^2)) - 2n \leq M$ for all integers $n = 0,1,2,\ldots$.
\end{itemize}
\end{problem}


\end{document}


