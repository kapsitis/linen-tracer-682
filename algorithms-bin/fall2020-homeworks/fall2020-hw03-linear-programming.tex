\documentclass[a4paper]{article}
\usepackage{ucs}
\usepackage[utf8x]{inputenc}
\usepackage{changepage}
\usepackage{graphicx}
\usepackage{amsmath}
\usepackage{gensymb}
\usepackage{amssymb}
\usepackage{enumerate}
\usepackage{tabularx}
\usepackage{lipsum}
\usepackage{amsthm}
\usepackage{thmtools}
\usepackage{xcolor}
\usepackage{caption}
\captionsetup[table]{name=Tabula}


%\documentclass[jou]{apa6}
%\usepackage[american]{babel}

%\usepackage{csquotes}
%\usepackage[style=apa,sortcites=true,sorting=nyt,backend=biber]{biblatex}
%\DeclareLanguageMapping{american}{american-apa}
%\addbibresource{bibliography.bib}


%%%%%%%%%%%%%%%%%%%%%%%%%%%%%%%%%%%%%%%%
%% Discrete Structures
%% The start of RBS stuff
%%%%%%%%%%%%%%%%%%%%%%%%%%%%%%%%%%%%%%%%

% Working internal and external links in PDF
\usepackage{hyperref}
% Extra math symbols in LaTeX
\usepackage{amsmath}
\usepackage{gensymb}
\usepackage{amssymb}
% Enumerations with (a), (b), etc.
\usepackage{enumerate}
\usepackage[framemethod=TikZ]{mdframed}
\usepackage{xcolor}

\let\OLDitemize\itemize
\renewcommand\itemize{\OLDitemize\addtolength{\itemsep}{-6pt}}

\usepackage{etoolbox}
\makeatletter
\preto{\@verbatim}{\topsep=3pt \partopsep=3pt }
\makeatother

% These sizes redefine APA for A4 paper size
\oddsidemargin 0.0in
\evensidemargin 0.0in
\textwidth 6.27in
\headheight 1.0in
\topmargin -24pt
\headheight 12pt
\headsep 12pt
\textheight 9.19in



\setlength{\parindent}{0pt}
\setlength{\columnsep}{1cm}


\begin{document}

\twocolumn

\begin{center}
{\Large 3. Mājasdarbs}\\
{\Large Lineārā programmēšana}
\end{center}

%\begin{changemargin}{10pt}{10pt}
%{\footnotesize
%Vairāki uzdevumi šajā mājasdarbā iespaidojušies no MIT Open Courseware:
%
%}
%\end{changemargin}


{\bf Termiņš:} 2020.gada 11.decembris; līdz vakaram\\ (23:59:59 EET).\\
{\bf Iesniegšanas veids:} E-studiju vide.


\vspace{10pt}
{\bf 1.uzdevums (Simpleksalgoritma soļi).}

No noliktavas ik dienu jāizved uz galapunktu $480$ kastes
ar āboliem, $400$ kastes ar bumbieriem un $230$ kastes ar cidonijām. 
Katru no produktiem var piegādāt pa pastu, ar kurjeru vai ar dronu. 
Kurjeri katru dienu var piegādāt līdz $420$ kastēm ar jebkuru produktu, 
droni katru dienu var pie\-gā\-dāt līdz $250$ kastēm ar jebkuru produktu. 
Parastais pasts var piegādāt jebkuru kastu skaitu. 
Katram piegādes veidam atšķiras pašizmaksa, āt\-rums un tātad \textendash{} arī 
peļņa. Noliktavas peļņa par vienu kasti (atkarībā no produkta veida 
un pie\-gā\-des metodes) 
parādīta tabulā: 

\vspace{4pt}
\begin{tabular}{|l|c|c|c|} \hline
Produkts & Pasts & Kurjers & Drons \\ \hline
Āboli & $4$ EUR & $12$ EUR & $7$ EUR \\ \hline
Bumbieri & $4$ EUR & $13$ EUR & $9$ EUR \\ \hline
Cidonijas & $8$ EUR & $14$ EUR & $11$ EUR \\ \hline
\end{tabular}

\vspace{4pt}
{\bf (A)} Uzrakstīt lineāru programmu, kas ir standartformā, lai ar to varētu sākt simpleksu metodi.\\
{\bf (B)} Uzrakstīt visus simpleksa metodes soļus, izmantojot Danciga metodi, nomainot kārtējo brī\-vo mainīgo 
ar standartmainīgo. (Šī metode ir izplatītākā. Nomaiņas metožu jeb {\em Pivot rules} pār\-ska\-tu sk. 
šeit: \url{https://bit.ly/3fDMzkn}).\\
{\bf (C)} Atrast, cik kastes jāpiegādā ar katru no me\-to\-dēm, lai maksimizētu peļņu. 
(Jūsu atbildes var nebūt veseli skaitļi \textendash{} veselu skaitļu programmēšanas uzdevumu 
ar simpleksa metodi parasti nevar atrisināt.) 



%% (Only Simplex algorithm; the course focuses on "Advanced algorithms" in general)
%% Added to the simplex chapter (5.2)
%% https://ocw.mit.edu/courses/electrical-engineering-and-computer-science/6-046j-design-and-analysis-of-algorithms-spring-2015/assignments/MIT6_046JS15_pset8.pdf
% https://ocw.mit.edu/courses/electrical-engineering-and-computer-science/6-046j-design-and-analysis-of-algorithms-spring-2015/index.htm





%% Newest LP Course
%% https://ocw.mit.edu/courses/electrical-engineering-and-computer-science/6-854j-advanced-algorithms-fall-2008/
%% HW3 (P4: Scooter problem)
%% https://ocw.mit.edu/courses/electrical-engineering-and-computer-science/6-854j-advanced-algorithms-fall-2008/assignments/ps3.pdf
%% (Added to the simplex chapter 5.1, 5.2, 5.3, 5.4)



%% Combinatorial Optimization stuff - in 5.1, 5.2, 5.3, 5.4
%% https://ocw.mit.edu/courses/mathematics/18-433-combinatorial-optimization-fall-2003/assignments/a3.pdf
%% https://www.chegg.com/homework-help/questions-and-answers/problem-4-meat-packing-plant-produces-480-hams-400-pork-bellies-230-picnic-hams-every-day--q19404846







\vspace{20pt}
{\bf 2.uzdevums (Tuvināšana ar taisni).}

Plaknē doti šādi $5$ punkti:
$$(1, 3), (2, 5), (3, 6), (5, 8), (8, 13).$$
Apzīmējam tos attiecīgi ar $(x_1,y_1)$, $\ldots$, $(x_5,y_5)$.\\
{\bf (A)} Atrast taisni $ax + by = c$, kas iet iespējami tuvu šiem punktiem (neviena taisne nevar 
iet cauri visiem šiem punktiem). Uzrakstīt li\-ne\-ā\-ru programmu (tā nav jārisina!), lai atrastu taisni, 
kura minimizē maksimālo absolūto kļūdu:
$$\max_{i \in \{1,\ldots,5\}} |ax_i + by_i - c|.$$
{\bf (B)} Vai eksistē lineāra programma, kas minimizē pilno kvadrātisko kļūdu: 
$$\sum\limits_{i \in \{1,\ldots,5\}} (ax_i + by_i - c)^2.$$

(Parodija par uzdevumu 7.8, sk. p.240 \url{https://bit.ly/2YcgCtW}.)



\vspace{20pt}
{\bf 3.uzdevums (Veselo skaitļu programmēšana).}

Izpildīt ``branch and bound'' algoritmu: Atrast $x_i$, 
kas var pieņemt tikai veselas vērtības $0$ vai $1$, šādai lineārajai
programmai:
$$\min\;x_1 + 4x_2 + x_3 + x_4,$$
kur ir spēkā ierobežojumi: 
$$\left\{ \begin{array}{l} 
x_1 + x_2 + x_3 \geq 2,\\
x_1 + x_2 + x_4 \geq 2,\\
0 \leq x_1 \leq 1,\; 0 \leq x_2 \leq 1,\\
0 \leq x_3 \leq 1,\;0 \leq x_4 \leq 1.\\
\end{array} \right.$$
{\footnotesize
{\em Piezīme.} ``Branch and bound'' metode meklē atrisinājumu 
starp visām $16$ teorētiski iespējamām bitu $(x_1,x_2,x_3,x_4)$ 
kombinācijām (lielākam parametru skaitam to var būt daudz vairāk par $16$). 
Lai saprastu, kuras bitu kombinācijas vajag pārbaudīt (un kuras var atmest), 
dažus mainīgos $x_i$ piesaista noteiktām vērtībām ($0$ vai $1$) un 
{\em relaksē} uzdevumu līdz parastai lineārai programmai. 
Ja arī relaksētā lineārā programma nedod labāku risinājumu par labāko pašreiz zināmo, 
visu šo zaru (ar visiem apakšvariantiem) atmet. 
Tas parasti ļauj jau uzreiz ārkārtīgi strauji samazināt meklējumu telpu, 
aprobežoties tikai ar tām bitu kombinācijām, kas dod risinājumu, kas tuvs optimālajam. 
Algoritms tiks aplūkots 23.novembra nodarbībā; relaksēto LP risināšanai varat izmantot
jebkuru programmēšanas rīku vai Web lietotni, kas risina lineāras programmas.
}

 





% https://homepages.rpi.edu/~mitchj/handouts/interior_html/interior.html


\vspace{20pt}
{\bf 4.uzdevums (Ceļošana ar motorolleru)}

Komanda ar $n$ cilvēkiem grib iespējami 
ātri no\-kļūt no punkta $A$ līdz punktam $B$, 
virzoties pa ceļu garumā $d$. 
Katrs dalībnieks var iet kājām un viņiem ir uz visiem 
viens motorollers (sākumā tas atrodas punktā $A$,  
vienlaikus var pārvietot vienu cilvēku). 
Katram cilvēkam $i$ ($1 \leq i \leq n$) 
ir dots gan viņa iešanas ātrums $w_i$, 
gan ātrums, braucot ar motorolleru, $s_i$.

Uzdevums ir atrast veidu, kā visiem $n$ cilvēkiem veikt ceļu
no $A$ uz $B$ tā, lai minimizētu laiku no kustības sākuma 
līdz brīdim, kurā pēdējais cilvēks nonāk punktā $B$. 
Motorolleru jebkurš var 
atrast ceļmalā un to var savākt jebkurš cits no komandas. 
Komandas biedri var iet vai braukt ar 
motorolleru arī pretējā virzienā (pa ceļu 
atpakaļ uz punktu $A$), ja tas viņiem kaut kā palīdz.

\vspace{4pt}
{\bf (A)} Aplūkojiet gadījumu, kad $n = 3$, 
$w_1 = w_2 = 1$, $s_1 = s_2 = 6$, $w_3 = 2$, $s_3 = 8$ un
attālums $d = 100$. Atrast ātrāko veidu, 
kā viņiem visiem trim veikt
attālumu $d$, nonākot no $A$ uz $B$.\\
{\bf (B)} Vispārīgajam gadījumam (jebkurai $n$ 
vēr\-tī\-bai un patvaļīgiem ātrumiem) izveidot lineāru programmu, 
kuras atrisinājums ir apakšējais 
no\-vēr\-tē\-jums laikam, 
kas vajadzīgs $n$ personu komandai 
pārvietojoties par attālumu $d$. 
Šai li\-ne\-ā\-ra\-jai programmai jābūt iespējami mazai 
(mainīgo un ierobežojumu skaits ir $O(n)$), 
tas nav atkarīgs no $d$ vai arī no konstrukcijā 
izmantoto posmu (pār\-sē\-ša\-nos un virziena maiņu) skaita.\\
{\bf (C)} Uzrakstīt citu variantu lineārajai programmai, ja neviens nedrīkst iet vai braukt
pretējā virzienā (virzienā no $B$ uz $A$). Vai šāds ierobežojums var palielināt 
ceļojumam nepieciešamo lai\-ku? (Parodija 
par 4.uzd. no \url{https://bit.ly/3ectpC2}.)


%% Older LP Course
%% https://ocw.mit.edu/courses/electrical-engineering-and-computer-science/6-854j-advanced-algorithms-fall-2005/
%% HW6 (P3: Polytope and ball problem; P4: Students and Desks problem)
%% https://ocw.mit.edu/courses/electrical-engineering-and-computer-science/6-854j-advanced-algorithms-fall-2005/assignments/ps6.pdf








\vspace{20pt}
{\bf 5.uzdevums (I-Iespēja)}

Definēsim ``maksimālā sapārojuma'' uzdevumu. 
Doti $n$ cilvēki un $n$ uzdevumi; zināms, kuri cilvēki drīkst pildīt kurus uzdevumus. 
Jānosaka maksimālais uzdevumu
skaits, ko var izpildīt vienlaikus, ja viens cilvēks drīkst pildīt ne vairāk
kā vienu uzdevumu (un vienu uzdevumu drīkst pildīt ne vairāk kā viens
cilvēks). Šo uzdevumu var modelēt ar lineāru programmu ar mainīgajiem
$x_{ij}$ katram $i$,$j$, kur $i$ ir cilvēks, kas drīkst pildīt uzdevumu $j$:
$$\text{Maksimizēt:}\;\;\sum\limits_{ij} x_{ij},$$
ar nosacījumiem
$$\sum\limits_{j} x_{ij} \leq 1,\;\mbox{katram $i$};$$
$$\sum\limits_{i} x_{ij} \leq 1,\;\mbox{katram $j$};$$
$$0 \leq x_{ij} \leq 1,\;\mbox{katram $i,j$}.$$

Pierādīt, ka šīs programmas maksimums reālos skaitļos sakrīt ar tās 
maksimumu veselos skaitļos (tas ir, katram atrisinājumam reālos skaitļos, kas
sasniedz summu $S$, ir atrisinājums veselos skait\-ļos, kas sasniedz vismaz
tikpat lielu summu $S$).



\end{document}

