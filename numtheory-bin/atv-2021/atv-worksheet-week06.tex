\documentclass[a4paper,12pt]{article}

%\usepackage{amsmath,amssymb,multicol,tikz,enumitem}
\usepackage[margin=2cm]{geometry}
%\usetikzlibrary{calc}
\usepackage{amsmath}
\usepackage{amsthm}
\usepackage{thmtools}
\usepackage{hyperref}
\usepackage{enumerate}
\usepackage{xcolor}
\usepackage{fancyvrb}

\pagestyle{empty}

\newcommand\Q{\mathbf{Q}}
\newcommand\R{\mathbf{R}}
\newcommand\Z{\mathbf{Z}}

\usepackage{array}
\newcolumntype{P}[1]{>{\centering\arraybackslash}p{#1}}

\newcommand\indd{${}$\hspace{20pt}}

\declaretheoremstyle[headfont=\normalfont\bfseries,notefont=\mdseries\bfseries,bodyfont = \normalfont,headpunct={:}]{normalhead}
\declaretheorem[name={Uzdevums}, style=normalhead,numberwithin=section]{problem}

\setcounter{section}{102}

\setlength\parindent{0pt}

\begin{document}

\clearpage
\begin{center}
\parbox{3.5cm}{\flushleft\bf Skaitļu teorija \newline ATV} \hfill {\bf\LARGE Uzdevumu lapa \#6} \hfill \parbox{3.5cm}{\flushright\bf 2021-02-10} %\\[2pt]
\end{center}

%\hrule\vspace{2pt}\hrule
\hrule


\vspace{20pt}
Dirihlē principu ({\em Pigeonhole principle}) bieži var izmantot neapzināti. 
Vienkāršākajā formā to formulē šādi: ``Ja $n$ trusīšus jāizvieto $n-1$ būrīšos, tad kādā no būrīšiem atradīsies vismaz divi trusīši.''
Šis princips nepiedāvā nekādu metodi atrast šo būrīti (un svarīgi, ka trusīšu skaits tur ir {\em vismaz divi}, tātad varētu būt arī lielāks).

Tālāk dosim precīzus formulējumus vairākiem Dirihlē principa variantiem:
\begin{itemize}
\item Ja kopa no $m$ elementiem ir sadalīta $n$ grupās, un  $n<m$, tad kādā no grupām ir ne mazāk kā divi elementi.
\item Ja $m$-elementu kopā ir izvēlēti $n$ elementi, un $n<m$, tad kopā ir vismaz viens elements, kurš atšķiras no izvēlētajiem elementiem.
\item Ja kopa, kura satur vairāk kā $mn$ elementus, ir sadalīta $n$ grupās, tad kādā no grupām ir ne mazāk kā $m+1$ elements.
\end{itemize}

\vspace{20pt}
{\bf Atlikumu salīdzināšana pēc moduļa.}

\vspace{10pt}
Risinot tālākos uzdevumus, izmantosim faktu, ka atlikumu skaits pēc moduļa $n$ ir vienāds ar $n$.


\vspace{10pt}
\begin{problem}
%8.1. 
Pierādiet, ka no patvaļīgiem trim veseliem skaitļiem var izvēlēties divus tā, ka to summa dalās ar $2$.
\end{problem}

\vspace{10pt}
\begin{problem}
%8.2. 
Pierādiet, ka patvaļīgiem veseliem skaitļiem  reizinājums
\[ (a-b)(a-c)(a-d)(b-c)(b-d)(c-d) \]
dalās ar $12$.
\end{problem}


\vspace{10pt}
\begin{problem}
%8.3. 
Pierādiet, ka no patvaļīgiem pieciem veseliem skaitļiem 
var izvēlēties tādus trīs, kuru summa dalās ar $3$.
\end{problem}


\vspace{10pt}
\begin{problem}
%8.4. 
Pierādīt: ja $a,b,c$ ir veseli skaitļi, $n>3$ ir naturāls skaitlis, 
tad eksistē tāds naturāls skaitlis $k$, 
ka neviens no skaitļiem $k+a$, $k+b$, $k+c$ nedalās ar $n$.
\end{problem}


\vspace{10pt}
\begin{problem}
%8.5. 
Doti $12$ dažādi divciparu skaitļi. 
Pierādiet, ka no tiem var izvēlēties $2$ skaitļus, 
kuru starpība ir divciparu skaitlis, un kurš ir 
uzrakstāms ar diviem vienādiem cipariem.
\end{problem}


\vspace{10pt}
\begin{problem}
%8.6. 
Pierādiet, ka no patvaļīgiem $5$ naturāliem skaitļiem 
var izvēlēties $2$ skaitļus tā, 
lai to kvadrāti dotu vienādus atlikumus pēc moduļa $7$.
\end{problem}


\vspace{10pt}
\begin{problem}
%8.7. 
Pierādīt: ja $a,a+d,a+2d,\ldots,a+(n-1)d$ 
ir savstarpēji pirmskaitļi ar skaitli $n$, 
tad skaitļi $d$ un $n$ nav savstarpēji pirmskaitļi. 
\end{problem}


\vspace{10pt}
\begin{problem}
%8.8. 
Doti $37$ veseli skaitļi. Pierādīt, ka no tiem var 
atrast septiņus skaitļus, kuru summa dalās ar $7$.
\end{problem}


\vspace{10pt}
\begin{problem}
%8.9. 
Dots, ka $x,y,z,t$ ir nepāra skaitļi. Pierādīt, ka reizinājums
\[ (x-y) \cdot (x-z) \cdot (x-t) \cdot (y-z) \cdot (y-t) \cdot (z-t) \]
\begin{enumerate}[(a)]
\item dalās ar $64$,
\item dalās ar $256$,
\item dalās ar $768$.
\end{enumerate}
\end{problem}


\vspace{10pt}
\begin{problem}
%8.10. 
Pierādīt, ka katram pirmskaitlim $p$ var atrast tādus 
naturālus skaitļus $x$ un $y$, ka $x^2 + y^2 + 1$ dalās ar $p$.
\end{problem}


\vspace{10pt}
\begin{problem}
%8.11. 
Doti $19$ veseli skaitļi. Pierādīt, ka no tiem var 
izvēlēties tieši $10$ skaitļus tā, ka visu izraudzīto 
skaitļu summa dalās ar $10$.
\end{problem}

 
 
\vspace{20pt}
{\bf Skaitļu kombināciju salīdzināšana}

\vspace{10pt}
Turpmākajos uzdevumos Dirihlē princips ir jāpielieto 
skaitļu pāriem, skaitļu virknēm un citām skaitļu kombinācijām.


\vspace{10pt}
\begin{problem}
%8.12. 
Plaknē doti pieci punkti ar veselām koordinātēm. Pierādiet, 
ka no tiem var izraudzīties divus punktus tā, ka nogrieznim, 
kura gala punkti atrodas šajos punktos, viduspunkta 
koordinātes ir veseli skaitļi,
\end{problem}




\vspace{10pt}
\begin{problem}
%8.13. 
Telpā novietots izliekts daudzskaldnis tā, ka tā virsotnes 
atrodas punktos, kuru koordinātes ir veseli skaitļi. 
Daudzskaldņa iekšpusē, uz skaldnēm un šķautnēm nav citu punktu, 
kuru koordinātes ir veseli skaitļi. Pierādiet, ka daudzskaldnim 
ir ne vairāk kā $8$ virsotnes.
\end{problem}



\vspace{10pt}
\begin{problem}
%8.14. 
Izliekta $63$-stūra virsotnēs uzrakstīti dažādi naturāli skaitļi, 
kuri nepārsniedz $1997$. Pierādiet, ka atradīsies divas 
daudzstūra diagonāles, kuru galapunktos uzrakstīto skaitļu 
starpības ir vienādas.
\end{problem}



\vspace{10pt}
\begin{problem}
%8.15. 
Pierādiet, ka no $10$ patvaļīgiem dažādiem divciparu skaitļiem 
var izvēlēties divas dažādas nešķeļošas skaitļu grupas tā, 
ka skaitļu summas abās grupās ir vienādas.
\end{problem}



\vspace{10pt}
\begin{problem}
%8.16. 
Dota tāda augošu naturālu skaitļu virkne $\left( a_i \right)$, 
ka katrs naturāls skaitlis ir vai nu šīs virknes loceklis 
vai arī divu dažādu šīs virknes locekļu summa. Pierādīt, ka $a_n \leq n^2$  
visiem naturāliem skaitļiem $n$.
\end{problem}



\vspace{10pt}
\begin{problem}
%8.17. 
Pierādiet, ka katram naturālam skaitlim $n>1$  
eksistē tāds vesels skaitlis $M$, ka nekādiem veseliem skaitļiem 
$x$ un $y$ skaitlis $x^n + y^n - M$ nedalās ar $n^2$.
\end{problem}



\vspace{10pt}
\begin{problem}
%8.18. 
Divdesmit četri studenti risināja $25$ uzdevumus. 
Pasniedzējam ir tabula ar izmēriem $24 \times 25$, 
kurā ierakstīts kādus uzdevumus risinājis katrs students. 
Izrādījās, ka katru uzdevumu risināja vismaz viens students. 
Pierādiet, ka var atzīmēt dažus uzdevumus tā, ka katrs 
students risināja pāra skaitu atzīmēto uzdevumu.
\end{problem}




\vspace{10pt}
\begin{problem}
%8.19. 
Šaha dēlīša $8 \times 8$ katrā rūtiņā ir ierakstīts naturāls skaitlis. 
Atļauts izdalīt patvaļīgu kvadrātu ar izmēriem $3 \times 3$  
vai $4 \times 4$  un visus skaitļus, kas atrodas šajā kvadrātā, 
palielināt par $1$. 
Atkārtoti izpildot šādas operācijas, mēs gribam panākt, 
lai visi rūtiņās ierakstītie skaitļi dalītos ar $10$. 
Vai vienmēr to ir iespējams izdarīt?
\end{problem}




\vspace{10pt}
\begin{problem}
%8.20. 
Doti naturāli skaitļi $A,B,C$, kas nepārsniedz $100$. 
Pierādiet, ka eksistē trīs veseli skaitļi $a,b,c$, 
kas pēc moduļa nepārsniedz $18$, 
visi vienlaicīgi nav vienādi ar $0$, un kuriem izpildās vienādība
$aA + bB + cC = 0$.
\end{problem}



\vspace{10pt}
\begin{problem}
%8.21.
Taisnstūris, kura izmēri ir $300 \times 1000$, 
sadalīts vienības kvadrātiņos. Patvaļīgās $30$ kvadrātu virsotnēs 
novietotas vienādas lodes. Pierādīt, ka var izvēlēties 
nešķeļošu ložu grupas (ne vairāk kā $10$ lodes katrā grupā) tā, 
ka to smaguma centri sakrīt.
\end{problem}



\vspace{10pt}
\begin{problem}
%8.22. 
Doti $17$ dažādi naturāli skaitļi; neviens no tiem nepārsniedz $25$. 
Pierādīt, ka no tiem var izvēlēties divus tādus, 
kuru reizinājums ir vesela skaitļa kvadrāts.
\end{problem}



\vspace{10pt}
\begin{problem}
%8.23. 
Naturāla skaitļa pierakstā nav nuļļu, un tas sastāv no $28$ cipariem.
Pierādīt, ka dažus tā ciparus var izsvītrot tā, lai pāri palikušie 
veidotu skaitli, kas dalās ar $101$.
\end{problem}
 
 
\vspace{10pt}
\begin{problem}
%8.24. 
Jānis sareizināja divus trīsciparu skaitļus. Vai var gadīties, 
ka visi cipari, kas ietilpst reizinātājos un reizinājumā, ir dažādi?
\end{problem}



\vspace{10pt}
\begin{problem}
%8.25. 
Jānis raksta uz tāfeles skaitļus. Pirmais skaitlis ir $23$, 
katrs nākošais ir divas reizes lielāks par iepriekšējo. 
(Tātad pirmie uzrakstītie skaitļi ir $23; 46; 92; 184;\ldots$.) 
Vai starp Jāņa uzrakstītajiem skaitļiem atradīsies divi tādi skaitļi, 
kuru pirmie cipari ir vienādi, otrie \textendash{} arī vienādi, 
priekšpēdējie \textendash{} arī vienādi? 
(Uzskatām, ka Jānis turpina rakstīšanu neierobežoti ilgi).
\end{problem}


\vspace{10pt}
\begin{problem}
%8.26.  
Vai naturālos skaitļus
\begin{enumerate}[(a)]
\item no 1 līdz 12 ieskaitot,
\item no 1 līdz 50 ieskaitot
\end{enumerate}
var tā sadalīt pa pāriem, lai visas pāros ieejošo skaitļu summas 
būtu dažādas, un katra no tām būtu pirmskaitlis?
(Piemēram, skaitļus no $1$ līdz $6$ varētu sadalīt tā:  
$1+2 = 3$, $3+4 =7$, $5+6 = 11$).
\end{problem}



\vspace{10pt}
\begin{problem}
%8.27. 
Sniegbaltīte uzrakstīja rindā kaut kādā kārtībā visus 
naturālos skaitļus no $1$ līdz $7$, katru skaitli tieši vienu reizi. 
To pašu izdarīja katrs no $7$ rūķīšiem. Katrs rūķītis atrada, 
kurus skaitļus viņš uzrakstījis tajās pašās vietās 
(pirmajā, otrajā, trešajā, ...) kā Sniegbaltīte. 
Visiem rūķīšiem šādu skaitļu daudzumi izrādījās atšķirīgi. 
Vai var gadīties, ka neviens rūķītis neuzrakstīja skaitļus 
tieši tādā pašā kārtībā kā Sniegbaltīte?
\end{problem}


\end{document}

