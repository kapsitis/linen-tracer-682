\documentclass[a4paper]{article}
\usepackage{ucs}
\usepackage[utf8x]{inputenc}
\usepackage{changepage}
\usepackage{graphicx}
\usepackage{amsmath}
\usepackage{gensymb}
\usepackage{amssymb}
\usepackage{enumerate}
\usepackage{tabularx}
\usepackage{lipsum}
\usepackage{amsthm}
\usepackage{thmtools}


%% COLORED TABLES
\usepackage{colortbl}% http://ctan.org/pkg/colortbl
\usepackage{xcolor}% http://ctan.org/pkg/xcolor
\usepackage{booktabs}
\newcommand{\ra}[1]{\renewcommand{\arraystretch}{#1}}
\colorlet{tableheadcolor}{gray!25} % Table header colour = 25% gray
\newcommand{\headcol}{\rowcolor{tableheadcolor}} %
\colorlet{tablerowcolor}{gray!10} % Table row separator colour = 10% gray
\newcommand{\rowcol}{\rowcolor{tablerowcolor}} %
\usepackage{multirow}


\usepackage{fontspec} % loaded by polyglossia, but included here for transparency
\usepackage{polyglossia}

%\usepackage{xeCJK}
%\setCJKmainfont{SimSun}
%\setmainlanguage{russian}
%\setotherlanguage{english}

%\newfontfamily\cyrillicfont[Script=Cyrillic]{Times New Roman}
%\newfontfamily\cyrillicfontsf[Script=Cyrillic]{Arial}
%\newfontfamily\cyrillicfonttt[Script=Cyrillic]{Courier New}

\oddsidemargin -1.27cm
\evensidemargin -1.27cm
%\textwidth 6.27in
\textwidth 18.46cm
\topmargin -1.27cm
\headheight 0.0cm
\headsep 0.0cm
\textheight 27.16cm

\setlength\parindent{0pt}




% http://tex.stackexchange.com/questions/196961/thmtools-declaration-for-theorem-and-proof
\declaretheoremstyle[headfont=\normalfont\bfseries,notefont=\mdseries\bfseries,bodyfont = \normalfont,headpunct={:}]{normalhead}
\declaretheorem[name={Uzdevums}, style=normalhead,numberwithin=section]{problem}

\def\changemargin#1#2{\list{}{\rightmargin#2\leftmargin#1}\item[]}
\let\endchangemargin=\endlist


\newcommand{\subf}[2]{%
  {\small\begin{tabular}[t]{@{}c@{}}
  #1\\#2
  \end{tabular}}%
}



\newcounter{alphnum}
\newenvironment{alphlist}{\begin{list}{(\Alph{alphnum})}{\usecounter{alphnum}\setlength{\leftmargin}{2.5em}} \rm}{\end{list}}

\newenvironment{zhtext}{\fontfamily{MS PGothic}\selectfont}{\par}


\makeatletter
\let\saved@bibitem\@bibitem
\makeatother

\usepackage{bibentry}
\usepackage{hyperref}

\pagenumbering{gobble} 

\begin{document}

\arrayrulecolor[HTML]{CCCCCC}
\renewcommand{\arraystretch}{1.2}
\begin{table}[ht!]\centering
{\small
\begin{tabular*}{18.46cm}{@{}|p{10.35cm}|p{7.25cm}|@{}} \hline
\headcol \multicolumn{2}{|p{18.05cm}|}{
{\bf NMS Izlases nodarbības: Skaitļu teorija.} Sk. \texttt{http://www.dudajevagatve.lv/nt/index.html}
} \\ \hline
\multicolumn{2}{|p{18.05cm}|}{
{\bf Definīcijas:} Veseliem $a$ un $d$ ($d \neq 0$) rakstām $d\,\mid a$, ja $a$ dalās ar $d$. Atlikumu, $a$ dalot ar $b$, 
apzīmē ar $a\;\operatorname{mod}\;b$.\newline
Ja veseli skaitļi $a$ un $b$ dod vienādus atlikumus, dalot ar $m$, raksta $a \equiv b\;(\operatorname{mod}\,m)$: $a$ un $b$ ir {\em kongruenti} pēc moduļa $m$.
} \\ \hline
\textbf{Aritmētikas pamatteorēma:} Katru $n \in \mathbb{N}$ var tieši vienā veidā izteikt kā pirmskaitļu 
pakāpju reizinājumu: $n=p_1^{a_1}p_2^{a_2}\cdots{}p_k^{a_k}$. & 
$2016 = 2^{5}\cdot{}3^{2}\cdot{}7$; $\mathbf{2017} = 2017^1$; $2018=2^1\cdot{}1009^1$; $2019=3^1\cdot{}673^1$; $2020=2^2\cdot{}5^1\cdot{}101^1$ \\ \hline
\textbf{Dalītāju skaits:} Katram $n=p_1^{a_1}p_2^{a_2}\cdots{}p_k^{a_k}$ pozitīvo dalītāju skaits, 
ieskaitot $1$ un $n$, ir $d(n)=(a_1+1)\cdots(a_k+1)$. &
$60=2^2\cdot{}3^1\cdot{}5^1$ ir $(2+1)\cdot{}(1+1)\cdot{}(1+1) = 12$ dalītāji. \\ \hline
\textbf{Dalītāju skaits:} Skaitlis $n \in \mathbb{N}$ ir pilns kvadrāts tad un tikai tad, ja tam ir nepāru skaits 
pozitīvu dalītāju. &
Piemēri: $100$ ir pilns kvadrāts, tam ir $9$ dalītāji. $1000$ nav pilns kvadrāts, tam ir $16$ dalītāji. \\ \hline
\multicolumn{2}{|p{18.05cm}|}{
\cellcolor[HTML]{E1FFE1}
{\bf BW.2016.11} Kopa $A$ sastāv no $2016$ dažādiem skaitļiem, visi šo skaitļu
pirmreizinātāji ir mazāki par $30$. Pierādīt, ka kopā $A$ var atrast tādus $4$ dažādus skaitļus $a$, $b$,
$c$ un $d$, ka $abcd$ ir naturāla skaitļa kvadrāts.
} \\ \hline
{\bf Pirmskaitļu pārbaudes algoritms (ļoti lēns lieliem $n$)}\newline
{\scshape isPrime}($n$)\newline
1.\hspace{1em}{\bf for} $d=2$ {\bf to} $\lfloor \sqrt{n} \rfloor$\newline
2.\hspace{1em}\hspace{2em}{\bf if} $n\;\mbox{mod}\;d\;==\;0$\hspace{1em}// $n\;\mbox{mod}\;d$ {\em apzīmē atlikumu, $n$ dalot ar $d$}\newline
3.\hspace{1em}\hspace{2em}\hspace{2em}{\bf return} {\scshape false}\newline
4.\hspace{1em}{\bf return} {\scshape true} &
Ja $n=2017$, tad $\sqrt{2017} \approx 44.91$. Apakšējā veselā daļa ir $44$.
$2017$ nedalās ar $2,3,\ldots,44$ $\Rightarrow$ $2017$ ir pirmskaitlis.
(Varētu nedaudz uzlabot, izlaižot pāru dalītājus $d>2$ vai dalot tikai ar pirmskaitļiem. 
Bet lieliem skaitļiem izmanto {\em Miller-Rabin} (1980.g.)
vai {\em Agrawal–Kayal–Saxena} (2002.g.) algoritmus.) \\ \hline 
{\bf Eiklīda algoritms LKD atrašanai:}\newline
{\scshape Euclid}($a,b$)\newline
1.\hspace{1em}{\bf if} $b\;==\;0$\newline
2.\hspace{1em}\hspace{2em}{\bf return} $a$\newline
3.\hspace{1em}{\bf else} {\bf return} {\scshape Euclid}($b, a\;\mbox{mod}\;b$) &
Funkcijas, kuras izsauc pašas sevi, bet ar citiem argumentiem, sauc par {\em rekursīvām}. Eiklīda
algoritms ir rekursīvs. Piemēram, {\scshape Euclid}($30,21$) =
{\scshape Euclid}($21,9$) = {\scshape Euclid}($9,3$) = {\scshape Euclid}($3,0$) = 3. \\ \hline
\multicolumn{2}{|p{17.00cm}|}{
{\bf Bezū (Bézout) lemma:} Ja naturālu skaitļu $a$ un $b$ lielākais kopīgais dalītājs ir $d$, 
tad eksistē veseli skaitļi $x$ un $y$, kuriem $ax + by = d$. Visi citi skaitļi, ko 
var izteikt formā $ax + by$, dalās ar $d$. 
} \\ \hline
\multicolumn{2}{|p{17.00cm}|}{
{\footnotesize
Ja $a$ un $b$ ir savstarpēji pirmskaitļi, tad eksistē tādi veseli $x,y$, kam $ax+by=1$; citiem vārdiem, 
ar $a$ un $b$ centu monētām var nomaksāt jebkuru naudas summu. ``Bezū koeficientus'' $x,y$ var atrast ar
pielāgotu Eiklīda algoritmu. Piemērs: $a=99$, $b=78$, $\operatorname{LKD}(a,b)=3$. Meklējam 
$x,y$, kam $ax+by=3$.\newline
$\left\{ \begin{array}{l}
1 \cdot a + 0 \cdot b = 99 \\
0 \cdot a + 1 \cdot b = 78 
\end{array} \right.$ 
$\begin{array}{c}
\mbox{\em no 1.rindas}\\
\mbox{\em atņem 2.rindu}
\end{array}$ 
$\left\{ \begin{array}{lcl}
1 \cdot a - 1 \cdot b = 21 \\
0 \cdot a + 1 \cdot b = 78 
\end{array} \right.$ 
$\begin{array}{c}
\mbox{\em no 2.rindas}\\
\mbox{\em atņem $3\times$1.rindu}
\end{array}$ 
$\left\{ \begin{array}{lcl}
1 \cdot a - 1 \cdot b = 21 \\
-3 \cdot a + 4 \cdot b = 15 
\end{array} \right.$ $\Rightarrow$
$\left\{ \begin{array}{lcl}
4 \cdot a - 5 \cdot b = 6 \\
-3 \cdot a + 4 \cdot b = 15 
\end{array} \right.$ $\Rightarrow$
$\left\{ \begin{array}{lcl}
4 \cdot a - 5 \cdot b = 6 \\
-11 \cdot a + 14 \cdot b = 3 
\end{array} \right.$ Tātad $(-11)\cdot 99 + 14 \cdot 78 = 3$ jeb $x=-11$, $y=14$.
} } \\ \hline
\multicolumn{2}{|p{18.05cm}|}{
\cellcolor[HTML]{E1FFE1}
{\bf LV.VO.2014.10.2} Atrast visas tādas vesela skaitļa $n$ vērtības, kurām gan 
$\frac{n^3+3}{n+3}$, gan $\frac{n^4+4}{n+4}$ ir veseli skaitļi.
} \\ \hline
\multicolumn{2}{|p{18.05cm}|}{
{\bf Apgalvojums par polinomu dalīšanu ar atlikumu:} Jebkuriem polinomiem $A(x)$ un $B(x)$ eksistē to ``dalījums'' $Q(x)$ un ``atlikums'' 
$R(x)$, t.i.\ tādi polinomi, kam $A(x)=Q(x)\cdot{}B(X)+R(x)$ un
$R(x)$ pakāpe ir mazāka par $B(x)$ pakāpi.
} \\ \hline
\multicolumn{2}{|p{18.05cm}|}{
{\footnotesize
Polinomos $A(x),B(x)$ atrod vecākos locekļus un dala tos --- iegūst $Q(x)$ kārtējo locekli. Pēc tam pārveido:  
$\frac{n^3+3}{n+3}=\frac{n^2(n+3)-3n^2+3}{n+3}=n^2+\frac{-3n^2+3}{n+3}=n^2+\frac{-3n(n+3)+9n+3}{n+3}=n^2-3n+\frac{9n+3}{n+3}=n^2-3n+\frac{9(n+3)-27+3}{n+3}=n^2-3n+9+\frac{-24}{n+3}$. Iegūstam, ka $A(n)=n^3+3$ un $B(n)=n+3$ dalījums ir $Q(n)=n^2-3n+9$, bet atlikums $R(n)=-24$. (Tā kā 
$B(n)=n+3$ ir 1.pakāpes polinoms, tad $R(n)$ ir 0.pakāpes polinoms: konstante $-24$. Iegūstam, ka $\frac{-24}{n+3}$ ir vesels 
jeb $n+3$ ir kāds no skaitļa $24$ dalītājiem.
}
} \\ \hline 
\multicolumn{2}{|p{18.05cm}|}{
\cellcolor[HTML]{E1FFE1}
{\bf BW.TST.2016.16} Kāda ir izteiksmes $\operatorname{LKD}\left(n^2+3,(n+1)^2+3\right)$
lielākā iespējamā vērtība naturāliem $n$?
} \\ \hline
\multicolumn{2}{|p{18.05cm}|}{
{\footnotesize
Eiklīda algoritmu lieto, dalot polinomus ar atlikumu:
$\operatorname{LKD} \left( n^2 + 3, n^2 + 2n + 4 \right) = 
\operatorname{LKD} \left( n^2 + 3, 2n+1 \right) = 
\operatorname{LKD}\left( 2n^2 + 6, 2n+1 \right) = 
\operatorname{LKD} \left( -n + 6, 2n+1 \right) = \operatorname{LKD} \left(n-6, 13 \right)$. Ja $n-6$ dalās ar $13$, tad 
$\operatorname{LKD}\left(n^2+3,(n+1)^2+3\right)=13$. Citos gadījumos LKD ir $1$.
}
} \\ \hline
{\bf Teorēma par inverso elementu:} Ja $p$ ir pirmskaitlis, tad katram $a \not\equiv 0\;(\operatorname{mod}\,p)$
eksistē tāds $a^{-1}$, ka $a^{-1}\cdot{}a \equiv 1\;(\operatorname{mod}\,p)$ &
Ja $p=7$, tad $1^{-1}=1$, $2^{-1}=4$, $3^{-1}=5$, $4^{-1}=2$, $5^{-1}=3$, $6^{-1}=6$.  \\ \hline
{\bf Ķīniešu atlikumu teorēma:} Ja $n_1,\ldots,n_k$ ir pa pāriem savstarpēji pirmskaitļi, 
tad jebkuriem atlikumiem $x_1,\ldots,x_k$ eksistē atrisinājums $x$, kurš dod vajadzīgos atlikumus
$x_i$, dalot ar $n_i$. T.i. $0 \leq x < n_1n_2\cdots{}n_k$ un 
$x \equiv x_i\;(\operatorname{mod}\,n_i)$ katram $i=1,\ldots,n$.
& Aplūkojam savstarpējus pirmskaitļus $2,3,5$:\newline $\left\{ \begin{array}{c}
x\equiv{}1\;(\operatorname{mod}\,2) \\
x\equiv{}2\;(\operatorname{mod}\,3) \\
x\equiv{}3\;(\operatorname{mod}\,5) 
\end{array} \right.$ $\Leftrightarrow$ $x \equiv 23\;(\operatorname{mod}\,30)$. \\ \hline
{\bf Apgalvojums par pirmskaitļu bezgalīgo skaitu.} Pirmskaitļu $2,3,5,\ldots$ ir bezgalīgi daudz. 
(Pierādījums no pretējā: ja būtu galīgs skaits, tad $p_1p_2\cdots{}p_k+1$ 
nedalītos ne ar vienu no tiem.) & 
Eksistē cik patīk garas $\mathbb{N}$ apakšvirknes bez pirmskaitļiem. 
(Piemēram, $m!+2, m!+3, m!+m$ satur $m-1$ saliktu skaitli.) \\ \hline
\multicolumn{2}{|p{18.05cm}|}{
\cellcolor[HTML]{E1FFE1}
Ir bezgalīgi daudzi tādi pirmskaitļi $p$, kam $p \equiv 3\;(\operatorname{mod}\,4)$. (Līdzīgi, ir 
bezgalīgi daudzi pirmskaitļi $p$, kam $p \equiv 5 \;(\operatorname{mod}\,6)$) 
} \\
\multicolumn{2}{|p{18.05cm}|}{
Pierādījums no pretējā: Ja to ir galīgs skaits, tad apzīmē visu to reizinājumu ar $P$ 
un aplūko $4P-1$. $4P-1$ dod atlikumu $3$, dalot ar $4$ --- tātad nevar sastāvēt tikai no pirmreizinātājiem, 
kas visi dod atlikumu $1$, dalot ar $4$.
} \\ \hline
\cellcolor[HTML]{E1FFE1}
{\bf USA.MO.2008.1} Pierādīt, ka jebkuram naturālam $n$ eksistē $n+1$ savstarpēji pirmskaitļi 
$k_0,k_1,\ldots,k_n$, kas visi lielāli par $1$ un kuriem $k_0k_1\cdots{}k_n-1$ ir divu pēc kārtas 
sekojošu naturālu skaitļu reizinājums. &
Ja sekojoši naturāli skaitļi ir $t, t+1$, vai starp $P(t)=t(t+1)+1$ vērtībām var būt tādas, kurām ir
patvaļīgi daudz dažādu pirmreizinātāju? \\ \hline
\multicolumn{2}{|p{18.05cm}|}{
{\bf Definīcija:} Skaitļus formā $M_n=2^n-1$ sauc par {\em Mersena (Mersenne) skaitļiem}. 
Ja turklāt $M_n$ ir pirmskaitlis, tad to sauc par {\em Mersena pirmskaitli}.
} \\ \hline
{\bf Apgalvojums par Mersena pirmskaitļiem:} Lai $M_n=2^n-1$ būtu pirmskaitlis, ir 
{\em nepieciešami}, lai pats $n$ būtu pirmskaitlis.\newline
({\em Pavisam zināmi 51 Mersena pirmskaitļi. Lielākais ir $2^{82,589,933}-1$, ko atrada 2018.g. decembrī.
Tas ir arī lielākais šobrīd zināmais pirmskaitlis.}) &
Ja $n$ dalās reizinātājos, tad arī pakāpju starpība $2^n-1$ dalās reizinātājos. 
Piemēram, $2^{15}-1 = (2^5)^3-1^3=(2^5-1)((2^5)^2+2^5+1)$. Arī, 
piemēram, $2^{11}-1=2047=23\cdot{}89$. \\ \hline
\end{tabular*}
}
\end{table}


\arrayrulecolor[HTML]{CCCCCC}
\renewcommand{\arraystretch}{1.2}
\begin{table}[ht!]\centering
{\small
\begin{tabular*}{18.46cm}{@{}|p{10.35cm}|p{7.25cm}|@{}} \hline
{\bf Definīcija:} Skaitļus formā $F_n=2^{2^n}+1$ sauc par {\em Fermā skaitļiem}. 
Ja turklāt $F_n$ ir pirmskaitlis, tad to sauc par {\em Fermā pirmskaitli}.\newline
({\em Ja $m$ ir kāds nepāru dalītājs $d>1$, tad $2^m+1$ nevar būt pirmskaitlis. 
Teiksim, $2^{24}+1=(2^8)^3+1^3$ dalās reizinātājos pēc $a^3+b^3=(a+b)(a^2-ab+b^2)$ identitātes.})
&
Šobrīd zināmi pieci Fermā pirmskaitļi: $F_0=2^1+1=3$, $F_1=2^2+1=5$, 
$F_2=2^4+1=17$, $F_3=2^8+1=257$, $F_4=2^{16}+1=65537$. Bet 
$F_5=2^{32}+1=4,294,967,297=641\cdot{}6,700,417$ \\ \hline
\cellcolor[HTML]{E1FFE1}
{\bf Andreescu.2006.1.78} Dažādiem naturāliem $m$ un $n$, Fermā skaitļi 
$F_m$ un $F_n$ ir savstarpēji pirsmkaitļi.\newline ({\em Piemēram, tā kā 
$F_5$ dalās ar $641$, tad neviens cits Fermā skaitlis ar $641$ nedalās.}) & 
Atkārtoti lietojot kvadrātu starpības formulu $a^2-b^2$, var pamatot, ka
$F_m-2$ dalās ar $F_n$, ja $m>n$. Tādēļ pēc Eiklīda algoritma. 
$\operatorname{LKD}(F_m,F_n)=\operatorname{LKD}((F_m-2)+2,F_n)=
\operatorname{LKD}(2,F_n)=1$. \\ \hline
{\bf Mazā Fermā teorēma:} Ja $p$ ir pirmskaitlis un $\operatorname{gcd}(a,p)=1$, tad 
$a^{p-1} \equiv 1\;(\operatorname{mod}\,p)$. &
$1^6 \equiv 2^6 \equiv 3^6 \equiv 4^6 \equiv 5^6 \equiv 6^6 \equiv 1\;(\operatorname{mod}\,7)$. \\ \hline
\multicolumn{2}{|p{18.05cm}|}{
\cellcolor[HTML]{E1FFE1}
{\bf BW2016.3} Kuriem naturāliem $n=1,\ldots{}6$ vienādojumam
$a^n + b^n = c^n + n$ eksistē atrisinājums veselos skaitļos?
} \\ \hline
{\bf Teorēma par primitīvo sakni:} Katram pirmskaitlim $p$ eksistē tāds 
$a$, kuram kongruenču klases $a^1,a^2,\ldots,a^{p-1}$ pieņem visas
vērtības $1,2,\ldots,p-1$. &
Ja $p=7$, tad $3^k$ pieņem visus iespējamos atlikumus, dalot ar $7$ (izņemot pašu $7$):\newline
$3^k \equiv$ $3$, $2$, $6$, $4$, $5$, $1\;(\operatorname{mod}\,7)$ ja $k=1,\ldots,6$. \\ \hline
\multicolumn{2}{|p{18.05cm}|}{
\cellcolor[HTML]{E1FFE1}
{\bf BW.2016.5} Dots pirmskaitlis $p > 3$, kuram $p \equiv 3\;(\operatorname{mod}\,4)$. Dotam naturālam
skaitlim $a_0$ virkni $a_0,a_1,\ldots$ definē kā $a_n=a_{n-1}^{2^n}$ visiem $n=1,2,\ldots$. Pierādīt, ka $a_0$ 
var izvēlēties tā, ka apakšvirkne $a_{N},a_{N+1},a_{N+2},\ldots$ nav konstanta pēc moduļa $p$ nevienam naturālam $N$.
} \\ \hline
{\bf Definīcija:}
{\em Eilera funkcija} $\varphi(n)$ apzīmē, cik ir veselu skaitļu $x \in [1,n]$, kas ir savstarpēji pirmskaitļi 
ar $n$. &
Pirmskaitļiem $\varphi(p)=p-1$. Pirmskaitļu pakāpēm $\varphi(p^k)=p^k-p^{k-1}$.  
\\ \hline
{\bf Definīcija:}
Funkciju $f(n)$, kas definēta naturāliem skaitļiem  
sauc par {\em multiplikatīvu}, ja jebkuriem diviem savstarpējiem pirmskaitļiem $a,b$: $f(ab)=f(a)f(b)$.
& Eilera funkcija ir multiplikatīva. Piemēram $\varphi(100)=\varphi(4)\varphi(25)=(4-2)(25-5)=2\cdot{}20=40$. \\ \hline
{\bf Eilera teorēma:} Ja $a$ un $n$ ir savstarpēji pirmskaitļi, tad 
$a^{\varphi(n)} \equiv 1\;(\operatorname{mod}\,n)$. &
Ja $a$ nedalās ar $2$ un $5$, tad $a^k$ decimālpieraksta pēdējie divi cipari ir 
tādi paši kā $a^{k+\varphi(100)}=a^{k+40}$. \\ \hline
\cellcolor[HTML]{E1FFE1}
{\bf Apgalvojums:} Ja $q$ nedalās ar $2$ un $5$, tad racionāla skaitļa $p/q$ ir tīri periodiska decimāldaļa
(bez priekšperioda). Eilera funkcija $\varphi(q)$ dalās ar ciparu skaitu periodā. &
$1/41=0.(02439)$. $\varphi(41)=40$ dalās ar $5$.\newline
$1/13=0.(076923)$. $\varphi(13)=12$ dalās ar $6$. Bet $1/12=0.08(3)$ satur priekšperiodu. \\ \hline
{\bf Definīcija:} Naturāla skaitļa $n$ pozitīvo dalītāju skaitu apzīmē ar $d(n)$, 
pozitīvo dalītāju summu - ar $\sigma(n)$, pozitīvo dalītāju 
kvadrātu summu - ar $\sigma_2(n)$.
$d(n)$, $\sigma(n)$, $\sigma_2(n)$ ir multiplikatīvas funkcijas. &
Ja $n=p_1^{a_1}p_2^{a_2}$, tad $d(n)=(a_1+1)(a_2+1)$,\newline 
$\sigma(n)=\left(1+p_1^1+\cdots+p_1^{a_1}\right)\left(1+p_2^1+\cdots+p_2^{a_2}\right)$. \\ \hline
\multicolumn{2}{|p{18.05cm}|}{
\cellcolor[HTML]{E1FFE1}
{\bf Apgalvojums:}
$d(1)+d(2)+\cdots+d(n)=\left\lfloor \frac{n}{1} \right\rfloor + 
\left\lfloor \frac{n}{2} \right\rfloor + \cdots + 
\left\lfloor \frac{n}{n} \right\rfloor$.\newline
$\sigma(1)+\sigma(2)+\cdots+\sigma(n)=1\cdot\left\lfloor \frac{n}{1} \right\rfloor + 
2\cdot\left\lfloor \frac{n}{2} \right\rfloor + \cdots + 
n\cdot\left\lfloor \frac{n}{n} \right\rfloor$. 
} \\ \hline
\multicolumn{2}{|p{18.05cm}|}{
{\bf Definīcija:} Ja $p$ ir pirmskaitlis, tad par naturāla skaitļa $n$ $p$-valuāciju sauc lielāko 
pakāpi $p^a$, ar kuru dalās $n$. Apzīmē $\nu_p(n) = a$. Skaitlim $0$ valuācijas nedefinētas, tas dalās ar jebko.
Grieķu burtu $\nu$ lasa ``nī'' (angl. ``nu'' [nju:]).   
} \\ \hline
{\bf Ležandra (Legendre) formula:} Ja $p$ ir pirmskaitlis, tad jebkuram naturālam $n$ 
$\nu(n!)=\left\lfloor \frac{n}{p^1} \right\rfloor + 
\left\lfloor \frac{n}{p^2} \right\rfloor +
\left\lfloor \frac{n}{p^3} \right\rfloor + \cdots$. &
Augstākā pakāpe $5^k$, ar ko dalās $100!$ ir $\lfloor 100/5 \rfloor + \lfloor 100/25 \rfloor = 24$. 
Tādēļ $100!$ decimālpieraksts beidzas ar $24$ nullēm. \\ \hline
{\bf Kummera (Ernst Kummer) teorēma:} Ja $p$ ir pirmskaitlis un $n \geq m \geq 0$, tad 
$\nu_p\left(C^m_n\right)=\nu_p\left( \frac{n!}{m!(n-m)!} \right)$
vienāds ar pārnesumu skaitu, stabiņā saskaitot $m$ un $n-m$, pierakstīti skaitīšanas
sistēmā ar bāzi $p$. &
$C^2_8$ dalās ar $2^2$, bet ne ar $2^3$, jo $2=10_2$ un $6=110_2$ saskaitīšanā 
$10_2+110_2=1000_2$ ir divi pārnesumi. \\ \hline 
{\bf Kāpinātāja pacelšanas (Lifting the exponent, LTE) lemma 1:} Ja $x$ un $y$ ir veseli skaitļi (ne obligāti pozitīvi),
$n$ ir naturāls skaitlis un $p$ ir nepāru pirmskaitlis, kuram $x-y$ dalās ar $p$,
bet ne $x$, ne $y$ nedalās ar $p$, tad
$\nu_p\left( x^n - y^n \right) = \nu_p(x - y) + \nu_p(n)$. &
$\nu_3(10^9 - 1^9) = \nu_3(10-1)+\nu_3(9)=2+2=4$. Pārbaudām: $999999999=1001001 \cdot 111 \cdot 9$. 
Skaitlis $999999999$ dalās ar $3^4$, bet ne ar $3^5$. \\ \hline
\multicolumn{2}{|p{18.05cm}|}{
\cellcolor[HTML]{E1FFE1}
{\bf BW.2015.16} Ar $P(n)$ apzīmējam lielāko pirmskaitli, ar ko dalās $n$. Atrast
visus naturālos skaitļus $n \geq 2$, kam $P(n)+\lfloor \sqrt{n} \rfloor = P(n+1)+\lfloor \sqrt{n+1} \rfloor$. 
} \\ \hline
{\bf LTE lemma 2:} Ja $x$ un $y$ ir veseli skaitļi (ne obligāti pozitīvi),
$n$ ir nepāru naturāls skaitlis un $p$ ir nepāru pirmskaitlis tāds, ka $p\,\mid\,x+y$,
bet ne $x$ ne $y$ nedalās ar $p$, tad
$\nu_p\left( x^n + y^n \right) = \nu_p(x + y) + \nu_p(n)$. &
$\nu_{11}(10^{121}+1)=\nu_{11}(10+1) + \nu_{11}(121)=1+2=3$. 
Skaitlis $1\underbrace{0\ldots0}_{120}1$ dalās ar $11^3$, bet ne ar $11^4$. \\ \hline
{\bf LTE lemma 3:} Ja $x$ un $y$ ir nepāru skaitļi, kam $x-y$ dalās ar $4$, tad 
$\nu_2(x^n-y^n) = \nu(x − y) + \nu_2(n)$. &
$\nu_2(5^{128} - 1) = 2+7 = 9$ \\ \hline
\multicolumn{2}{|p{18.05cm}|}{
\cellcolor[HTML]{E1FFE1}
{\bf LV.TST.1993.2} Dots naturāls skaitlis $a > 2$. Pierādīt, ka eksistē tikai galīgs skaits tādu naturālu 
$n$, ka $a^n-1$ dalās ar $2^n$.
} \\ \hline
\multicolumn{2}{|p{18.05cm}|}{
\cellcolor[HTML]{E1FFE1}
{\bf BW2015.17} Atrast visus naturālos skaitļus $n$, kuriem $n^{n−1}−1$ dalās ar
$2^{2015}$, bet nedalās ar $2^{2016}$.
} \\ \hline
{\bf LTE lemma 4:}  Ja $x$ un $y$ ir divi nepāru veseli skaitļi
un $m$ ir pāru naturāls skaitlis. Tādā gadījumā:
$\nu_2(x^m - y^m) = \nu_2(x-y) + \nu_2(x+y) + \nu_2(m) - 1$. &
$\nu_2(3^{16} - 1) = 1+2+4-1=6$. \\ \hline
\multicolumn{2}{|p{18.05cm}|}{
\cellcolor[HTML]{E1FFE1}
{\bf LV.TST.1979.10.2} Pierādīt, ka eksistē tāds naturāls skaitlis $n$, ka $n^2+1$ dalās ar $5^{1979}$.
} \\ \hline
\multicolumn{2}{|p{18.05cm}|}{
{\bf Henzela (Hensel) lemma:} Ja polinomam $P(x)$ ir vienkārša sakne pēc kāda pirmskaitļa moduļa $p$, 
tad $P(x)$ būs vienkārša sakne arī pēc jebkuras šī pirmskaitļa pakāpes $p^k$, kuru var iegūt, pakāpeniski 
"paceļot" pakāpi. ($P(x)$ ir vienkārša sakne $x_0$ pēc moduļa $p$, 
ja $P(x_0) \equiv 0\;(\operatorname{mod}\,p)$, bet polinoma atvasinājuma vērtība $P'(x_0)$ ar $p$ vairs nedalās.)
} \\ \hline
\end{tabular*}
}
\end{table}




\end{document}


