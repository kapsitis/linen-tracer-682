\documentclass[a4paper,12pt]{article}

\usepackage{amsmath,amssymb,amsthm,multicol,tikz,enumitem}
\usepackage{hyperref}
\usepackage{fancyvrb}
%\usepackage{enumerate}
\usepackage[margin=2cm]{geometry}
\usetikzlibrary{calc,arrows.meta}

\newcommand\N{\mathbf{N}}
\newcommand\Q{\mathbf{Q}}
\newcommand\R{\mathbf{R}}
\newcommand\Z{\mathbf{Z}}

\def\ojoin{\setbox0=\hbox{$\bowtie$}%
  \rule[-.02ex]{.25em}{.4pt}\llap{\rule[\ht0]{.25em}{.4pt}}}
\def\leftouterjoin{\mathbin{\ojoin\mkern-5.8mu\bowtie}}
\def\rightouterjoin{\mathbin{\bowtie\mkern-5.8mu\ojoin}}
\def\fullouterjoin{\mathbin{\ojoin\mkern-5.8mu\bowtie\mkern-5.8mu\ojoin}}

% Comment out one or the other

\ifdefined\mysolution
  \newcommand\answer[1]{\mbox{}\\[-15pt]{\color{blue}{#1}}\hfill{\color{blue}$\qed$}\mbox{}\\[-15pt]} 
  \newcommand\ans[1]{{\color{blue}{#1}}}
\else 
  \newcommand\answer[1]{}
  \newcommand\ans[1]{}
\fi


\setlength{\parindent}{0pt}

\begin{document}

\thispagestyle{empty}

\begin{center}
{\bf\Huge Gala eksāmens} \\[5pt]
Lietišķie algoritmi, 2019.g.\ rudens\\
Termiņš: 2019-12-17
\end{center}

\hrule
\vspace{2pt}
\hrule
\vspace{12pt}


% https://learning.oreilly.com/library/view/operations-research-2nd/9789332537392/xhtml/chapter002-001.xhtml
% Example 5
\noindent
{\bf 1.uzdevums. LP uzdevums ar simpleksu metodi.}  
Maksimizēt $z = 6x_1 + 4x_2$, kur izpildās šādi nosacījumi:
$$\left\{ \begin{array}{l}
2x_1 + 3x_2 \leq 30\\
3x_1 + 2x_2 \leq 24\\
x_1 + x_2 \geq 3\\
x_1,x_2 \geq 0
\end{array} \right.$$

\vspace{6pt}
{\bf 2.uzdevums. Primārais un duālais LP uzdevumi.}
Dots primārais LP uzdevums: Maksimizēt $z = 2 x_1 + 5 x_2$, 
kur $3 x_1 + 7x_2  = 12$ un $x_1, x_2 \geq 0$. 
\begin{enumerate}[label=(\alph*)]
\item Kāds ir primārā LP mērķfunkcijas $2x_1 + 5x_2$ maksimumu un pie kuriem $x_i$ to sasniedz.
\item Formulēt duālo LP uzdevumu. 
\item Atrast duālā uzdevuma mērķfunkcijas minimumu un kādiem mainīgajiem to sasniedz.
\end{enumerate}


\vspace{6pt}
{\bf 3.uzdevums. Meklēšanas algoritmu salīdzināšana.} Meklējam paraugu $P = \mathtt{abcbcab}$
tekstā $T = \mathtt{abcabbcabcbcababababcbcab}$ (sk.\ 4.\ mājasdarba 4.\ uzdevumu). 
\begin{enumerate}[label=(\alph*)]
\item Izveidot Knuta-Morisa-Prata algoritmam nepieciešamo prefiksu funkciju $\pi$. 
\item Atrast, cik reizes tekstā $T$ ielūkojas
Knuta-Morisa-Prata algoritms. Attēlot to tabuliņā \textendash{} 
tās augšējā rinda ir pats teksts $T$. 
Visas nākamās rindiņas parāda paraugu $P$ (kas nobīdīts
atbilstoši ikreizējai KMP algoritma hipotēzei). 
Šajās rindiņās vajag apvilkt visus tos parauga $P$ 
simbolus, kas tika salīdzināti ar teksta $T$ simboliem. 
\item Atrast, cik reizes tekstā $T$ ielūkojas
Bojera-Mūra algoritms. Arī attēlot to tabuliņā, kur 
redzamas visas parauga $P$ nobīdes un apvilkti simboli, 
kuri tika salīdzināti ar teksta $T$ simboliem. 
\end{enumerate}

\vspace{6pt}
{\bf 4.uzdevums. Kļūdu labošana $[7,4,1]$ Heminga kodā:} 
Mums jānosūta četru bitu vektors $\mathbf{x}=(x_1,x_2,x_3,x_4)^T$ 
pa trokšņainu sakaru kanālu, izmantojot kļūdu korekcijas kodu. 
Heminga kodu  $[7,4,1]$ aprēķina, reizinot $\mathbf{x}$ ar 
ģeneratormatricu
$$G\mathbf{x} = \left(
\begin{array}{cccc}
1 & 0 & 0 & 0 \\
0 & 1 & 0 & 0 \\
0 & 0 & 1 & 0 \\
1 & 1 & 1 & 0 \\
0 & 0 & 0 & 1 \\
1 & 1 & 0 & 1 \\
1 & 0 & 1 & 1 
\end{array} \right) \left( \begin{array}{c}
x_1\\
x_2\\
x_3\\
x_4
\end{array} \right).$$
\begin{enumerate}[label=(\alph*)]
\item Alise grib nosūtīt Bobam vektoriņu $\mathbf{x} = (1,0,1,0)^T$, izmantojot šo Heminga kodu.
Uzrakstīt Alises pusē iegūto Heminga kodējumu.
\item Sakaru kanāls sabojāja pašu pirmo un pašu pēdējo bitu ("0" vietā Bobs saņēma "1",
bet "1" vietā "0"). Uzrakstīt, kādu vektoriņu saņēma Bobs.
\item Bobs pieņēma, ka sakaru kanāls ir sabojājis ne vairāk kā vienu bitu un atkodēja
saņemto vektoriņu, izmantojot Heminga $[7,4,1]$ kļūdu labošanas algoritmu. 
Uzrakstīt, kāds izskatās Boba pusē atkodētais ziņojums.
Kurš bits Bobam izskatās kļūdains?
\end{enumerate}




\end{document}



