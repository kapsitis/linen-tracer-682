\documentclass[11pt]{article}
\usepackage{ucs}
\usepackage[utf8x]{inputenc}
\usepackage{changepage}
\usepackage{graphicx}
\usepackage{amsmath}
\usepackage{gensymb}
\usepackage{amssymb}
\usepackage{enumerate}
\usepackage{tabularx}
\usepackage{lipsum}

\oddsidemargin 0.0in
\evensidemargin 0.0in
\textwidth 6.27in
\headheight 1.0in
\topmargin -0.1in
\headheight 0.0in
\headsep 0.0in
%\textheight 9.69in
\textheight 9.50in

\setlength\parindent{0pt}

\newenvironment{myenv}{\begin{adjustwidth}{0.4in}{0.4in}}{\end{adjustwidth}}
\renewcommand{\abstractname}{Anotācija}
\renewcommand\refname{Atsauces}

\newenvironment{uzdevums}[1][\unskip]{%
\vspace{3mm}
\noindent
\textbf{#1:}
\noindent}
{}

\newcommand{\subf}[2]{%
  {\small\begin{tabular}[t]{@{}c@{}}
  #1\\#2
  \end{tabular}}%
}



\newcounter{alphnum}
\newenvironment{alphlist}{\begin{list}{(\Alph{alphnum})}{\usecounter{alphnum}\setlength{\leftmargin}{2.5em}} \rm}{\end{list}}


\makeatletter
\let\saved@bibitem\@bibitem
\makeatother

\usepackage{bibentry}
%\usepackage{hyperref}


\begin{document}

\thispagestyle{empty}

\begin{uzdevums}[IMO.SHL.2014.N6]
Ar $a_1 < a_2 <  \cdots <a_n$ apzīmējam naturālus skaitļus, kas ir 
savstarpēji pirmskaitļi. Turklāt $a_1$ ir pirmskaitlis un
$a_1 \geq n + 2$. Reālās taisnes nogrieznī 
$I = [0, a_1 a_2  \cdots a_n ]$ 
nokrāsojam sarkanus visus veselos skaitļus, kas dalās ar vismaz vienu no 
skaitļiem
$a_1, \ldots, a_n$. Sarkanie punkti sadala nogriezni $I$ mazākos nogriežņos.
Pierādīt, ka šo nogriežņu garumu kvadrātu summa dalās ar $a_1$. 
\end{uzdevums}

{\em Pierādījums.} \\
Apzīmējam reizinājumu $A=a_1\cdot\ldots\cdot{}a_n$. 
Intervāla $X$ garumu apzīmējam ar $|X|$.

Ar $\mathcal{S}$ apzīmējam visus intervālus 
$[x,y] \subseteq [0,A]$, $x < y$, kam abi galapunkti 
bija sarkani.

Ar $\mathcal{T}$ apzīmējam visus  
tos intervālus ar veseliem galapunktiem, kam $[x,y] \subseteq [0,A]$
un kuru iekšpusē nav neviena sarkanā punkta. (Šo intervālu skaitu un garumus
būs vieglāk novērtēt nekā tos, kurus mums vajag - no kopas $\mathcal{S}$.

Mums jāpamato, ka summa ${\displaystyle \sum_{X \in \mathcal{S}} |X|^2}$ 
dalās ar $p$. Piekārtosim katram intervālam $Y \in \mathcal{T}$ svaru
$w(|Y|)$, kas atkarīgs tikai no tā garuma:

$$w(|Y|) = \left\{
\begin{array}{ll}
1, & \mbox{ja $|Y| = 1$} \\
2, & \mbox{ja $|Y| \geq 2$}
\end{array} \right.$$

Apskatīsim jebkuru intervālu $X \in \mathcal{S}$; un tos intervālus 
no $\mathcal{T}$, kuri tajā ietilpst. Intervāls $X$ saturēs vienu
intervālu $Y \in \mathcal{T}$ garumā $|X|$, divus intervālus garumā
$|X|-1$, utt., visbeidzot $|X|$ intervālus no $\mathcal{T}$ garumā $1$. 
Visu šo apakšintervālu svaru summa:
$$\sum_{Y \subseteq X} w(|Y|) = (1+2+\ldots+(|X|-1))\cdot 2 + |X| \cdot 1 = |X|^2.$$

Tā kā intervāli no $\mathcal{S}$ nepārklājas, katrs intervāls $Y \in \mathcal{T}$
pieder tieši vienam no tiem. Lai atrastu visu intervālu $X \in  \mathcal{S}$ kvadrātu summu $|X|^2$, 
tai vietā saskaitīsim svarus intervāliem $Y \in \mathcal{T}$. 

Katram iespējamam garumam $d = 1,\ldots,a_1$, noskaidrosim, cik ir intervālu no $\mathcal{T}$ garumā
tieši $d$. Tātad - cik ir tādu veselu $x \in [0;A-1]$ vērtību, kam $[x,x+d]$ nesatur nevienu sarkanu punktu?
Šai nolūkā dalām $x$ ar visiem skaitļiem $a_1,a_2,\ldots,a_n$ un iegūstam atlikumus $r_1,r_2,\ldots,r_n$. 
Pēc ķīniešu atlikumu teorēmas, katram $x$ šis komplekts ar atlikumiem $(r_1,\ldots,r_n)$ būs cits 
(un arī katram atlikumu komplektam atbilst noteikts $x \in [0;A)$). 

Bez sarkanajiem punktiem būs tie nogriežņi $[x;x+d]$, kuriem $r_i + d \leq a_i$ (katram $i=1,\ldots,n$). 
Pēc reizināšanas likuma, šādu atlikumu būs 
$$f(d) = (a_1 + 1 - d)\cdot(a_2 + 1 - d)\cdot \ldots \cdot (a_n + 1-d).$$

Izmantojot apzīmējumu $f(d)$, varam izteikt tālāk:
$$\sum_{X \in \mathcal{S}} |X|^2 = \sum_{Y \in \mathcal{T}} w(|Y|) = 
2\sum_{d=1}^{a_1} f(d) - f(1).$$

Viegli redzēt, ka $f(1)=a_1a_2\ldots{}a_n$ dalās ar $a_1$.
Savukārt summa $\sum f(d)$ ir $n$-tās pakāpes polinoms attiecībā pret mainīgo $d$. 
Tā kā summēšana notiek pa visām $a_1$ kongruences klasēm 
(no $1$ līdz $a_1$ ieskaitot), tad varam pamatot, ka tā dalās ar $a_1$, 
izmantojot sekojošo Lemmu. Tātad arī $\sum |X|^2$ dalās ar $a_1$. 


{\bf Lemma.} Ja $p$ ir pirmskaitlis, $F(x)$ ir polinoms ar veseliem koeficientiem, 
kura pakāpe nepārsniedz $p-2$, tad $\sum_{x=1}^p F(x)$ dalās ar $p$. 

{\em Pierādījums.} Pietiek pamatot šo rezultātu visiem $F(x)=x^k$, kur 
$k \leq p-2$. Pierāda pēc indukcijas. Ja $k=0$, tad summa ir vienāda 
ar $p$ - tātad dalās ar $p$.

Izvēlamies $k \leq p-2$ un pieņemam, ka visām mazākām pakāpēm lemma ir spēkā. 
Tad
$$0 \equiv p^{k+1} = \sum_{x=1}^p \left( x^{k+1} - (x-1)^{k+1} \right) \equiv 
(k+1)\sum_{x=1}^p x^k\;(\mbox{mod}\,p)$$
Tā kā $0 < k+1 < p$, tad ar $(k+1)$ var noīsināt un iegūt 
$0 \equiv \sum_{x=1}^p x^k\;(\mbox{mod}\,p)$. $\square$









\newpage


\begin{uzdevums}[IMO.SHL.2014.N7]
Dots naturāls skaitlis $c \ge 1$. Definējam naturālu skaitļu 
virkni ar vienādībām $a_1 = c$ un
$$a_{n+1}=a_n^3-4c\cdot a_n^2+5c^2\cdot a_n+c$$ 
visiem $n \geq 1$. 
Pierādīt, ka jebkuram naturālam $n \geq 2$ eksistē
pirmskaitlis $p$, ar kuru dalās $a_n$, bet nedalās 
neviens no skaitļiem $a_1,\ldots,a_{n-1}$.
\end{uzdevums}

\vspace{10pt}
{\em Pierādījums.} Definējam $x_0 = 0$ un $x_n = a_n/c$ visiem 
$n \geq 1$. Tad jaunā virkne $(x_n)$ izpilda šādu rekurentu sakarību:
$$x_{n+1} = c^2\left( x_n^3 - 4x_n^2 + 5x_n \right) + 1$$
visiem veseliem $n \geq 0$. Šī sakarība parāda arī, ka visi 
virknes locekļi ir naturāli skaitļi 
(piemēram, $x_1 = 1$ un $x_2 = 2c^2 + 1$). No šīs sakarības var arī pamatot, 
ka virkne ir stingri augoša ($x_{n+1} > x_n$) - piemēram, 
iznesot pirms iekavām $x_n$ un atdalot pilno kvadrātu. 

Pirmie locekļi ($x_1$, $x_2$) ir savstarpēji pirmskaitļi. 
Vēlamies pamatot, ka arī lielākiem $n$ ($n \geq 2$) 
eksistēs pirmskaitlis $p$, kas ir $x_n$ dalītājs, 
bet nedala nevienu no skaitļiem $x_1,\ldots,x_{n-1}$. 
Šajā nolūkā pamatosim trīs apgalvojumus. 

{\bf Apgalvojums 1:} Ja $i\equiv{}j\;(\mbox{mod}\,m)$ izpildās
kaut kādiem $i,j \geq 0$ un $m \geq 1$, tad izpildās arī
$x_i \equiv x_j\;(\mbox{mod}\,x_m)$. 

{\em Pierādījums.} Pamatosim, 
ka $x_i$ un $x_{i+m}$ dod vienādus
atlikumus, dalot ar $x_m$. (T.i. apskatām gadījumu, kad $j-i = m$. 
Gadījumi, kad $j-i = 2m, 3m, \ldots$ iegūstami, izejot atlikumu ciklu 
divas, trīs vai vairāk reizes.)\\
Indukcijas bāze $i = 0$: Tad jebkuram fiksētam $m$ būs 
spēkā $x_0 \equiv x_m\;(\mbox{mod}\,x_m)$, jo $x_0=0$.\\
Induktīvā pāreja: $i \rightarrow i+1$. Pieņemsim, ka 
$x_{i+m} \equiv x_i\;(\mbox{mod}\,m)$ kādam $i$. 
Tad rekursīvā sakarība, kas izsaka $x_{i+m+1}$ no $x_{i+m}$ un 
$i_{i+1}$ no $x_i$ parāda, ka arī $x_{i+m+1}$ un $x_{i+1}$ ir
kongruenti pēc $x_m$ moduļa. $\square$

{\bf Apgalvojums 2.} 
Ja veseli skaitļi $i,j \geq 2$ un $m \geq 1$ apmierina 
sakarību $i \equiv j\;(\mbox{mod}\,m)$, tad 
ir spēkā arī $x_i \equiv x_j\;(\mbox{mod}\,x_m^2)$. 

{\em Pierādījums.} Pietiek parādīt, ka $x_{i+m} \equiv x_i\;(\mbox{mod}\,x_m^2)$
visiem veseliem $i \geq 2$ un $m \geq 1$. Indukcijas pāreja var izmantot 
rekurento sakarību, kas $x_{n+1}$ izsaka ar $x_n$, bet šoreiz 
indukcijas bāze ($i=2$) ir grūtāka. 

Apzīmējam $L= 5c^2$. Tad pēc $x_n$ rekurentās sakarības būs arī
$x_{m+1} \equiv Lx_m + 1\;(\mbox{mod}\,x_m^2)$. Un tātad:
$$x_{m+1}^3 - 4x_{m+1}^2 + 5x_{m+1} \equiv (Lx_m + 1)^3 - 4(Lx_m + 1)^2 + 5(Lx_m+1)
\equiv 2\;(\mbox{mod}\,x_m^2).$$
No šejienes savukārt seko, ka 
$x_{m+2} \equiv 2c^2 + 1 \equiv x_2\;(\mbox{mod}\,x_m^2)$. $\square$


{\bf Apgalvojums 3.} Katram veselam skaitlim $n \geq 2$, ir sp'ek'a 
$x_n > x_1 \cdot x_2 \cdot \ldots \cdot x_{n-2}$. 

{\em Pierādījums.} Pēc indukcijas pa $n$. Gadījumi $n=2$ un $n=3$ ir vienkārši. 
Pieņemsim, ka apgalvojums spēkā kādam $n \geq 3$. Tad 
$$x_{n+1} > x_n^3 - 4x_n^2 + 5x_n > 7x_n^2 - 4x_n^2 > x_n^2 > x_nx_{n-1},$$
kas kopā ar indukcijas hipotēzi dod vajadzīgo apgalvojumu. $\square$

Visbeidzot - pēc Apgalvojuma 3, atradīsies pirmskaitlis $p$, 
kura kāpinātājs $x_n$ sadalījumā pirmreizinātājos ir augstāks 
nekā tā kāpinātājs  reizinājumā $x_1 \cdot x_2 \cdot \ldots \cdot x_{n-2}$. 
Pamatosim, ka šis pirmskaitlis nevar dalīt nevienu skaitli 
$x_1,\ldots,x_{n-1}$. 

No pretējā - pieņemsim, ka $k$ ir mazākais koeficients, kam $x_k$ dalās ar $p$.
Tā kā $x_{n-1}$ un $x_n$ ir savstarpēji pirmskaitļi un $x_1=1$, tad 
šis minimālais koeficients izpilda $2 \leq k \leq n-2$. Izsakām 
$n = qk+r$ (dalījums ar atlikumu). Pēc Apgalvojuma 1 
ir jāizpildās $x_n \equiv x_r\;(\mbox{mod}\,x_k)$, tātad
$p$ dala arī $x_r$. Tā kā $k$ bija minimālais, ir jābūt $r=0$
(t.i. $n$ dalās ar $k$). 

Pēc Apgalvojuma 2 secinām, ka $x_n \equiv x_k\;(\mbox{mod}\,x_k^2)$. 
Apzīmējam $\alpha = \nu_p(x_k)$ - skaitļa $x_k$ $p$-valuācija. 
Tātad $x_k^2$ un arī $x_n$ dalās ar $p^{\alpha+1}$. Iegūta 
pretruna, jo $x_n$ bija jādalās ar augstāku pirmskaitļa $p$ pakāpi 
nekā jebkuram $x_k$ ($k<n$). $\square$


\newpage

\begin{uzdevums}[IMO.SHL.2018.N6]
Dota $f\,:\,\{1,2,3,\ldots\}\,\rightarrow\{2,3,\ldots\}$, 
funkcija, kas apmierina sakarību
$f(m+n)\,\mid\,f(m)+f(n)$ ($f(m+n)$ ir $f(m)+f(n)$ dalītājs)
visiem naturālu skaitļu pāriem $m,n$. Pierādīt, ka 
eksistē naturāls skaitlis $c>1$, kurš ir visu 
$f$ vērtību dalītājs.
\end{uzdevums}

{\em Atrisinājums.}\\
Katram naturālam $m$ apzīmēsim ar $S_m$ visu to argumentu $n$ kopu, 
kam $f(n)$ dalās ar $m$. 

{\bf Lemma.} Ja $S_m$ ir bezgalīga, tad $S_m = \{ d, 2d, 3d,\ldots\}$, 
t.i. satur tieši skaitļa $d$ daudzkārtņus.

{\em Pierādījums.} Apzīmējam $d = \mbox{min} S_m$; skaitlis $m$ ir $f(d)$ 
dalītājs. Ja $n \in S_m$ un $n > d$, tad 
$$m\,\mid\,f(n)\;\;\mbox{un}\;\;f(n)\,\mid\,f(n-d) + f(d),$$
tātad $m$ dala arī $f(n-d)$ un $n-d \in S_m$. Pēc indukcijas arī
$n-2d, n-3d,\ldots \in S_m$. Tā kā $m$ bija mazākais $S_m$ elements, 
tad nevar rasties pozitīvs atlikums $r\in (0;d)$ (ja no $n$ atņem 
pietiekami daudzus $d$). Tāpēc $n$ dalās ar $d$ bez atlikuma. $\square$


{\bf Apgalvojums 1.} Ja funkcija $f$ ir ierobežota (t.i. 
tās vērtības nepārsniedz kādu fiksētu naturālu skaitli), tad
visas šīs vērtības dalās ar vienu un to pašu pirmskaitli. 

{\em Pierādījums.} Ir tikai galīgs skaits pirmskaitļu, kuri 
dala kaut vienu $f(n)$ vērtību. Starp tiem varētu 
būt tādi pirmskaitļi ("sarkanie"), kuri ir $f(n)$ dalītāji
galīgi daudziem argumentiem $n$ (un visi citi - "zilie", kuri ir $f(n)$
dalītāji bezgalīgi daudziem $n$). 

\begin{enumerate}
\item Apzīmēsim ar $N$ tik lielu naturālu skaitli, lai tas pārsniegtu visus
tos $n_i$, kam $f(n_i)$ dalās ar kādu "sarkanu" pirmskaitli. 
\item Katram no "zilajiem" pirmskaitļiem $p_1,\ldots,p_k$ eksistēs 
tāds $d_1,\ldots,d_k$, ka $S_{p_i}$ satur tieši visus $d_i$ daudzkārtņus. 
\end{enumerate}

Konstruējam jaunu skaitli:
$$n^{\ast} = N\cdot{}d_1\cdot{}d_2\cdot\ldots\cdot{}d_k + 1.$$
Visi $f(n^{\ast})$ dalītāji ir zilie pirmskaitļi, jo $n^{\ast}>N$. 
Ar $p_i$ apzīmēsim kādu zilo pirmskaitli, kas ir 
$f(n^{\ast})$ dalītājs. Tad $n^{\ast} \in S_{p_i}$ un tātad
$d_i$ ir $n^{\ast}$ dalītājs (jo ir spēkā Lemma). 

Bet tanī pašā laikā $n^{\ast}$ dod atlikumu $1$, dalot ar $d_i$, 
tātad $d_i = 1$. Tas nozīmē, ka $S_{p_i}$ satur visus naturālos skaitļus
un tātad $p_i$ dala visas $f(n)$ vērtības. $\square$



{\bf Apgalvojums 2.} Ja funkcija $f$ ir neierobežota, tad $f(1) = a$ 
ir jebkuras vērtības $f(n)$ dalītājs. 

{\em Pierādījums.} Tā kā $1 \in S_a$, tad saskaņā ar lemmu pietiek pamatot, 
ka $S_a$ ir bezgalīga. 

Sauksim naturālu skaitli $q$ par "lokālu maksimumu", ja 
$f(q)$ ir lielāks nekā jebkura no iepriekšējām vērtībām 
($f(1),\ldots,f(q-1)$). 
Tā kā $f$ ir neierobežota, šādu lokālo maksimumu būs bezgalīgi daudz. 
Apzīmēsim visu šo maksimumu virkni ar $1=q_1 < q_2<\ldots$, un 
$h_k = f(q_k)$. katram no maksimumiem $q_i$ un katram 
$k < q_i$ izpildās 
$f(q_i)\,\mid\,f(k)+f(q_i-k)<2f(q_i)$, tātad
$$f(k)+f(q_i-k) = f(q_i) = h_i.$$

Pēc Dirihlē principa, starp skaitļiem $h_1,h_2,\ldots$ ir bezgalīgi 
daudzi, kas kongruenti pēc $a$ moduļa. Šo apakšvirkni apzīmējam ar 
$h_{k_0}\equiv h_{k_1} \equiv \ldots\;(\mbox{mod}\,a)$. Tad
$$f(q_{k_i} - q_{k_0}) = f(q_{k_i}) - f(p_{k_0}) = h_{k_i} - h_{k_0}$$
t.i. $q_{k_i} - q_{k_0}$ pieder $S_a$.

Tādēļ kopā $S_a$ ir bezgalīgi daudz elementu un $f(1) = a$ ir jebkura $f(n)$ 
dalītājs. $\square$



\newpage

\begin{uzdevums}[IMO.SHL.2018.N7]
Dots vesels skaitlis $n \geq 2018$ un 
$a_1,a_2,\ldots,a_n,b_1,b_2,\ldots,b_n$
ir pa pāriem dažādi naturāli skaitļi, kas 
nepārsniedz $5n$. Pieņemsim, ka virkne
$$\frac{a_1}{b_1},\frac{a_2}{b_2},\ldots,\frac{a_n}{b_n}$$
veido aritmētisku progresiju. Pierādīt, ka visi virknes locekļi 
ir savā starpā vienādi.
\end{uzdevums}

{\em Atrisinājums:}\\
Pieņemsim no pretējā, ka $\frac{a_1}{b_1},\ldots$ ir aritmētiska
progresija un tās diference ir $\Delta = \frac{c}{d} > 0$, kas
izteikta kā nesaīsināma daļa. 

Atradīsim, cik daudzi no saucējiem $b_i$ dalās ar $d$. 
Šai nolūkā katram $i \in [1;n]$ un katram skaitļa 
$d$ pirmreizinātājam $p$ teiksim, ka indekss $i$ ir 
$p$-savāds, ja $b_i$ dalās ar zemāku pirmskaitļa $p$ 
pakāpi nekā $d$, t.i. $\nu_p(b_i) < \nu_p(d)$
(kur $\nu_p(x)$ apzīmē skaitļa $x$ $p$-valuāciju - kāpinātāju 
pie $p$, kur $x$ sadalīts pirmreizinātājos). 

{\bf Apgalvojums 1.} Katram pirmskaitlim $p$, visi $p$-savādie
indeksi atšķiras par $p$ daudzkārtni (t.i. viņi visi pieder 
kaut kādai aritmētiskai progresijai ar difierenci $p$). 

{\em Pierādījums:} Pieņemsim no pretējā, ka $i$ un $j$ abi ir
$p$-savādi (un nedalās ar $p^{\alpha}$, kur $d$ satur pirmreizinātāju 
$p$ precīzi pakāpē $\alpha$), turklāt $i-j$ nedalās ar $p$.
Šajā gadījumā arī daļu $\frac{a_i}{b_i}$ un $\frac{a_j}{b_j}$
mazākais kopsaucējs nedalās ar $p^{\alpha}$. 
Tas nav iespējams, jo šo daļu starpība ir $(i-j)\Delta = \frac{(i-j)c}{d}$ --
nesaīsināma daļa, kuras saucējs dalās ar $p^{\alpha}$. 
Pretruna. $\square$

{\bf Apgalvojums 2.} Skaitlim $d$ nav pirmreizinātāju, kas 
pārsniedz $5$. 

{\em Pierādījums:} Pieņemsim, ka $d$ dalās ar pirmskaitli $p \geq  7$. 
No indeksiem $1,2,\ldots,n$ ne vairāk kā 
$\left\lceil \frac{n}{p} \right\rceil < \frac{n}{p} + 1$ ir $p$-savādi. 
Tātad $p$ dala visus pārējos no skaitļiem $b_1,\ldots,b_n$, kuru pavisam 
ir $n - \left( \frac{n}{p} + 1 \right)$. Tā kā viņi visi ir dažādi, 
lielākais no tiem ir vismaz
\[ \left( \frac{p-1}{p}\cdot{}n - 1\right)p  = pn-n-p = (p-1)(n-1) 
\geq 6(n-1) > 5n, \]
kas ir pretrunā ar nosacījumu. $\square$

{\bf Apgalvojums 3.} Starp jebkuriem $30$ pēc kārtas sekojošiem daļu 
saucējiem $b_{k+1},b_{k+2},\ldots,b_{k+30}$ vismaz $8$ ir tādi, kas dalās ar $d$. 

{\em Pierādījums:} Ir pavisam $\varphi(30)=8$ (Eilera funkcija) 
dažādi atlikumi, dalot ar $30$, kuri 
nepieder tām aritmētiskajām progresijām ar diferencēm $2$, $3$ vai $5$, kas
varētu būt $2$-savādas, $3$-savādas vai $5$-savādas. (Tie indeksi, kuri nav savādi, 
satur pirmreizinātājus $2,3,5$ vismaz tādā pašā pakāpē kā $d$ - tātad
attiecīgie $b_i$ dalās ar $d$.)

{\bf Apgalvojums 4.} $|\Delta| < \frac{20}{n-2}$ un tātad saucējs $d$ 
daļskaitlī $\Delta=c/d$ ir lielāks par apgriezto lielumu $\frac{n-2}{20}$. 

{\em Pierādījums.} Starp visām daļām $\frac{a_i}{b_i}$ aplūkosim 
tikai tās daļas, kurām saucēji $b_i \geq n/2$. Tādu daļu ir 
vismaz $n-n/2 = n/2$, katrs saucējs ir vismaz $n/2$, bet attiecīgās
daļas skaitītājs ir $a_i \leq 5n$. Neviena šāda attiecība
$\frac{a_i}{b_i}$ nepārsniedz $\frac{5n}{n/2}=10$. 
Tātad visas šīs daļas (izņemot izmestās ar ļoti maziem 
saucējiem) izpilda $\frac{a_i}{b_i} \in (0;10]$

Aritmētiskās progresijas diference $\Delta$ nevar pārsniegt $\frac{10}{n/2 - 1} = 
\frac{20}{n-2}$, jo citādi $n/2$ reizes pa šo diferenci paejot uz priekšu, 
progresija "izlīstu" no intervāla $(0;10]$. $\square$

{\bf Secinājums.} Skaitlis $\Delta = c/d$ nevar būt pozitīvs.

{\em Pierādījums.} Ir vismaz $\lfloor n/30 \rfloor \cdot 8$ dažādi 
$b_i$, kuri (pēc Apgalvojuma 3) dalās ar $d$. Tātad vislielākais no 
tiem ir vismaz
$$ \left( \left\lfloor \frac{n}{30} \right\rfloor \cdot 8 \right) \cdot d = 
\left( \frac{n}{30} - 1 \right) \cdot 8 \cdot \frac{n-2}{20} > 5n,$$
ja $n$ vietā ievieto $n \geq 2018$, jo $\frac{n-2}{20}$ ir lielāks par $100$. 
Pretruna, jo $(\forall i)(b_i < 5n)$. $\square$

\end{document}
