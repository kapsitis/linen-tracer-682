\documentclass[jou]{apa6}

\usepackage[american]{babel}
\usepackage{hyperref}
\usepackage{amsthm}
\usepackage{thmtools}

\usepackage{csquotes}
\usepackage[style=apa,sortcites=true,sorting=nyt,backend=biber]{biblatex}
\DeclareLanguageMapping{american}{american-apa}
\addbibresource{bibliography.bib}

\title{Lab 1: Boolean Expressions}

\author{Kalvis Aps\={\i}tis}
\affiliation{RBS}


\leftheader{Lab 1: Boolean Expressions}

\declaretheoremstyle[headfont=\normalfont\bfseries,notefont=\mdseries\bfseries,bodyfont = \normalfont,headpunct={:}]{normalhead}
\declaretheorem[name={Example}, style=normalhead,numberwithin=section]{problem}

\setcounter{section}{2}

\abstract{This document lists the key results from the December 14, 2019 class in 
Number Theory. This training for competition math is aimed at 16\textendash{}18 year 
olds (typically, Grades 10\textendash{}12).
}

\keywords{Chinese remainder theorem, Bezout's identity, Modular arithmetic.}

\begin{document}
\maketitle


\section{Objectives}

\begin{enumerate}
\item Run Coq command-line from the {\tt coqtop} utility and also {\tt CoqIDE}.
\item Check types in Coq, including atomic types, tuples, function types.
\item Use and build truth tables for Boolean operators.
\item Use precedence and associativity to parse Boolean expressions.
\item Create (abstract) syntax trees.
\item Fill in truth tables.
\item Check, if a Boolean expression is a tautology.
\item Check, if a Boolean expression is satisfiable.
\item Find equivalent Boolean expressions.
\item Rewrite expressions as CNF (conjunctive normal form).
\item Write your answers into a LaTeX template, compile and submit.
\end{enumerate}


\section{Datatypes}

If you need more information on datatypes, visit this site \url{https://coq.inria.fr/stdlib/Coq.Init.Datatypes.html}. 

Strong datatypes is not always convenient. 

But this attempt to evaluate expression is wrong:
\begin{verbatim}
Eval compute in if 1 = 2 then 3 else 4.
\end{verbatim}
The reason is displayed by Coq \textendash{} 

Define nandb. 



\section{Truth Tables}

Boolean expressions can always be computed using truth tables. 

\begin{tabular}{|l|l|l|} \hline
{\tt A} & {\tt B} & {\tt A \&\& B} \\ \hline
{\tt false} & {\tt false} & {\tt false} \\ \hline
{\tt false} & {\tt true} & {\tt false} \\ \hline
{\tt true} & {\tt false} & {\tt false} \\ \hline
{\tt true} & {\tt true} & {\tt true} \\ \hline
\end{tabular} 






\section{Precedence and Associativity}







\end{document}


